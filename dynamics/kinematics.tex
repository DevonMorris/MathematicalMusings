%!TEX TS-program = xelatex
%!TEX encoding = UTF-8 Unicode

\documentclass[a4paper]{article}

\usepackage{xltxtra}
\usepackage{amsfonts}
\usepackage{polyglossia}
\usepackage{fancyhdr}
\usepackage{geometry}
\usepackage{dsfont}
\usepackage{amsmath}
\usepackage{amsthm}

\geometry{a4paper,left=15mm,right=15mm,top=20mm,bottom=20mm}
\pagestyle{fancy}
\lhead{Devon Morris}
\chead{Dynamics}
\rhead{\today}
\cfoot{\thepage}

\setlength{\headheight}{23pt}
\setlength{\parindent}{0.0in}
\setlength{\parskip}{0.0in}

\newtheorem{prop}{Proposition}

\begin{document}

\section*{Kinematics}

\subsection*{Vectors}
Vectors will always be denoted by $\mathbf{q}$, using a bolded letter. Vectors always have a magnitude and direction (LIES, only in a normed space). Unit vectors can specify directions, such a vector $\hat{\mathbf{e}}$. We construct the vector such that $||\hat{\mathbf{e}}|| = 1$. We can rewrite $\mathbf{q} = q\hat{\mathbf{e}}$. Equivalently, $\hat{\mathbf{e}} = \frac{\mathbf{q}}{||\mathbf{q}||}$.

\subsection*{Coordinate Frames}
Set of three orthonormal vectors in a right handed fashion. To be right-handed we have $\hat{\mathbf{e}}_3 = \hat{\mathbf{e}}_1 \times \hat{\mathbf{e}}_2$. We can now write $\mathbf{q} = q_1 \hat{\mathbf{e}}_1 + q_2 \hat{\mathbf{e}}_2 + q_3 \hat{\mathbf{e}}_3$. We can sometimes write
\[
    \mathbf{q} = 
    \begin{bmatrix}
        q_1 \\
        q_2 \\
        q_3
    \end{bmatrix}
\]
when the basis is implied and we care more about computation on components.

\subsection*{Position, Velocity, Acceleration}
The position of a particle relative to the origin is denoted as $\mathbf{r}(t)$. We can write $\Delta \mathbf{r} = \mathbf{r}(t + \Delta t) - \mathbf{r}(t)$. We define the velocity as 
\[
    \mathbf{v} = \frac{d\mathbf{r}}{dt} = \lim_{\Delta t \rightarrow 0} \frac{\Delta \mathbf{r}}{\Delta t} = \dot{\mathbf{r}}
\]
We can write this as $\mathbf{v} = ||\mathbf{v}||\hat{\mathbf{e}}_{\mathbf{v}}$. Repeating this process with the velocity vector we have

\[
    \mathbf{a} = \frac{d\mathbf{v}}{dt} = \lim_{\Delta t \rightarrow 0} \frac{\Delta \mathbf{v}}{\Delta t} = \dot{\mathbf{v}} = \ddot{\mathbf{r}} = \frac{d^2 \mathbf{r}}{dt^2}
\]
From the discussion in class we can see that there is some component in the direction of the path and some component normal to the path. Let's say we have a position vector given by 
\[
    \mathbf{r} = \sum_{i=1}^3 r_i \hat{\mathbf{e}}_i
\]
Thus we have that, by the product rule
\[
    \mathbf{v} = \sum_{i=1}^3 \dot{r}_i \hat{\mathbf{e}}_i + \sum_{i=1}^3 r_i \dot{\hat{\mathbf{e}}}_i
\]
Doing the same thing with acceleration gives us
\[
    \mathbf{a} = \sum_{i=1}^3 \ddot{r}_i \hat{\mathbf{e}}_i + 2\sum_{i=1}^3 \dot{r}_i\dot{\hat{\mathbf{e}}}_i + \sum_{i=1}^3 r_i \ddot{\hat{\mathbf{e}}}_i
\]

\subsubsection*{Example}
A coordinate frame $B$ is given by $x,y,z$ axes. Suppose we have a particle moving in this frame. The motion of particle $p$ is given by $\mathbf{r}_{p/B}(t) = 3t \hat{\mathbf{i}} + \cos t \hat{\mathbf{j}} + 5\hat{\mathbf{k}}$. Now we will find the velocity $\mathbf{v}_{p/B} = 3\hat{\mathbf{i}} + -\sin t \hat{\mathbf{j}}$ and acceleration $\mathbf{a}_{p/B} = -\cos t \hat{\mathbf{j}}$.

\subsubsection*{Example}
We have two frames $O$ and $B$ and a vector between them $\mathbf{r}_{B/O}$ and a vector to the particle $\mathbf{r}_{p/O}$. We want to find the point $p$ relative to a stationary frame. We have that $\mathbf{r}_{p/O} = \mathbf{r}_{B/O} + \mathbf{r}_{p/B}$ by using the derivatives we hae that $\mathbf{v}_{p/O} = \mathbf{v}_{B/O} + \mathbf{v}_{p/B}$ and $\mathbf{a}_{p/O} = \mathbf{a}_{B/O} + \mathbf{a}_{p/B}$. Let $\mathbf{r}_{B/O} = 4t^2 \hat{\mathbf{J}} + 5t^3\hat{\mathbf{K}}$, we have that $\mathbf{v}_{B/O} = 8t \hat{\mathbf{J}} + 15t^2 \hat{\mathbf{K}}$ and $\mathbf{a}_{B/O} = 8\hat{\mathbf{J}} + 30t\hat{\mathbf{K}}$. At this point, we assume that the two frames are aligned, so therefore we have that 

\[
\mathbf{r}_{p/O} = 3t\hat{\mathbf{I}} + (4t^2 + \cos t) \hat{\mathbf{J}} + (5t^3 + 5)\hat{\mathbf{K}}
\]

\subsubsection*{Terminology}

These three phrases are synonomous
\begin{itemize}
    \item "Relative To"
    \item "Respect To"
    \item "Reference To"
    
\end{itemize}
that mean the point that things are measured from.

The three phrases
\begin{itemize}
    \item "In terms of"
    \item "Described in"
    \item "Expressed in"
\end{itemize}
are synonomous and mean the frame that we use to write the vector.
Usually we measure things relative to an intertial frame and express them in the local frame.

\subsubsection*{Comments}

\begin{itemize}
    \item Terms must be expressed in the same frame to be used together, but they don't have to be with relative to the same frame
    \item Inertial frames are frames that are not accelerating or rotating.
    \item To use newton's laws (including energy, momentum, Lagrange, Hamilton, Kane?, Gibbs?)
    \item All motion must be relative to some inertial frame, but can expressed in any frame
\end{itemize}

\subsection*{Coordinate Transformations}
Suppose we have two coordinate frames given by basis vectors $\hat{\mathbf{e}}_1, \hat{\mathbf{e}}_2, \hat{\mathbf{e}}_3$ and $\hat{\mathbf{e}}'_1, \hat{\mathbf{e}}'_2, \hat{\mathbf{e}}'_3$. If the axes are aligned we have that
\[
    \begin{bmatrix}
        \hat{\mathbf{e}}'_1 \\
        \hat{\mathbf{e}}'_2 \\
        \hat{\mathbf{e}}'_3
    \end{bmatrix}
    = \mathds{1}
    \begin{bmatrix}
        \hat{\mathbf{e}}_1 \\
        \hat{\mathbf{e}}_2 \\
        \hat{\mathbf{e}}_3
    \end{bmatrix}
\]
In general, we can create some transformation matrix such that
\[
    \begin{bmatrix}
        \hat{\mathbf{e}}'_1 \\
        \hat{\mathbf{e}}'_2 \\
        \hat{\mathbf{e}}'_3
    \end{bmatrix}
    = R
    \begin{bmatrix}
        \hat{\mathbf{e}}_1 \\
        \hat{\mathbf{e}}_2 \\
        \hat{\mathbf{e}}_3
    \end{bmatrix}
\]
Where $R \in SO(3)$, is a rotation matrix. We can figure out the $R_{ij}$ by noting that
$R_{ij} = \hat{\mathbf{e}}'_i \cdot \hat{\mathbf{e}}_j$. We sometimes call this the direction cosine matrix or DCM since $R_{ij} = \hat{\mathbf{e}}'_i \cdot \hat{\mathbf{e}}_j = |\hat{\mathbf{e}}_i||\hat{\mathbf{e}}'_j|\cos \theta_{ij} = \cos \theta_{ij} = \ell_{ij}$. Simply the cosine of the angles between the axes, thus our matrix is ultimately
\[
    R = 
    \begin{bmatrix}
        \ell_{11} & \ell_{12} & \ell_{13} \\
        \ell_{21} & \ell_{22} & \ell_{23} \\
        \ell_{31} & \ell_{32} & \ell_{33}
    \end{bmatrix}
\]
Similarly, we have that
\[
    \begin{bmatrix}
        \hat{\mathbf{e}}_1 \\
        \hat{\mathbf{e}}_2 \\
        \hat{\mathbf{e}}_3
    \end{bmatrix}
    = R^{-1}
    \begin{bmatrix}
        \hat{\mathbf{e}}'_1 \\
        \hat{\mathbf{e}}'_2 \\
        \hat{\mathbf{e}}'_3
    \end{bmatrix}
\]
If we go through the derivation of $R$ we quickly realize that $R^{-1} = R^T$. We further note that $|R| = 1$. We note that the components of a vector transform the same way as the basis vectors. Suppose that $\mathbf{x} = x_1\hat{\mathbf{e}}_1 + x_2\hat{\mathbf{e}}_2 + x_3\hat{\mathbf{e}}_3$, or equivalently
\[
    \mathbf{x} = [x_1, x_2, x_3]
    \begin{bmatrix}
        \hat{\mathbf{e}}_1 \\
        \hat{\mathbf{e}}_2 \\
        \hat{\mathbf{e}}_3
    \end{bmatrix}
    = [x_1 x_2, x_3]
    R^T
    \begin{bmatrix}
        \hat{\mathbf{e}}'_1 \\
        \hat{\mathbf{e}}'_2 \\
        \hat{\mathbf{e}}'_3
    \end{bmatrix}
    = [x'_1, x'_2, x'_3]
    \begin{bmatrix}
        \hat{\mathbf{e}}'_1 \\
        \hat{\mathbf{e}}'_2 \\
        \hat{\mathbf{e}}'_3
    \end{bmatrix}
\]
So we have said that
\[
    [x_1, x_2, x_3]R^T = 
    [x'_1, x'_2, x'_3]
\]
Or equivalently as column vectors
\[
    \begin{bmatrix}
        x'_1 \\
        x'_2 \\
        x'_3
    \end{bmatrix}
    = R
    \begin{bmatrix}
        x_1 \\
        x_2 \\
        x_3
    \end{bmatrix}
\]

\end{document}
