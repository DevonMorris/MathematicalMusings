%!TEX TS-program = xelatex
%!TEX encoding = UTF-8 Unicode

\documentclass[a4paper]{article}

\usepackage{xltxtra}
\usepackage{amsfonts}
\usepackage{polyglossia}
\usepackage{fancyhdr}
\usepackage{geometry}
\usepackage{dsfont}
\usepackage{amsmath}
\usepackage{amsthm}

\geometry{a4paper,left=15mm,right=15mm,top=20mm,bottom=20mm}
\pagestyle{fancy}
\lhead{Devon Morris}
\chead{Dynamics}
\rhead{\today}
\cfoot{\thepage}

\setlength{\headheight}{23pt}
\setlength{\parindent}{0.0in}
\setlength{\parskip}{0.0in}

\newtheorem{prop}{Proposition}

\begin{document}

\section*{Kinematics}

\subsection*{Vectors}
Vectors will always be denoted by $\mathbf{q}$, using a bolded letter. Vectors always have a magnitude and direction (LIES, only in a normed space). Unit vectors can specify directions, such a vector $\hat{\mathbf{e}}$. We construct the vector such that $||\hat{\mathbf{e}}|| = 1$. We can rewrite $\mathbf{q} = q\hat{\mathbf{e}}$. Equivalently, $\hat{\mathbf{e}} = \frac{\mathbf{q}}{||\mathbf{q}||}$.

\subsection*{Coordinate Frames}
Set of three orthonormal vectors in a right handed fashion. To be right-handed we have $\hat{\mathbf{e}}_3 = \hat{\mathbf{e}}_1 \times \hat{\mathbf{e}}_2$. We can now write $\mathbf{q} = q_1 \hat{\mathbf{e}}_1 + q_2 \hat{\mathbf{e}}_2 + q_3 \hat{\mathbf{e}}_3$. We can sometimes write
\[
    \mathbf{q} = 
    \begin{bmatrix}
        q_1 \\
        q_2 \\
        q_3
    \end{bmatrix}
\]
when the basis is implied and we care more about computation on components.

\subsection*{Position, Velocity, Acceleration}
The position of a particle relative to the origin is denoted as $\mathbf{r}(t)$. We can write $\Delta \mathbf{r} = \mathbf{r}(t + \Delta t) - \mathbf{r}(t)$. We define the velocity as 
\[
    \mathbf{v} = \frac{d\mathbf{r}}{dt} = \lim_{\Delta t \rightarrow 0} \frac{\Delta \mathbf{r}}{\Delta t} = \dot{\mathbf{r}}
\]
We can write this as $\mathbf{v} = ||\mathbf{v}||\hat{\mathbf{e}}_{\mathbf{v}}$. Repeating this process with the velocity vector we have

\[
    \mathbf{a} = \frac{d\mathbf{v}}{dt} = \lim_{\Delta t \rightarrow 0} \frac{\Delta \mathbf{v}}{\Delta t} = \dot{\mathbf{v}} = \ddot{\mathbf{r}} = \frac{d^2 \mathbf{r}}{dt^2}
\]
From the discussion in class we can see that there is some component in the direction of the path and some component normal to the path. Let's say we have a position vector given by 
\[
    \mathbf{r} = \sum_{i=1}^3 r_i \hat{\mathbf{e}}_i
\]
Thus we have that, by the product rule
\[
    \mathbf{v} = \sum_{i=1}^3 \dot{r}_i \hat{\mathbf{e}}_i + \sum_{i=1}^3 r_i \dot{\hat{\mathbf{e}}}_i
\]
Doing the same thing with acceleration gives us
\[
    \mathbf{a} = \sum_{i=1}^3 \ddot{r}_i \hat{\mathbf{e}}_i + 2\sum_{i=1}^3 \dot{r}_i\dot{\hat{\mathbf{e}}}_i + \sum_{i=1}^3 r_i \ddot{\hat{\mathbf{e}}}_i
\]

\end{document}
