%!TEX TS-program = xelatex
%!TEX encoding = UTF-8 Unicode

\documentclass[a4paper]{article}

\usepackage{xltxtra}
\usepackage{amsfonts}
\usepackage{polyglossia}
\usepackage{fancyhdr}
\usepackage{geometry}
\usepackage{dsfont}
\usepackage{amsmath}
\usepackage{amsthm}

\geometry{a4paper,left=15mm,right=15mm,top=20mm,bottom=20mm}
\pagestyle{fancy}
\lhead{Devon Morris}
\chead{Dynamics}
\rhead{\today}
\cfoot{\thepage}

\setlength{\headheight}{23pt}
\setlength{\parindent}{0.0in}
\setlength{\parskip}{0.0in}

\newtheorem{prop}{Proposition}

\begin{document}

\section*{Introduction}
People orignally studied dynamics because they were interested in simple machines, and then celestial mechanics etc. Newton and Galileo added math in 1600s. We will be principally studying two branches: kinematics and kinetics. Kinematics describes motion without mass, energy, force, etc. Kinetics show relationships between energy, mass and kinematics. We are going to focus on dynamics of particles, then systems of particles, then rigid bodies. 

\end{document}
