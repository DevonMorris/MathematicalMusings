%!TEX TS-program = xelatex
%!TEX encoding = UTF-8 Unicode

\documentclass[a4paper]{article}

\usepackage{xltxtra}
\usepackage{amsfonts}
\usepackage{polyglossia}
\usepackage{fancyhdr}
\usepackage{geometry}
\usepackage{dsfont}
\usepackage{amsmath}
\usepackage{amsthm}
\usepackage{amssymb}
\usepackage{bm}

\geometry{a4paper,left=15mm,right=15mm,top=20mm,bottom=20mm}
\pagestyle{fancy}
\lhead{Devon Morris}
\chead{Detection \& Estimation Theory}
\rhead{\today}
\cfoot{\thepage}

\setlength{\headheight}{23pt}
\setlength{\parindent}{0.0in}
\setlength{\parskip}{0.0in}

\newtheorem{prop}{Proposition}
\newtheorem*{defn}{Definition}

\begin{document}
\section*{Neyman-Pearson Detection}%
Any yes-no question can be presented as a binary hypothesis. Here is some terminology. $\bm{X}$ is a random vector drawn from distribution $F_{\bm{\theta}}(\bm{X})$. Such that $\bm{\theta} \in \Theta$ and $\Theta = \Theta_0 \cup \Theta_1$, and we stipulate $\Theta_0 \cap \Theta_1 = \varnothing$. We want to test for the hypotheses, $H_0: \bm{\theta} \in \Theta_0$ and $H_1: \bm{\theta} \in \Theta_1$ and $\bm{\theta}$ is unknown. We first estimate $\bm{\theta}$ and then have a classification test. If $\Theta_i$ contains a single element then $H_i$ is a simple hypothesis, else $H_i$ is composite.

\end{document}
