%!TEX TS-program = xelatex
%!TEX encoding = UTF-8 Unicode

\documentclass[a4paper]{article}

\usepackage{xltxtra}
\usepackage{amsfonts}
\usepackage{polyglossia}
\usepackage{fancyhdr}
\usepackage{geometry}
\usepackage{dsfont}
\usepackage{amsmath}
\usepackage{amsthm}
\usepackage{physics}
\usepackage{float}
\usepackage{tikz}
\usepackage{bm}
\usetikzlibrary{shapes,arrows,positioning}

\geometry{a4paper,left=15mm,right=15mm,top=20mm,bottom=20mm}
\pagestyle{fancy}
\lhead{Devon Morris}
\chead{Detection \& Estimation Theory - Homework 1}
\rhead{\today}
\cfoot{\thepage}

\setlength{\headheight}{23pt}
\setlength{\parindent}{0.0in}
\setlength{\parskip}{0.0in}

\newtheorem{prop}{Proposition}
\newtheorem*{sol}{Solution}

\tikzset{
block/.style = {draw, fill=white, rectangle, minimum height=3em, minimum width=3em},
tmp/.style  = {coordinate}, 
sum/.style= {draw, fill=white, circle, node distance=1cm},
input/.style = {coordinate},
output/.style= {coordinate},
pinstyle/.style = {pin edge={to-,thin,black}
}
}

\begin{document}
\section*{Problem 1}%
Let $\mathbf{y} = f(\mathbf{x})$ denote a linear mapping from $\mathds{R}^p$ to $\mathds{R}^n$ i.e.
\[
  \begin{aligned}
    \mathbf{y} &= f(\mathbf{x}) \\
    f(a \mathbf{x}) &= a f(\mathbf{x}) \\
    f(\mathbf{x}_1 + \mathbf{x}_2) &= f(\mathbf{x}_1) + f(\mathbf{x}_2)
  \end{aligned}
\]
Show that this mapping may always be written
\[
  \mathbf{y} = A\mathbf{x}
\]
Determine $A = [\mathbf{a}_1, \cdots, \mathbf{a}_p]$

\subsection*{Solution}%
Using the canonical basis $\{\mathbf{e}_1, \dots, \mathbf{e}_p\}$ for  $\mathds{R}^p$, we can express any vector $\mathbf{x}$ in terms of its components, i.e.
\[
  \mathbf{x} = x_1\mathbf{e}_1 + \cdots + x_p\mathbf{e}_p = \sum_{j=1}^p x_j\mathbf{e}_j
\]
Using the linearity of $f$, we have
\[
  \mathbf{y} = f(\mathbf{x}) = f \left( \sum_{j=1}^p x_j\mathbf{e}_j\right) = \sum_{j=1}^p x_jf(\mathbf{e}_j)
\]
We can rewrite this as a matrix vector product
\[
  \mathbf{y} = [f(\mathbf{e}_1), \cdots, f(\mathbf{e}_p)]
  \begin{bmatrix}
    x_1 \\
    \vdots  \\
    x_p
  \end{bmatrix}
  =[f(\mathbf{e}_1), \cdots, f(\mathbf{e}_p)]\mathbf{x}
\]
If we let $A = [f(\mathbf{e}_1), \cdots, f(\mathbf{e}_p)]$, then we have that $\mathbf{y} = A\mathbf{x}$. (Technically, we can do this in any bases for $\mathds{R}^p$ and $\mathds{R}^n$ , but the representation of $\mathbf{y}, \mathbf{x},$ and $A$, will change.)

\section*{Problem 2}%
Prove that the matrix $Q = I - 2(\mathbf{v}\mathbf{v}^{\top}/\mathbf{v}^{\top}\mathbf{v})$ is orthogonal and that it projects through any vector that is perpendicular to $\mathbf{v}$.

\subsection*{Solution}%
Consider $Q^\top Q$
\[
  \begin{aligned}
    Q^\top Q &= (I - 2(\mathbf{v}\mathbf{v}^{\top}/\mathbf{v}^{\top}\mathbf{v}))^{\top}(I - 2(\mathbf{v}\mathbf{v}^{\top}/\mathbf{v}^{\top}\mathbf{v})) \\
             &= I - 4(\mathbf{v}\mathbf{v}^\top/\mathbf{v}^\top\mathbf{v}) + 4(\mathbf{v}\mathbf{v}^\top \mathbf{v}\mathbf{v}^\top/(\mathbf{v}^\top\mathbf{v})^2) \\
             &= I
  \end{aligned}
\]
Therefore, we have that this matrix is orthogonal.





\end{document}
