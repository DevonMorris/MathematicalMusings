%!TEX TS-program = xelatex
%!TEX encoding = UTF-8 Unicode

\documentclass[a4paper]{article}

\usepackage{xltxtra}
\usepackage{amsfonts}
\usepackage{polyglossia}
\usepackage{fancyhdr}
\usepackage{geometry}
\usepackage{dsfont}
\usepackage{amsmath}
\usepackage{amsthm}

\geometry{a4paper,left=15mm,right=15mm,top=20mm,bottom=20mm}
\pagestyle{fancy}
\lhead{Devon Morris}
\chead{Computer Vision}
\rhead{\today}
\cfoot{\thepage}

\setlength{\headheight}{23pt}
\setlength{\parindent}{0.0in}
\setlength{\parskip}{0.0in}

\newtheorem{prop}{Proposition}

\begin{document}

\section*{Kinematics for Two Moving Cameras}
The homography is a matrix that relates the two images of the same planar scene. The homography can be written as
\[
    H = \gamma \left( R + \frac{Tn^T}{d} \right)
\]
Where $R$ is a rotation matrix, $T$ is the translation between camera frames, $n$ is the vector normal to the scene, $d$ is the orthogonal distance to the scene and $\gamma$ is a scale factor. The goal of this document is to derive the kinematics on $SL(3)$ given by
\[
    \dot{H} = HX
\]
Where $H \in SL(3)$ is the homography and $X \in \mathfrak{sl}(3)$ is the group velocity. Furthermore, we wish to derive this group velocity with respect to intertial body velocities of each camera. We begin by laying down our coordinate frames.

Suppose that we have two camera frames $\{A\}$ and $\{B\}$. Let $R_B^A$ be the rotation matrix expressing the orientation of $\{B\}$ with respect to $\{A\}$ expressed in $\{A\}$, $T_{B/A}^A$ be the translation of $\{B\}$ with respect to $\{A\}$ expressed in $\{A\}$, $n^B$ the normal vector of the planar scene expressed in frame $\{B\}$, and $d_B$ be the orthogonal distance from frame $\{B\}$ to the planar scene.

We can then write the homography as 
\[
    H_{B}^{A} = \gamma \left( R_{B}^{A} + \frac{T_{B/A}^A (n^B)^T}{d_{B}} \right)
\]
It is important to note that $n$ is NOT expressed in the same frame as $T$ and $R$. This fact becomes very apparent in the derivation of the homography (See An Invitation to 3-D Vision Chapter 5, Ma \& Soatto). The problem with writing the homography in this fashion, is that we don't measure derivative of $R_{B}^A$ or $T_{B/A}^A$ directly. Thus, we will introduce an intermediate frame $\{0\}$ and first derive the dynamics in that frame. Consider the homography 
\[
    H_A^0 = \gamma_A \left( R_A^0 + \frac{T_{A/0}^0(n^A)^T}{d_A} \right)
\]
We wish to write the left invariant kinematics in the form $\dot{H}_A^0 = H_A^0X$, where $X \in \mathfrak{sl}(3)$. Taking the derivative yields
\[
    \begin{aligned}
        \dot{H}_{A}^0 &= \frac{\dot{\gamma}_A}{\gamma_A}H_{A}^0 + \gamma_A \left( \dot{R}_A^0 + \frac{\dot{T}_{A/0}^0 (n^A)^T + T_{A/0}^0(\dot{n}^A)^T}{d_A} - \frac{\dot{d}_AT_{A/0}^0(n^A)^T}{d_A^2}\right) \\
                      &= \frac{\dot{\gamma}_A}{\gamma_A}H_{A}^0 + \gamma_A \left( R_A^0 \left( \omega_{A/0}^A \right)^{\wedge} + \frac{R_{A}^0V_{A/0}^A(n^A)^T + T_{A/0}^0(n^A)^T \left( \omega_{A/0}^A \right)^{\wedge}}{d_A} + \frac{(n^A)^TV_{A/0}^AT_{A/0}^A(n^A)^T}{d_A^2} \right) \\
                      &= \frac{\dot{\gamma}_A}{\gamma_A}H_{A}^0 + \gamma_A \left( \left[ R_{A}^0 + \frac{T_{A/0}^0(n^A)^T}{d_A} \right]\left(\omega_{A/0}^A\right)^{\wedge} + \left[ R_A^0 + \frac{T_{A/0}^A(n^A)^T}{d_A}\right]\frac{V_{A/0}^A(n^A)^T}{d_A} \right) \\
                      &= H_A^0 \left( \left( \omega_{A/0}^A \right)^{\wedge}  +  \frac{V_{A/0}^A(n^A)^T}{d_A} + \frac{\dot{\gamma}_A}{\gamma_A}I_3 \right)
    \end{aligned}
\]
Now, we recall the definition of $\mathfrak{sl}(3)$
\[
    \mathfrak{sl}(3) = \left\{ X \in \mathds{R}^{3 \times 3}\ |\ \text{tr}(X) = 0 \right\},
\]
which imposes the constraint
\[
    \text{tr} \left( \left( \omega_{A/0}^A \right)^{\wedge}  +  \frac{V_{A/0}^A(n^A)^T}{d_A} + \frac{\dot{\gamma}_A}{\gamma_A}I_3 \right) = \frac{(n^A)^TV_{A/0}^A}{d_A} + 3\frac{\dot{\gamma}_A}{\gamma_{A}} = 0
\]
Thus we have that
\[
    \frac{\dot{\gamma}_A}{\gamma_A} = -\frac{(n^A)^TV_{A/0}^A}{3d_A}
\]
Therefore, our kinematics for $H_A^0$ are
\[
    \dot{H}_A^0 = H_A^0 \left( \left( \omega_{A/0}^A \right)^{\wedge} +  \frac{V_{A/0}^A(n^A)^T}{d_A}-\frac{(n^A)^TV_{A/0}^A}{3d_A}I_3 \right) = H_A^0X_A
\]
Similarly for $H_B^0$, we have
\[
    \dot{H}_B^0 = H_B^0 \left( \left( \omega_{B/0}^B \right)^{\wedge} +  \frac{V_{B/0}^B(n^B)^T}{d_B}-\frac{(n^B)^TV_{B/0}^B}{3d_B}I_3 \right) = H_B^0X_B
\]
Now, we have written two systems in terms of quantities that are easily measured. To combine these systems into one system, we define $H_B^A = (H_A^0)^{-1}H_B^0$, and take the derivative as follows
\[
    \begin{aligned}
        \dot{H}_B^A &= \frac{d}{dt} \left({(H_A^0)^{-1}}\right)H_B^0 + (H_A^0)^{-1}\dot{H}_B^0 \\
                    &= -(H_A^0)^{-1}\dot{H}_A^0(H_A^0)^{-1}H_B^0 + (H_A^0)^{-1}\dot{H}_B^0 \\
                    &= -X_A(H_A^0)^{-1}H_B^0 + (H_A^0)^{-1}H_B^0X_B \\
                    &= H_B^AX_B - H_B^A(H_B^A)^{-1}X_AH_B^A \\
                    &= H_B^A \left( X_B - (H_B^A)^{-1}X_AH_B^A \right) \\
                    &= H_B^A \left( X_B - \text{Ad}_{(H_B^A)^{-1}}(X_A) \right) \\
                    &= H_B^A \left( X_B - \text{Ad}_{H_A^B}(X_A) \right)
    \end{aligned}
\]
Thus, we have kinematics for a system of two moving cameras. We know that $X_B - \text{Ad}_{H_A^B}(X_A) \in \mathfrak{sl}(3)$ since the trace is zero. 

We note at this point that the vector field we have created is not left-invariant, this is equivalent to saying that there is no single element of the lie algebra that characterizes the vector field at all points on $SL(3)$. So we can't take the matrix exponential to get our position on the manifold at any point in time, however since both $X_B$ and $X_A$ are assumed to be time-varying we don't need to use the matrix exponential over long time periods$.

\end{document}
