%!TEX TS-program = xelatex
%!TEX encoding = UTF-8 Unicode

\documentclass[a4paper]{article}

\usepackage{xltxtra}
\usepackage{amsfonts}
\usepackage{polyglossia}
\usepackage{fancyhdr}
\usepackage{geometry}
\usepackage{dsfont}
\usepackage{amsmath}
\usepackage{amsthm}
\usepackage{amssymb}
\usepackage{physics}

\geometry{a4paper,left=15mm,right=15mm,top=20mm,bottom=20mm}
\pagestyle{fancy}
\lhead{Devon Morris}
\chead{Differential Geometry}
\rhead{\today}
\cfoot{\thepage}

\setlength{\headheight}{23pt}
\setlength{\parindent}{0.0in}
\setlength{\parskip}{0.0in}

\newtheorem*{prop}{Proposition}
\newtheorem*{defn}{Definition}
\newtheorem*{thm}{Theorem}
\newtheorem*{rem}{Remark}
\begin{document}

\section*{Linear Functionals}%
Given a vector space $V$, we can analyze functions $\alpha: V \rightarrow \mathds{R}$, that are linear in their arguments. Thus, for $a,b \in \mathds{R}$, $x,y \in V$ we have
\[
  \alpha(a x + b y) = a \alpha(x) + b \alpha(y)
\]
We can define operations on thse functionals such as addition and multiplication by
\[
  \begin{aligned}
    (\alpha + \beta)(x) &= \alpha(x) + \beta(x) \\
    (c\alpha)(x) &= c\alpha(x)
  \end{aligned}
\]
With these operations, it can be easily shown that these linear functionals form a vector space in their own right called the dual space and we denote it by $V^*$. Regarding this dual space an interesting question is the existence of a canoncical basis for $V^*$. Given a basis $\{e_i\}$ for $V$, we define the functionals $\{\sigma^i\}$ such that
\[
  \sigma^i(e_j) = \delta^i_j
\]
Imposing the constraint of linearity, we have
\[
  \sigma^i(v) = \sigma^i \left( v^je_j \right) = v^j\sigma^i(e_j) = v^j \delta_j^i = v^i
\]
or in other words, these functionals index the vector and read off the $i$th component. 

\begin{thm}
  Let $\{e_i\}$ be a basis for $V$. The linear functionals $\{\sigma^i\}$ given by 
  \[
    \sigma^i(e_j) = \delta^i_j
  \]
  form a basis for the cotangent space $V^*$
\end{thm}

\begin{proof}
  First we will verify that the $\sigma^i$ are linearly independent. Thus suppose that
  \[
    a_i \sigma^i = 0
  \]
  where $0$ is the zero functional. Thus we have that
  \[
    a_i \sigma^i(e_j) = a_i \delta^i_j = a_j = 0
  \]
  therefore we have that $a_j = 0$ for all $j$. Thus, we have linear independence. Now, we wish to check and see if the $\sigma^i$ span $V^*$. Given any $\alpha \in V^*$ and $v \in V$ we have that
  \[
    \alpha(v) =  \alpha(v^ie_i) = v^i\alpha(e_i) = \delta_j^iv^j \alpha(e_i) = \sigma^i(v) \alpha(e_i) = (\alpha(e_i)\sigma^i)(v)
  \]
  Thus, we have expressed any $\alpha$ as a linear combination of $\{\sigma^i\}$ and therefore, $\{\sigma^i\}$ forms a basis for $V^*$.
\end{proof}

\section*{Cotangent Vectors}%
Now that we have proven the existence of a dual space for any vector space, we can define analyze the dual of the tangent space $T_pM$.

\begin{defn}
  Let $M$ be a smooth manifold. At every point $p \in M$, there is a dual space to the tangent space $(T_pM)^*$, we call this the cotangent space and denote it by $T_p^*M$.
\end{defn}

\begin{rem}
  The tangent space $T_pM$ can be thought of as derivatives or derivations $[\gamma]_p$ that act on function $f \in C^{\infty}(M)$. For example if $[\gamma]_p = X_p \in T_pM$ then $X_p$ is really defined by the way it acts on $f$, 
  \[
    X(f) = (f \circ \gamma)'(0).
  \]
\end{rem}
  In other words $T_pM$ can be thought of as smooth functions $X_p: U \rightarrow \mathds{R}$ where $U$ is a neighborhood of $p$. The cotangent space then $T_p^*M$ should be thought of as linear functionals $\alpha: T_pM \rightarrow \mathds{R}$, in other words, for $X_p,Y_p \in T_pM$ we have
  \[
    \alpha(a X_p + b Y_p) = a \alpha(X_p) + b \alpha(Y_p) 
  \]
  We already have a structure that acts linearly on the tangent space, namely the differential. Note for  $f \in C^{\infty}(M)$, $df: T_pM \rightarrow T_{f(p)}\mathds{R}$. However, $T_{f(p)}\mathds{R} \cong \mathds{R}$ so we can think of $df: T_pM \rightarrow \mathds{R}$ as a linear functional on $T_pM$. Thus 
  \[
    df(X_p) = X_p(f) = (f \circ \gamma)'(0)
  \]
  In local coordinates $(U, (x^j))$, we have
  \[
    df(X_p) = df \left(X^j \pdv{x^j} \right) = X^j df \left( \pdv{x^j} \right) = X^j \pdv{f}{x^j}
  \]
  We can also think of the coordinates $x^j$ as $x^j \in C^{\infty}(M)$. From this formulation, it's clear that $\{dx^i\}$ forms the dual basis for $T^*_pM$ since
  \[
    dx^i \left( \pdv{x^j} \right) = \pdv{x^j}{x^i} = \delta^i_j
  \]
  Thus any $\alpha \in T_p^*M$ can be expressed as 
  \[
    \alpha = \alpha_i dx^i
  \]
  We call these expressions differential forms. In local coordinates, we have
  \[
    \alpha = df = \pdv{f}{x^j} dx^j
  \]
  At this point we note that elements of $T_p^*M$ are called covariant vectors, covectors or 1-forms. We can also create a vector bundle called the cotangent bundle
   \begin{defn}
     The cotangent bundle $T*M$ is the disjoint union of all the cotangent spaces $T*M = \coprod_{p \in M} T^*_p$
   \end{defn}
  
  Similar to vector fields, covector fields $\alpha(p)$ are sections of the cotangent bundle, $\alpha(p) \in \Gamma^{\infty}(T^*M)$.


\end{document}
