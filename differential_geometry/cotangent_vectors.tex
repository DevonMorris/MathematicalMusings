%!TEX TS-program = xelatex
%!TEX encoding = UTF-8 Unicode

\documentclass[a4paper]{article}

\usepackage{xltxtra}
\usepackage{amsfonts}
\usepackage{polyglossia}
\usepackage{fancyhdr}
\usepackage{geometry}
\usepackage{dsfont}
\usepackage{amsmath}
\usepackage{amsthm}
\usepackage{amssymb}
\usepackage{physics}

\geometry{a4paper,left=15mm,right=15mm,top=20mm,bottom=20mm}
\pagestyle{fancy}
\lhead{Devon Morris}
\chead{Differential Geometry}
\rhead{\today}
\cfoot{\thepage}

\setlength{\headheight}{23pt}
\setlength{\parindent}{0.0in}
\setlength{\parskip}{0.0in}

\newtheorem*{prop}{Proposition}
\newtheorem*{defn}{Definition}
\newtheorem*{thm}{Theorem}
\begin{document}

\section*{Linear Functionals}%
Given a vector space $V$, we can analyze functions $\alpha: V \rightarrow \mathds{R}$, that are linear in their arguments. Thus, for $a,b \in \mathds{R}$, $x,y \in V$ we have
\[
  \alpha(a x + b y) = a \alpha(x) + b \alpha(y)
\]
We can define operations on thse functionals such as addition and multiplication by
\[
  \begin{aligned}
    (\alpha + \beta)(x) &= \alpha(x) + \beta(x) \\
    (c\alpha)(x) &= c\alpha(x)
  \end{aligned}
\]
With these operations, it can be easily shown that these linear functionals form a vector space in their own right called the dual space and we denote it by $V^*$. Regarding this dual space an interesting question is the existence of a canoncical basis for $V^*$. Given a basis $\{e_i\}$ for $V$, we define the functionals $\{\sigma^i\}$ such that
\[
  \sigma^i(e_j) = \delta^i_j
\]
Imposing the constraint of linearity, we have
\[
  \sigma^i(v) = \sigma^i \left( v^je_j \right) = v^j\sigma^i(e_j) = v^j \delta_j^i = v^i
\]
or in other words, these functionals index the vector and read off the $i$th component. 

\begin{thm}
  Let $\{e_i\}$ be a basis for $V$. The linear functionals $\{\sigma^i\}$ given by 
  \[
    \sigma^i(e_j) = \delta^i_j
  \]
  form a basis for the cotangent space $V^*$
\end{thm}

\begin{proof}
  First we will verify that the $\sigma^i$ are linearly independent. Thus suppose that
  \[
    a_i \sigma^i = 0
  \]
  where $0$ is the zero functional. Thus we have that
  \[
    a_i \sigma^i(e_j) = a_i \delta^i_j = a_j = 0
  \]
  therefore we have that $a_j = 0$ for all $j$. Thus, we have linear independence. Now, we wish to check and see if the $\sigma^i$ span $V^*$. Given any $\alpha \in V^*$ and $v \in V$ we have that
  \[
    \alpha(v) =  \alpha(v^ie_i) = v^i\alpha(e_i) = \delta_j^iv^j \alpha(e_i) = \sigma^i(v) \alpha(e_i) = (\alpha(e_i)\sigma^i)(v)
  \]
  Thus, we have expressed any $\alpha$ as a linear combination of $\{\sigma^i\}$ and therefore, $\{\sigma^i\}$ forms a basis for $V^*$.
\end{proof}



\section*{Cotangent Vectors}%

\end{document}
