%!TEX TS-program = xelatex
%!TEX encoding = UTF-8 Unicode

\documentclass[a4paper]{article}

\usepackage{xltxtra}
\usepackage{amsfonts}
\usepackage{polyglossia}
\usepackage{fancyhdr}
\usepackage{geometry}
\usepackage{dsfont}
\usepackage{amsmath}
\usepackage{amsthm}
\usepackage{amssymb}
\usepackage{physics}

\geometry{a4paper,left=15mm,right=15mm,top=20mm,bottom=20mm}
\pagestyle{fancy}
\lhead{Devon Morris}
\chead{Differential Geometry}
\rhead{\today}
\cfoot{\thepage}

\setlength{\headheight}{23pt}
\setlength{\parindent}{0.0in}
\setlength{\parskip}{0.0in}

\newtheorem*{prop}{Proposition}
\newtheorem*{defn}{Definition}
\newtheorem*{thm}{Theorem}
\newtheorem*{rem}{Remark}
\begin{document}

\section*{Linear Functionals}%
Given a vector space $V$, we can analyze functions $\alpha: V \rightarrow \mathds{R}$, that are linear in their arguments. Thus, for $a,b \in \mathds{R}$, $x,y \in V$ we have
\[
  \alpha(a x + b y) = a \alpha(x) + b \alpha(y)
\]
We can define operations on thse functionals such as addition and multiplication by
\[
  \begin{aligned}
    (\alpha + \beta)(x) &= \alpha(x) + \beta(x) \\
    (c\alpha)(x) &= c\alpha(x)
  \end{aligned}
\]
With these operations, it can be easily shown that these linear functionals form a vector space in their own right called the dual space and we denote it by $V^*$. Regarding this dual space an interesting question is the existence of a canoncical basis for $V^*$. Given a basis $\{e_i\}$ for $V$, we define the functionals $\{\sigma^i\}$ such that
\[
  \sigma^i(e_j) = \delta^i_j
\]
Imposing the constraint of linearity, we have
\[
  \sigma^i(v) = \sigma^i \left( v^je_j \right) = v^j\sigma^i(e_j) = v^j \delta_j^i = v^i
\]
or in other words, these functionals index the vector and read off the $i$th component. 

\begin{thm}
  Let $\{e_i\}$ be a basis for $V$. The linear functionals $\{\sigma^i\}$ given by 
  \[
    \sigma^i(e_j) = \delta^i_j
  \]
  form a basis for the cotangent space $V^*$
\end{thm}

\begin{proof}
  First we will verify that the $\sigma^i$ are linearly independent. Thus suppose that
  \[
    a_i \sigma^i = 0
  \]
  where $0$ is the zero functional. Thus we have that
  \[
    a_i \sigma^i(e_j) = a_i \delta^i_j = a_j = 0
  \]
  therefore we have that $a_j = 0$ for all $j$. Thus, we have linear independence. Now, we wish to check and see if the $\sigma^i$ span $V^*$. Given any $\alpha \in V^*$ and $v \in V$ we have that
  \[
    \alpha(v) =  \alpha(v^ie_i) = v^i\alpha(e_i) = \delta_j^iv^j \alpha(e_i) = \sigma^i(v) \alpha(e_i) = (\alpha(e_i)\sigma^i)(v)
  \]
  Thus, we have expressed any $\alpha$ as a linear combination of $\{\sigma^i\}$ and therefore, $\{\sigma^i\}$ forms a basis for $V^*$.
\end{proof}
Thus any $\alpha \in V^*$ can be written as $a_i \sigma^i$. This gives a nice computation of $\alpha(v)$ in coordinates
\[
  \alpha(v) = a_i\sigma^i(v^je_j) = a_iv^j \sigma^i(e_j) = a_iv^j \delta^i_j = a_iv^i
\]
or put simply you just multiply the components and sum them.

\section*{Cotangent Vectors}%
Now that we have proven the existence of a dual space for any vector space, we can define analyze the dual of the tangent space $T_pM$.

\begin{defn}
  Let $M$ be a smooth manifold. At every point $p \in M$, there is a dual space to the tangent space $(T_pM)^*$, we call this the cotangent space and denote it by $T_p^*M$.
\end{defn}

\subsection*{Definition from Equivalence Classes}%

\begin{rem}
  The tangent space $T_pM$ can be thought of as derivatives or derivations $[\gamma]_p$ that act on function $f \in C^{\infty}(M)$. For example if $[\gamma]_p = X \in T_pM$ then $X$ is really defined by the way it acts on $f$, 
  \[
    X(f) = (f \circ \gamma)'(0).
  \]
\end{rem}
In other words $T_pM$ can be thought of as smooth functions $X: U \rightarrow \mathds{R}$ where $U$ is a neighborhood of $p$. However, we can think about this same structure from a different angle. Consider $f,g \in C^{\infty}(U)$ such that for all $X \in T_pM$
  \[
    X(f) = (f \circ \gamma)'(0) = (g \circ \gamma)'(0) = X(g)
  \]
  In local coordinates, we have
  \[
    X(f) = \dv{\gamma^j}{t} \pdv{f}{x^j} = \dv{\gamma^j}{t} \pdv{g}{x^j} = X(g)
  \]
  Since this must hold for all $X$ we necessarily have that $\pdv{f}{x^j} = \pdv{g}{x^j}$. This definition forms an equivalence relation. Under this relation we say that $f \sim_p g$ and denote its classes by $[f]_p$. In this way we can consider $[f]_p: T_pM \rightarrow \mathds{R}$ acting on $X$ by
  \[
    [f]_p(X) = X(f) = (f \circ \gamma)'(0)
  \]
  We note that this operation is well-defined because of the relation we defined above. Furthermore, we note that $[f]_p$ is linear in its vector argument (thinking of tangent vectors as derivations)
  \[
    [f]_p(a X + bY) = (a X + b Y)(f) = aX(f) + bY(f).
  \]
  Thus we must have that $[f]_p \in T_p^*M$, implying that elements of $T_p^*M$ are made up of these equivalence classes of functions. Now, we wish to find a dual basis. Given a coordinate basis $\{ \pdv{x^j}\}$ for $T_pM$, we want an $h^i \in C^{\infty}(M)$ such that
  \[
    [h^i]_p \left( \pdv{x^j} \right) = (h \circ (x^j)^{-1})'(0) = \delta^i_j
  \]
  We note that we can always find $(x^j)^{-1}(0) = p$ just by shifting our charts. From this it should be obvious that $x^i$ should be the best representative of $[h^i]_p$. So our coordinate basis for the dual space $T_p^*M$ is given by $\{[x^i]_p\}$. So any $[f]_p$ can be written
  \[
    [f]_p = a_i [x^i]_p
  \]
  Using the properties of derivations we have
  \[
    [f]_p(X) = X(a_i[x^i]_p) = a_i X([x^i]_p) = a_i \dv{\gamma^j}{t} \pdv{x^i}{x^j} = a_i \dv{\gamma^i}{t} = \pdv{f}{x^i} \dv{\gamma^i}{t}
  \]
  where the last equality comes from the calculation of $X(f)$ in local coordinates. Since this relation must hold for any $X$ we have that
  \[
    a_i = \pdv{f}{x^i}
  \]
  for a suitable representative $f$ and local coordinates $(x^j)$. We call these the local coordinates of the dual space.
  
\subsection*{Definition as Differentials}%
  
If we look more carefully at the definition via equivalence classes, we can see that the distinguishing factor of $[f]_p \in T_p^*M$ is its first order behavior in a neighborhood of $p$. This is especially apparent in the coordinate description
\[
  [f]_p(X) = \dv{\gamma^j}{t}\pdv{f}{x^j}
\]
We have seen this expression before in a different context
\[
  df_p(X) = [f]_p(X) = \pdv{f}{x^j}\dv{\gamma^j}{t}
\]
Which makes sense because $df_p$ captures the first order behavior of $f$ at $p$. Thus, the cotangent space then $T_p^*M$ can also be thought of as differentials of functions $f \in C^{\infty}(U)$.

  \[
    df(X_p) = X_p(f) = (f \circ \gamma)'(0)
  \]
  So instead of denoting our local basis as $\{[x^j]_p\}$ we will instead denote them as $\{dx^j_p\}$.
  Thus any $\alpha \in T_p^*M$ can be expressed as 
  \[
    \alpha = \alpha_i dx^i_p
  \]
  We call these expressions differential forms. To simplify notation, we will usually write $dx^j$ when $p$ is clear from context. In local coordinates, we have
  \[
    \alpha = df_p = \pdv{f}{x^j} dx^j
  \]
  At this point we note that elements of $T_p^*M$ are called covariant vectors, covectors. We can also create a vector bundle called the cotangent bundle. 
   \begin{defn}
     The cotangent bundle $T^*M$ is the disjoint union of all the cotangent spaces $T^*M = \coprod_{p \in M} T^*_p$
   \end{defn}
  
   Similar to vector fields, covector fields $\alpha: M \rightarrow T^*M$ are smooth sections of the cotangent bundle, $\alpha \in \Gamma^{\infty}(T^*M)$. For $f \in C^{\infty}(M)$, these covector fields $df = \pdv{f}{x^j} dx^j$ are called 1-forms.

 \section*{Pullback}%
 Since $df \in T_p^*M$ are vectors (covectors) in their own right, we would like to move them from one manifold to another like we do with the push-forward (differential) $dF_p: T_pM \rightarrow T_{F(p)}N$, for $F: M \rightarrow N$.  However, we can think of this as some vector acting on functions $g \in C^{\infty}(V)$, or equivalently we can think of $dg_{F(p)} \in T_{F(p)}^N$ as a 1-form acting on these vectors
 \[
   dg_{F(p)} \left( dF_p \left(X\right) \right)
 \]
 Since $X$ is an input and both $dF_p$ and $dg_{F(p)}$ are linear, the function $dg_F \circ dF_p: T_pM \rightarrow \mathds{R}$, must be an element of the cotangent space $T_p^*M$. Thus we have moved a covector $dg_{F(p)}$ from $T_{F(p)}N$ to $T_pM$.
 \begin{defn}
   Let $M,N$ be smooth manifolds with $F: M \rightarrow N$ a smooth function. Furthermore let $dF_p: T_pM \rightarrow T_{F(p)}N$ be the push-forward (differential) of $F$ and let $X \in T_pM$. We define the pullback $dF*: T_{F(p)}N \rightarrow T_pM$ as 
   \[
     dF^*(dg_{F(p)})(X) = dg_{F(p)}(dF_p(X))
   \]
   or when $p$ and $F(p)$ are clear from context
   \[
     dF^*(dg)(X) = dg(dF(X))
   \]
   We say that this is the pullback of $T_{F(p)}N$ under $F$.
 \end{defn}
 We note that the pullback inherits linearity from the linearity of the differential. Consider how this pullback works in local coordinates $(U, (x^j))$ and $(V, (y^j))$
 \[
   \begin{aligned}
     dF^*(dg)(X) &= dF^* \left( \alpha_jdy^j \right) \left(X^i \pdv{x^i}  \right) \\
                 &= \alpha_j dF^*\left( dy^i \right) \left( X^i \pdv{x^i} \right) \\
                 &= \alpha_j dy^j \left( dF \left(X^i \pdv{x^i}\right) \right) \\
                 &= \alpha_j dy^j \left( \pdv{F^k}{x^i}X^i \pdv{y^k} \right) \\
                 &= \alpha_j \pdv{F^k}{x^i} X^i dy^j \left( \pdv{y^k} \right) \\
                 &= \alpha_j \pdv{F^k}{x^i} X^i \delta^j_k \\
                 &= \alpha_k \pdv{F^k}{x^i} X^i
   \end{aligned}
 \]
 From this we can see that the components of $dF^*(dg)$ in the $\{dx^i\}$ basis are 
 \[
   dF^*(dg) = \alpha_k \pdv{F^k}{x^i} dx^i
 \]
 Transforming with the transpose of the jacobian. One important consequence of this is that if $dg \in \Gamma^{\infty}(T^*N)$ then by definition $dF^*(dg) \in \Gamma^{\infty}(T^*M)$, or in other words covector fields pulled back are still covector fields. This is not the case for vectors fields pushed-forward with the differential map. 

\end{document}
