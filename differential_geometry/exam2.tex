%!TEX TS-program = xelatex
%!TEX encoding = UTF-8 Unicode

\documentclass[a4paper]{article}

\usepackage{xltxtra}
\usepackage{amsfonts}
\usepackage{polyglossia}
\usepackage{fancyhdr}
\usepackage{geometry}
\usepackage{dsfont}
\usepackage{amsmath}
\usepackage{amsthm}
\usepackage{amssymb}
\usepackage{physics}
\usepackage[shortlabels]{enumitem}
\usepackage{mathtools}

\geometry{a4paper,left=15mm,right=15mm,top=20mm,bottom=20mm}
\pagestyle{fancy}
\lhead{Devon Morris}
\chead{Differential Geometry - Exam 2}
\rhead{\today}
\cfoot{\thepage}

\setlength{\headheight}{23pt}
\setlength{\parindent}{0.0in}
\setlength{\parskip}{0.0in}

\newtheorem*{prop}{Proposition}
\newtheorem*{defn}{Definition}

\DeclarePairedDelimiterX{\inn}[2]{\langle}{\rangle}{#1, #2}

\begin{document}

\section*{Problem 1}%

Consider the sphere $S_R = \left\{ (x,y,z) \in \mathds{R}^3\ |\ x^2 + y^2 + z^2 = R^2 \right\}$ of radius R, and let $g$ be the induced metric on $S_R$ from the standard metric on $\mathds{R}^3$. 
\begin{enumerate}[(a)]
  \item Compute the components $g_{ij}$ of the metric $g$ in spherical coordinates.
  \item Compute the the components $R_{ijk}^l$ of the Riemann curvature tensor in these coordinates.
  \item Compute the sectional curvature $K(\sigma)$ on $S_R$, where $\sigma = T_pS_R$.
  \item Compute the Ricci curvature $\text{Ric}_p(x)$ where $x \in T_pS_R$ is a unit vector, and compute the scalar curvature $K(p)$ of $S_R$.
  \item Recall from class that we can interpret the scalar curvature $K(p)$ in the following way
    \[
      \text{Vol}_M(B_r(p)) = \left( 1 - K(p) \frac{r^2}{6(n+2)} + \mathcal{O}(r^4) \right)\text{Vol}_{\mathds{R}^n}(B_r(0))
    \]
\end{enumerate}

\begin{proof}[Solution]
  \begin{enumerate}[(a)]
    \item In this case, it is easier to specify the inverse chart $\varphi^{-1}: (0, 2\pi) \times (0, \pi) \rightarrow S_R^2$
      \[
        \varphi^{-1}(\theta, \phi) = (R \cos \theta \sin \phi, R \sin \theta \sin \phi, R \cos \phi)
      \]
  Now we can find the differential of the basis vectors $ \pdv{\theta}, \pdv{\phi}$ and compute the induce metric
  \[
    \begin{aligned}
      d\varphi^{-1} \left( \pdv{\theta} \right) &= -R \sin \theta \sin \phi \pdv{x} + R \cos \theta \sin \phi \pdv{y} \\
      d \varphi^{-1} \left( \pdv{\phi} \right) &= R \cos \theta \cos \phi \pdv{x} + R \sin \theta \cos \phi \pdv{y} - R \sin \phi \pdv{z}
    \end{aligned}
  \]
  Which gives us the components of our riemannian metric as
  \[
    \begin{aligned}
      g_{\theta \theta} &= R^2\sin^2 \phi \\
      g_{\phi \phi} &= R^2  \\
      g_{\theta \phi} &= 0
    \end{aligned}
  \]
\item  
  Now we can calculate the christoffel symbols. Due to symmetry we only need to calculate 
  \[
    \Gamma_{\theta \theta}^\theta, \Gamma_{\theta \phi}^\theta, \Gamma_{\phi \phi}^\theta, \Gamma_{\theta \theta}^\phi, \Gamma_{\phi \theta}^{\phi}, \Gamma_{\phi, \phi}^\phi.
  \] First, let us calculate all possible derivatives
  \[
    \begin{aligned}
      \pdv{\theta} g_{\theta \theta} &= 0 \\
      \pdv{\theta} g_{\theta \phi} &= 0 \\
      \pdv{\theta} g_{\phi \phi} &= 0 \\
      \pdv{\phi} g_{\theta \theta} &= 2R^2 \sin \phi \cos \phi \\
      \pdv{\phi} g_{\theta \phi} &= 0 \\
      \pdv{\phi} g_{\phi \phi} &= 0
    \end{aligned}
  \]
  Furthermore we have
  \[
    \begin{aligned}
      g^{\theta \theta} &= \frac{1}{R^2 \sin^2 \phi} \\
      g^{\theta \phi} &= 0 \\
      g^{\phi \phi} &= \frac{1}{R^2}
    \end{aligned}
  \]
  Therefore, our Christoffel symbols are
  \[
    \begin{aligned}
      \Gamma_{\theta \theta}^\theta &= \frac{1}{2}\frac{1}{R^2 \sin^2 \phi} \left(0 +  0 - 0\right) = 0 \\
      \Gamma_{\theta \phi}^\theta &=  \frac{1}{2} \frac{1}{R^2 \sin^2 \phi} \left(2R^2 \sin \cos\phi + 0 - 0  \right) = \cot\phi \\
      \Gamma_{\phi \phi}^\theta &= \frac{1}{2} \frac{1}{R^2 \sin^2 \phi} \left(0 + 0 - 0  \right) = 0 \\
      \Gamma_{\theta \theta}^\phi &= \frac{1}{2} \frac{1}{R^2} \left(0 + 0 - 2R^2 \sin \phi \cos \phi \right) = -\sin \phi \cos \phi \\
      \Gamma_{\theta \phi}^\phi &= \frac{1}{2} \frac{1}{R^2} \left(0 + 0 - 0\right) = 0 \\
      \Gamma_{\phi \phi}^\phi &= \frac{1}{2} \frac{1}{R^2} \left( 0 + 0 - 0 \right) = 0
    \end{aligned}
  \]
  From this we see that
  \[
    \begin{aligned}
      \pdv{\phi} \Gamma_{\theta \theta}^\phi &= \sin^2 \phi - \cos^2 \phi \\
      \pdv{\phi} \Gamma_{\theta \phi}^\theta &= - \frac{1}{\sin^2 \phi} 
    \end{aligned}
  \]
  which are the only interesting derivatives. Now, we will calculate the components of the curvature tensor
  \[
    \begin{aligned}
      R_{\theta \theta \theta}^\theta &= \Gamma_{\theta \theta}^\theta \Gamma_{\theta \theta}^\theta + \Gamma_{\theta \theta}^\phi \Gamma_{\theta \phi}^{\theta} + \pdv{\theta} \Gamma_{\theta \theta}^\theta - \Gamma_{\theta \theta}^\theta \Gamma_{\theta \theta}^\theta - \Gamma_{\theta \theta}^\phi \Gamma_{\theta \phi}^\theta - \pdv{\theta} \Gamma_{\theta \theta}^\theta = 0\\
      R_{\theta \theta \theta}^\phi &= \Gamma_{\theta \theta}^\theta \Gamma_{\theta \theta}^\phi + \Gamma_{\theta \theta}^\phi \Gamma_{\theta \phi}^\phi + \pdv{\theta} \Gamma_{\theta \theta}^\phi - \Gamma_{\theta \theta}^\theta \Gamma_{\theta \theta}^\phi  - \Gamma_{\theta \theta}^\phi \Gamma_{\theta \phi}^\phi - \pdv{\theta} \Gamma_{\theta \theta}^\phi = 0 \\
      R_{\phi \theta \theta}^\theta &= \Gamma_{\phi \theta}^\theta \Gamma_{\theta \theta}^\theta + \Gamma_{\phi \theta}^\phi \Gamma_{\theta \phi}^\theta + \pdv{\theta} \Gamma_{\phi \theta}^\theta - \Gamma_{\theta \theta}^\theta \Gamma_{\phi \theta}^\theta  - \Gamma_{\theta \theta}^\phi \Gamma_{\phi \phi}^\theta - \pdv{\phi} \Gamma_{\theta \theta}^\theta = 0 \\
      R_{\phi \theta \theta}^\phi &= \Gamma_{\phi \theta}^\theta \Gamma_{\theta \theta}^\phi + \Gamma_{\phi \theta}^\phi \Gamma_{\theta \phi}^\phi + \pdv{\theta} \Gamma_{\phi \theta}^\phi - \Gamma_{\theta \theta}^\theta \Gamma_{\phi \theta}^\phi  - \Gamma_{\theta \theta}^\phi \Gamma_{\phi \phi}^\phi - \pdv{\phi} \Gamma_{\theta \theta}^\phi = -\sin^2 \phi \\
      R_{\theta \phi \theta}^\theta &= \Gamma_{\theta \theta}^\theta \Gamma_{\phi \theta}^\theta + \Gamma_{\theta \theta}^\phi \Gamma_{\phi \phi}^\theta + \pdv{\phi} \Gamma_{\theta \theta}^\theta - \Gamma_{\phi \theta}^\theta \Gamma_{\theta \theta}^\theta  - \Gamma_{\phi \theta}^\phi \Gamma_{\theta \phi}^\theta - \pdv{\theta} \Gamma_{\phi \theta}^\theta = 0 \\
      R_{\theta \phi \theta}^\phi &= \Gamma_{\theta \theta}^\theta \Gamma_{\phi \theta}^\phi + \Gamma_{\theta \theta}^\phi \Gamma_{\phi \phi}^\phi + \pdv{\phi} \Gamma_{\theta \theta}^\phi - \Gamma_{\phi \theta}^\theta \Gamma_{\theta \theta}^\phi  - \Gamma_{\phi \theta}^\phi \Gamma_{\theta \phi}^\phi - \pdv{\theta} \Gamma_{\phi \theta}^\phi = \sin^2 \phi \\
      R_{\phi \phi \theta}^\theta &= \Gamma_{\phi \theta}^\theta \Gamma_{\phi \theta}^\theta + \Gamma_{\phi \theta}^\phi \Gamma_{\phi \phi}^\theta + \pdv{\phi} \Gamma_{\phi \theta}^\theta - \Gamma_{\phi \theta}^\theta \Gamma_{\phi \theta}^\theta  - \Gamma_{\phi \theta}^\phi \Gamma_{\phi \phi}^\theta - \pdv{\phi} \Gamma_{\phi \theta}^\theta = 0 \\
      R_{\phi \phi \theta}^\phi &= \Gamma_{\phi \theta}^\theta \Gamma_{\phi \theta}^\phi + \Gamma_{\phi \theta}^\phi \Gamma_{\phi \phi}^\phi + \pdv{\phi} \Gamma_{\phi \theta}^\phi - \Gamma_{\phi \theta}^\theta \Gamma_{\phi \theta}^\phi  + \Gamma_{\phi \theta}^\phi \Gamma_{\phi \phi}^\phi - \pdv{\phi} \Gamma_{\phi \theta}^\phi = 0 \\
      R_{\theta \theta \phi}^\theta &= \Gamma_{\theta \phi}^\theta \Gamma_{\theta \theta}^\theta + \Gamma_{\theta \phi}^\phi \Gamma_{\theta \phi}^\theta + \pdv{\theta} \Gamma_{\theta \phi}^\theta - \Gamma_{\theta \phi}^\theta \Gamma_{\theta \theta}^\theta  - \Gamma_{\theta \phi}^\phi \Gamma_{\theta \phi}^\theta - \pdv{\theta} \Gamma_{\theta \phi}^\theta = 0 \\
      R_{\theta \theta \phi}^\phi &= \Gamma_{\theta \phi}^\theta \Gamma_{\theta \theta}^\phi + \Gamma_{\theta \phi}^\phi \Gamma_{\theta \phi}^\phi + \pdv{\theta} \Gamma_{\theta \phi}^\phi - \Gamma_{\theta \phi}^\theta \Gamma_{\theta \theta}^\phi  - \Gamma_{\theta \phi}^\phi \Gamma_{\theta \phi}^\phi - \pdv{\theta} \Gamma_{\theta \phi}^\phi = 0 \\
      R_{\phi \theta \phi}^\theta &= \Gamma_{\phi \phi}^\theta \Gamma_{\theta \theta}^\theta + \Gamma_{\phi \phi}^\phi \Gamma_{\theta \phi}^\theta + \pdv{\theta} \Gamma_{\phi \phi}^\theta - \Gamma_{\theta \phi}^\theta \Gamma_{\phi \theta}^\theta  - \Gamma_{\theta \phi}^\phi \Gamma_{\phi \phi}^\theta - \pdv{\phi} \Gamma_{\theta \phi}^\theta =  -1 \\
      R_{\phi \theta \phi}^\phi &= \Gamma_{\phi \phi}^\theta \Gamma_{\theta \theta}^\phi + \Gamma_{\phi \phi}^\phi \Gamma_{\theta \phi}^\phi + \pdv{\theta} \Gamma_{\phi \phi}^\phi - \Gamma_{\theta \phi}^\theta \Gamma_{\phi \theta}^\phi  - \Gamma_{\theta \phi}^\phi \Gamma_{\phi \phi}^\phi - \pdv{\phi} \Gamma_{\theta \phi}^\phi =  0 \\
      R_{\theta \phi \phi}^\theta &= \Gamma_{\theta \phi}^\theta \Gamma_{\phi \theta}^\theta + \Gamma_{\theta \phi}^\phi \Gamma_{\phi \phi}^\theta + \pdv{\phi} \Gamma_{\theta \phi}^\theta - \Gamma_{\phi \phi}^\theta \Gamma_{\theta \theta}^\theta  - \Gamma_{\phi \phi}^\phi \Gamma_{\theta \phi}^\theta - \pdv{\theta} \Gamma_{\phi \phi}^\theta =  1 \\
      R_{\theta \phi \phi}^\phi &= \Gamma_{\theta \phi}^\theta \Gamma_{\phi \theta}^\phi + \Gamma_{\theta \phi}^\phi \Gamma_{\phi \phi}^\phi + \pdv{\phi} \Gamma_{\theta \phi}^\phi - \Gamma_{\phi \phi}^\theta \Gamma_{\theta \theta}^\phi  - \Gamma_{\phi \phi}^\phi \Gamma_{\theta \phi}^\phi - \pdv{\theta} \Gamma_{\theta \phi}^\phi =  0 \\
      R_{\phi \phi \phi}^\theta &= \Gamma_{\phi \phi}^\theta \Gamma_{\phi \theta}^\theta + \Gamma_{\phi \phi}^\phi \Gamma_{\phi \phi}^\theta + \pdv{\phi} \Gamma_{\phi \phi}^\theta - \Gamma_{\phi \phi}^\theta \Gamma_{\phi \theta}^\theta  - \Gamma_{\phi \phi}^\phi \Gamma_{\phi \phi}^\theta - \pdv{\phi} \Gamma_{\phi \phi}^\theta =  0 \\
      R_{\phi \phi \phi}^\phi &= \Gamma_{\phi \phi}^\theta \Gamma_{\phi \theta}^\phi + \Gamma_{\phi \phi}^\phi \Gamma_{\phi \phi}^\phi + \pdv{\phi} \Gamma_{\phi \phi}^\phi - \Gamma_{\phi \phi}^\theta \Gamma_{\phi \theta}^\phi  - \Gamma_{\phi \phi}^\phi \Gamma_{\phi \phi}^\phi - \pdv{\phi} \Gamma_{\phi \phi}^\phi =  0 \\
    \end{aligned}
  \]
  There are undoubtedly more efficient ways to do this calculation, but what's done is done.
  \item Let $p \in S_r^2$. We note that by definition $\pdv{\theta}, \pdv{\phi}$ are linearly independent vectors at this point. First let us calculate $R \left(\pdv{\theta}, \pdv{\phi} \right)\pdv{\theta}$ so we have
  \[
    \begin{aligned}
      R \left(\pdv{\theta}, \pdv{\phi} \right)\pdv{\theta} &= R_{ijk}^l \delta^{i}_\theta \delta^j_\phi \delta^k_\theta \pdv{x^l}\\
                                                           &= R_{\theta \phi \theta}^\theta \pdv{\theta} + R_{\theta \phi \theta}^\phi \pdv{\phi} \\
                                                           &= \sin^2 \phi \pdv{\phi}
    \end{aligned}
  \]
  and so we have
  \[
    \left( \pdv{\theta}, \pdv{\phi}, \pdv{\theta}, \pdv{\phi} \right) = \sin^2 \phi \inn*{\pdv{\phi}}{\pdv{\phi}} = R^2 \sin^2 \phi
  \]
  likewise
  \[
    \norm{\pdv{\theta}}^2 \norm{\pdv{\phi}}^2 - \inn*{\pdv{\theta}}{\pdv{\phi}} = R^4 \sin^2 \phi
  \]
  implying that
  \[
    K(\sigma) = \frac{1}{R^2}
  \]
\item We note that when $\dim M = 2$ that $\text{Ric}_p(x) = K(\sigma) = K(p)$. So we have
  \[
    \text{Ric}_p \left( \pdv{\theta} \right) = K(\sigma) = K(p) = \frac{1}{R^2}
  \]
\item The easiest way to do this is as a surface of revolution. First we create an arc $\gamma$ parameterized by arclength, where $\gamma(0)$ is the center of the ball and and $\gamma(r)$ touches the edge of the ball. For these types of curves we know that the magnitude of $\gamma'$ is 1. Furthermore, the distance of $\gamma(t)$ to the axis of rotation must be given by $R\sin(t/R)$. So we have
  \[
    \begin{aligned}
      \text{Vol}_{S_R}(B_r(p)) &= \int_0^{r} 2\pi R \sin(t/R) \ d\theta = 2\pi R^2 \left( 1 - \cos(r/R) \right) \\
    &= \text{Vol}_{\mathds{R}^2}(B_r(0)) \left( 2R^2/r^2 - 2R^2\cos(r/R)/r^2 \right) \\
    &= \text{Vol}_{\mathds{R}^2}(B_r(0)) \left( 2R^2/r^2 - 2\frac{R^2}{r^2}\sum_{n=0}^{\infty} (-1)^n \frac{r^{2n}}{R^{2n} (2n)!} \right) \\
    &= \text{Vol}_{\mathds{R}^2}(B_r(0)) \left( 1 - 2\frac{r^2}{R^2 4!} - 2\frac{R^2}{r^2}\sum_{n=3}^{\infty} (-1)^n \frac{r^{2n}}{R^{2n} (2n)!} \right) \\
    &= \text{Vol}_{\mathds{R}^2}(B_r(0)) \left( 1 - 2K(p)\frac{r^2}{6(2 + 2)} - 2\frac{R^2}{r^2}\sum_{n=3}^{\infty} (-1)^n \frac{r^{2n}}{R^{2n} (2n)!} \right) \\
    \end{aligned}
  \]
  So our $\mathcal{O}(r^4)$ term is given by
  \[
    -2 \sum_{n=3}^{\infty} (-1)^n \frac{r^{2(n-1)}}{R^{2n}(2n)!}
  \]
  \end{enumerate}
\end{proof}

\section*{Problem 2}%
Let $S \subset \mathds{R}^3$ be a paramaterized surface, i.e. $S$ is the image of an embedding $r: A \rightarrow \mathds{R}^3$ for some open, connected region $A \subset \mathds{R}^2$. A unit normal vector to $S$ at the point $p$ can be obtained by
\[
  N(p) = \frac{r_u(p) \times r_v(p)}{\norm{r_u(p) \times r_v(p)}}
\]
where $r_u(p)$ and $r_v(p)$ denote the partial derivatives of $r(u,v)$ with respect to $u$ and $v$, evaluated at the point $(u_0, v_0)$ with $r(u_0, v_0) = p$. The normal vector $N(p)$ defines a map $N: S \rightarrow S^2$, called the gauss map.

As the tangent planes $T_pS$ and $T_{N(p)}S^2$ are parallel, we can think of the differential of the Gauss map as giving a map
\[
  dN_p: T_pS \rightarrow T_{N(p)}S^2 \cong T_pS
\]
The eigenvalues $\kappa_1$ and $\kappa_2$ are called the principal curvatures of $S$ at $p$, while the determinant $\det(dN_p)$ is called the Gaussian curvature of $S$ at $p$.

Compute the principal curvatures and the Gaussian curvatures of the following surfaces in $\mathds{R}^3$:
\begin{enumerate}[(a)]
  \item The plane $ax + by + cz = d$, where $a,b,c,d \in \mathds{R}$.
  \item The sphere $S_R$ of radius $R$.
  \item The cylinder with $r(u,v) = (\cos u, \sin u, v)$.
  \item The hyperbolic paraboloid with $r(u,v) = (u, v, v^2 - u^2)$, at the point $p = (0,0,0)$.
\end{enumerate}

A point $p \in S$ is called elliptic if $\det(dN_p) > 0$, hyperbolic if $\det(dN_p) < 0$, parabolic if $\det(dN_p) = 0$ but $dN_p \neq 0$, and planar if $dN_p = 0$.

\begin{proof}[Solution]
  The computations are as follows  
  \begin{enumerate}[(a)]
    \item If we let $u = x$ and $v = y$, and assuming $c \neq 0$ then we have that $z = \frac{d - au - bv}{c}$. We get the parametrization
      \[
        r(u,v) = (u, v, (d - au - bv)/c)
      \]
      Taking our partials, we get
      \[
        \begin{aligned}
          r_u(u,v) &= (1, 0, -a/c) \\
          r_v(u,v) &= (0, 1, -b/c)
        \end{aligned}
      \]
      computing the cross product gives
      \[
        r_u(u,v) \times r_v(u,v) = (a/c, b/c, 1)
      \]
      and normalizing gives
      \[
        N(p) = \frac{1}{\sqrt{a^2 + b^2 + c^2}} (a, b, c)
      \]
      It is easy to see that the differential is given by
      \[
        dN_p = 0
      \]
      since there is no dependence on $u,v$. So this point is planar (makes sense).
    \item Under the map
      \[
        r(u,v) = (R \cos u \sin v, R \sin u \sin v, R \cos v)
      \]
      The normal vector is
      \[
        N(p) = (\cos u \sin v, \sin u \sin v, \cos v)
      \]
      since the normal vector is always in the direction of the vector that points to the point on the manifold. Taking the differential yields
      \[
        \begin{aligned}
          dN_p \left( \pdv{u} \right) &= -\sin u \sin v \pdv{x} +  \cos u \sin v \pdv{y} \\
          dN_p \left( \pdv{v} \right) &= \cos u \cos v \pdv{x} + \sin u \cos v \pdv{y} - \sin v \pdv{z}  \\
        \end{aligned}
      \]
      we also note that 
      \[
        \begin{aligned}
          dr \left( \pdv{u} \right) &= -R\sin u \sin v \pdv{x} +  R\cos u \sin v \pdv{y} \\
          dr \left( \pdv{v} \right) &= R\cos u \cos v \pdv{x} + R\sin u \cos v \pdv{y} - R\sin v \pdv{z}  \\
        \end{aligned}
      \]
      So we can naturally associate $T_{N(p)}S^2$ with $T_pS_R$ and say that
      \[
        \begin{aligned}
          dN_p \left( \pdv{u} \right) &= \frac{1}{R} \pdv{u} \\
          dN_p \left( \pdv{v} \right) &= \frac{1}{R} \pdv{v}
        \end{aligned}
      \]
      and we can write it in matrix form as
      \[
        dN_p =
        \begin{bmatrix}
          \frac{1}{R} & 0 \\
          0 & \frac{1}{R}
        \end{bmatrix}
      \]
      So the principal curvatures are $\kappa_1 = \frac{1}{R}$ and $\kappa_2 = \frac{1}{R}$, and the gaussian curvature is given by
      \[
        \det \left( dN_p \right) = \frac{1}{R^2}
      \]
    \item Let us first analyze $r_u$ and $r_v$.
      \[
        \begin{aligned}
          r_u &= (-\sin u, \cos u, 0) \\
          r_v &= (0, 0, 1)
        \end{aligned}
      \]
      And we have that
      \[
        r_u \times r_v = (\cos u, \sin u, 0)
      \]
      which is already normalized. So we have
      \[
        N(p) = (\cos(u), \sin(u), 0)
      \]
      taking the differential, we have
      \[
        \begin{aligned}
          dN_p \left( \pdv{u} \right) &= -\sin u  \pdv{x} + \cos u \pdv{y} \\
          dN_p \left( \pdv{v} \right) &= 0
        \end{aligned}
      \]
      however we also have
      \[
        dr \left( \pdv{u} \right) = -\sin u \pdv{x} + \cos u \pdv{y}
      \]
      so we can associate $T_{N(p)}S^2$ with $T_pS$
      \[
        \begin{aligned}
        dN_p \left( \pdv{u}  \right) = \pdv{u} \\
        dN_p \left( \pdv{v} \right) = 0
        \end{aligned}
      \]
      which gives us the matrix in components as
      \[
        dN_p = 
        \begin{bmatrix}
          1 & 0 \\
          0 & 0
        \end{bmatrix}
      \]
      which has principal curvature of $\kappa_1 = 1$ and $\kappa_2 = 0$ and gaussian curvature of 
      \[
        \det \left(dN_p\right) =  0
      \]
      which implies that a cylinder is parabolic.
    \item We will first analyze $r_u$ and $r_v$
      \[
        \begin{aligned}
          r_u &= (1, 0, -2u) \\
          r_v &= (0, 1, 2v)
        \end{aligned}
      \]
      And we have that
      \[
        r_u \times r_v = (2u, -2v, 1)
      \]
      which gives us the normalized vector as
      \[
        N(p) = \frac{1}{\sqrt{4u^2 + 4v^2 + 1}} (2u, -2v, 1)
      \]
      so the differential tells us that
      \[
        \begin{aligned}
          dN_p \left( \pdv{u} \right) &= \frac{8v^2 + 2}{(4u^2 + 4y^2 + 1)^{3/2}} \pdv{x} + \frac{8uv}{(4u^2 + 4v^2 + 1)^{3/2}}\pdv{y} \\
          dN_p \left( \pdv{v}  \right) &= - \frac{8uv}{(4u^2 + 4v^2 + 1)^{3/2}} \pdv{x} - \frac{8u^2 + 2}{(4u^2 + 4v^2 + 1)^{3/2}} \pdv{y}
        \end{aligned}
      \]
      Specifically at the point $r(0,0) = (0,0,0)$ we have
      \[
        \begin{aligned}
          dN_0 \left( \pdv{u} \right) = 2 \pdv{x} \\
          dN_0 \left( \pdv{v} \right) = -2 \pdv{y}
        \end{aligned}
      \]
      and at this same point
      \[
        \begin{aligned}
          dr \left( \pdv{u} \right) &=  \pdv{x} \\
          dr \left( \pdv{v} \right) &= \pdv{y}
        \end{aligned}
      \]
      so we can associate $T_{N(p)}S^2$ with $T_pS$
      \[
        \begin{aligned}
          dN_0 \left(\pdv{u}\right) &= 2 \pdv{u} \\
          dN_0 \left(\pdv{v}  \right) &= -2 \pdv{v}
        \end{aligned}
      \]
      which gives us the matrix in components as
      \[
        dN_0 =
        \begin{bmatrix}
          2 & 0 \\
          0 & -2
        \end{bmatrix}
      \]
      giving us the principal curvatures $\kappa_1 = 2$, $\kappa_2 = -2$ and the gaussian curvature
      \[
        \det(dN_0) = -4
      \]
      Thus, this point is hyperbolic.
  \end{enumerate}
\end{proof}

\section*{Problem 3}%
Prove the 2nd Bianchi Identity:
\[
  \nabla R(X,Y,Z,W,T) + \nabla R(X,Y,W,T,Z) + \nabla R(X,Y,T,Z,W) = 0
\]
for all $X,Y,Z,W,T \in \Gamma^{\infty}(TM)$.

\begin{proof}
  We note that in this case
  \[
    \nabla: \Gamma^{\infty}(TM) \times \Gamma^{\infty}(T_4M) \rightarrow \Gamma^{\infty}(T_4M)
  \]
  Or in other words
  \[
  \nabla R(X,Y,Z,W,T) + \nabla R(X,Y,W,T,Z) + \nabla R(X,Y,T,Z,W)
  \]
  is a $(0,4)$ tensor field. Therefore, we can prove the equality at an arbitrary point $p \in M$ and it will hold on the entire manifold. At this point we note that there exists a geodesic frame $\left\{ e_i \right\}$ (local basis) at $p$, such that
  \[
    \nabla_{e_i} e_j(p) = 0
  \]
  Therefore, let us examine
  \[
    \begin{aligned}
      \nabla R(e_i, e_j, e_k, e_l, e_h) =& e_h( R(e_i, e_j, e_k, e_l))   \\
                                         &- R(\nabla_{e_h}e_i, e_j, e_k, e_l) \\
                                         &- R(e_i, \nabla_{e_h}e_j, e_k, e_l) \\
                                         &- R(e_i, e_j, \nabla_{e_h}e_k, e_l) \\
                                         &- R(e_i, e_j, e_k, \nabla_{e_h}e_l)
    \end{aligned}
  \]
  Now we note that since we are using the geodesic frame that $\nabla_{e_i}e_j = 0$
  \[
    \begin{aligned}
      \nabla R(e_i, e_j, e_k, e_l, e_h) &= e_h( R(e_i, e_j, e_k, e_l)) \\ 
                                        &= e_h \left( \inn*{R(e_i, e_j)e_k}{e_l} \right) \\
                                        &= e_h \left( \inn*{R(e_k, e_l)e_i}{e_j} \right) \\
                                        &= e_h \left( \inn*{\nabla_{e_l}\nabla_{e_k}e_i - \nabla_{e_k}\nabla_{e_l}e_i + \nabla_{[e_k,e_l]}e_i}{e_j}\right)
    \end{aligned}
  \]
  Furthermore, since $\nabla$ is compatible with the metric, we have
  \[
    \begin{aligned}
    \nabla R(e_i,e_j,e_k,e_l,e_h) =& \inn*{\nabla_{e_h}\nabla_{e_l}\nabla_{e_k}e_i - \nabla_{e_h}\nabla_{e_k}\nabla_{e_l}e_i + \nabla_{e_h}\nabla_{[e_k,e_l]}e_i}{e_j} \\
                                   &+ \inn*{\nabla_{e_l}\nabla_{e_k}e_i - \nabla_{e_k}\nabla_{e_l}e_i + \nabla_{[e_k,e_l]}e_i}{\nabla_{e_h}e_j} \\
                                    =& \inn*{\nabla_{e_h}\nabla_{e_l}\nabla_{e_k}e_i - \nabla_{e_h}\nabla_{e_k}\nabla_{e_l}e_i + \nabla_{e_h}\nabla_{[e_k,e_l]}e_i}{e_j} \\
    \end{aligned}
  \]
  So now let us consider $\nabla R(e_i, e_j, e_k, e_l, e_h) + \nabla R(e_i,e_j, e_l, e_h, e_k) + \nabla  R(e_i, e_j, e_h, e_k, e_l)$, by the definition of curvature, we have
  \[
    \begin{aligned}
      \nabla R(e_i, e_j, e_k, e_l, e_h) &+ \nabla R(e_i,e_j, e_l, e_h, e_k) + \nabla  R(e_i, e_j, e_h, e_k, e_l)  \\ 
      =& \langle \nabla_{e_h}\nabla_{e_l}\nabla_{e_k}e_i - \nabla_{e_l}\nabla_{e_h}\nabla_{e_k}e_i \\ 
      &+ \nabla_{e_l}\nabla_{e_k}\nabla_{e_h}e_i - \nabla_{e_k}\nabla_{e_l}\nabla_{e_h}e_i \\
      &+ \nabla_{e_k}\nabla_{e_h}\nabla_{e_l}e_i - \nabla_{e_h}\nabla_{e_k}\nabla_{e_l}e_i \\
      &+ \nabla_{e_h} \nabla_{[e_k, e_l]} e_i + \nabla_{e_k}\nabla_{[e_l,e_h]}e_i + \nabla_{e_l} \nabla_{[e_h,e_k]}e_i, e_j\rangle \\
      =& \langle R(e_l, e_h)\nabla_{e_k}e_i - \nabla_{[e_l,e_h]}\nabla_{e_k}e_i \\
       &+ R(e_k, e_l) \nabla_{e_h}e_i - \nabla_{[e_k,e_l]}\nabla_{e_h}e_i \\
       &+ R(e_h, e_k) \nabla_{e_l}e_i - \nabla_{[e_h,e_k]}\nabla_{e_l}e_i \\
       &+\nabla_{e_h} \nabla_{[e_k, e_l]} e_i + \nabla_{e_k}\nabla_{[e_l,e_h]}e_i + \nabla_{e_l} \nabla_{[e_h,e_k]}e_i, e_j\rangle \\
      =& \langle R(e_l, e_h) \nabla_{e_k}e_i + R(e_k, e_l) \nabla_{e_h}e_i + R(e_h, e_k)\nabla_{e_l}e_i \\
       &+ R([e_k,e_l], e_h)e_i + R([e_l,e_h],e_k)e_i + R([e_h,e_l],e_k)e_i\\
       &- \nabla_{[[e_k,e_l],e_h]}e_i - \nabla_{[[e_l, e_h],e_k]}e_i - \nabla_{[[e_h,e_k],e_l]}e_i, e_j \rangle \\
      =& \langle R(e_l, e_h) \nabla_{e_k}e_i + R(e_k, e_l) \nabla_{e_h}e_i + R(e_h, e_k)\nabla_{e_l}e_i \\
       &+ R([e_k,e_l], e_h)e_i + R([e_l,e_h],e_k)e_i + R([e_h,e_l],e_k)e_i\\
       &- \nabla_{[[e_k,e_l],e_h] + [[e_l,e_h],e_k] + [[e_h,e_k]e_l]}e_i, e_j \rangle
    \end{aligned}
  \]
  Now we note that since $R$ is a tensor it only depends on its arguments at the point. Which implies that the terms $R(e_l, e_h)\nabla_{e_k}e_i$ are zero since $\nabla_{e_k}e_i =0$ at $p$. Furthermore we can use the jacobi identity on the last term leaving us with
  \[
    \begin{aligned}
      \nabla R(e_i, e_j, e_k, e_l, e_h) &+ \nabla R(e_i,e_j, e_l, e_h, e_k) + \nabla  R(e_i, e_j, e_h, e_k, e_l)  \\ 
      =& \inn*{R([e_k,e_l],e_h)e_i + R([e_l,e_h]e_k)e_i + R([e_h,e_l],e_k)e_i}{e_j} \\
      =& \inn*{R(\nabla_{e_l}e_k - \nabla_{e_k}e_l, e_h)e_i + R(\nabla_{e_h}e_l - \nabla_{e_l}e_h,e_k)e_i + R(\nabla_{e_l}e_h - \nabla_{e_h}e_l,e_k)e_i}{e_j} \\
      =& 0
    \end{aligned}
  \]
  Since $R$ is a tensor and $\nabla_{e_l}e_k = 0$ at p. Now let $X = X^i e_i$, $Y = Y^j e_j$, $Z = Z^k e_k$, $W = W^l e_l$, $T = T^k e_k$. We have that
  \[
    \begin{aligned}
      \nabla R(X,Y,Z,W,T) + \nabla R(X,Y,W,T,Z) + \nabla R(X,Y,T,Z,W) &= \\
      \nabla R(X^i e_i,Y^j e_j,Z^k e_k,W^l e_l,T^h e_h) + \nabla R(X^i e_i,Y^j e_j,W^l e_l,T^h e_h,Z^k e_k) + \nabla R(X^i e_i,&Y^j e_j,T^h e_h,Z^k e_k,W^l e_l) = \\
      X^i Y^j Z^k W^l T^h \left( \nabla R(e_i, e_j, e_k, e_l, e_h) + \nabla R(e_i,e_j, e_l, e_h, e_k) + \nabla  R(e_i, e_j, e_h, e_k, e_l) \right)  \\
      X^i Y^j Z^k W^l T^h(0) = 0 &
    \end{aligned}
  \]
\end{proof}

\section*{Problem 4}%
Let $M$ be a connected Rimannian manfiold with $n \geq 3$. Suppose that $M$ is isotropic, that is, for each $p \in M$, the sectional curvature $K(p,\sigma)$ does not depend on $\sigma \subset T_pM$. Prove that $M$ has constant sectional curvature, that is, $K(p, \sigma)$ also does not depend on $p$.

\begin{proof}
  Let us consider the tensor field given by
  \[
    R'(W,Z,X,Y) = \inn*{W}{X}\inn*{Z}{Y} - \inn*{Z}{X}\inn*{W}{Y}
  \]
  Then we can invoke lemma 5.4.3 and say that $R = KR'$ pointwise ($R(p) = K(p,\sigma)R'(p)$).  Now let $T \in \Gamma^\infty(TM)$. We will take the covariant derivative of each side of this equality with respect to $T$. So we get
  \[
    \begin{aligned}
      \nabla_T R = \nabla_T \left(KR'\right) =&  T(KR'(W,Z,X,Y)) \\
                                             &- KR'(\nabla_T W, Z,X,Y) \\
                                             &- KR'(W, \nabla_T Z,X,Y) \\
                                             &- KR'(T W, Z,\nabla_T X,Y) \\
                                             &- KR'(W, Z,X,\nabla_T Y) \\
      =& T(KR'(W,Z,X,Y)) \\
       &- K\inn*{\nabla_T W}{X}\inn*{Z}{Y} - K\inn*{Z}{X}\inn*{\nabla_T W}{Y} \\
       &- K\inn*{W}{X}\inn*{\nabla_T Z}{Y} - K\inn*{\nabla_T Z}{X}\inn*{W}{Y} \\
       &- K\inn*{W}{\nabla_T X}\inn*{Z}{Y} - K\inn*{Z}{X}\inn*{W}{Y} \\
       &- K\inn*{W}{X}\inn*{Z}{\nabla_T Y} - K\inn*{Z}{X}\inn*{W}{\nabla_T Y}
    \end{aligned}
  \]
  Since $\nabla_T$ is Levi-Civita we have compatibility with the metric and we get
  \[
    \begin{aligned}
      \nabla_T R = \nabla_T \left(KR'\right) =&  T(KR'(W,Z,X,Y)) \\
                                              &- KT \left( \inn*{W}{Z} \right)\inn*{Z}{Y} \\
                                              &- K\inn*{Z}{X} T \left( \inn*{W}{Y} \right) \\
                                              &- K\inn*{W}{X} T \left( \inn*{Z}{Y} \right) \\
                                              &- KT \left( \inn*{Z}{X} \right) \inn*{W}{Y}
    \end{aligned}
  \]
  Now treating T like a derivation and expanding the first term gives us
  \[
    \begin{aligned}
      \nabla_T R = \nabla_T \left(KR'\right) =&  T(K)R'(W,Z,X,Y) \\
                                              &+ KT(\inn*{W}{X})\inn*{Z}{Y} \\
                                              &+ K\inn*{W}{X}T(\inn*{Z}{Y}) \\
                                              &+ KT(\inn*{Z}{X})\inn*{W}{Y} \\
                                              &+ K\inn*{Z}{X}T(\inn*{W}{Y}) \\ 
                                              &- KT \left( \inn*{W}{Z} \right)\inn*{Z}{Y} \\
                                              &- K\inn*{Z}{X} T \left( \inn*{W}{Y} \right) \\
                                              &- K\inn*{W}{X} T \left( \inn*{Z}{Y} \right) \\
                                              &- KT \left( \inn*{Z}{X} \right) \inn*{W}{Y}\\
      =& (TK)(R'(W,Z,X,Y))
    \end{aligned}
  \]
  We can write this in abbreviated form as $\nabla_T R = T(K)R'$. Now applying the second bianchi identity we just proved, we have
  \[
    \begin{aligned}
      0 =& T(K) \left( \inn*{W}{X}\inn*{Z}{Y} - \inn*{Z}{X}\inn*{W}{Y} \right) \\
         &+ X(K) \left( \inn*{W}{Y}\inn*{Z}{T} - \inn*{Z}{Y}\inn*{W}{T} \right) \\
         &+ Y(K) \left( \inn*{W}{T}\inn*{Z}{X} - \inn*{Z}{T}\inn*{W}{X} \right) \\
    \end{aligned}
  \]
  which inherently holds for all $X,Y,Z,W,T \in \Gamma^\infty (TM)$. Now let $p \in M$ be a fixed arbitrary point. Given the metric and since $n \geq 3$, we can choose $Y \neq 0$ and $Z$ at $p$ such that$ \inn*{X}{Y} = \inn*{Y}{Z} = \inn*{Z}{X} = 0$  and  $\inn*{Z}{Z} = 1$ (gram schmidt). Furthermore let $T = Z$ at $p$. Now we have that the above expression is
  \[
    \begin{aligned}
      0 &= T(K)\left( \inn*{W}{X}0 - 0\inn*{W}{Y} \right) + X(K) \left(\inn*{W}{Y}1 - 0\inn*{W}{Z} \right) + Y(K)\left(\inn*{W}{Z}0 - 1\inn*{W}{X} \right) \\
        &= X(K) (\inn*{W}{Y}) - Y(K)(\inn*{W}{X}) \\
        &= \inn*{W}{X(K)Y - Y(K)X}
    \end{aligned}
  \]
  Since this relation holds for arbitrary $W$, we have $X(K)Y - Y(K)X$, since $X \perp Y$ and $Y \neq 0$ we also have that $X,Y$ are linearly independent. This implies that $X(K)Y = 0$. However, since $Y \neq 0$, we have $X(K) = 0$ for all $X \in \Gamma^{\infty}(TM)$. Thinking of $X$ as a derivative, it follows that $K(p) $ is constant on a neighborhood of $p$. Since $M$ is connected, it is also path connected which means given any $q \in M$, there exists a $\gamma: [0,1] \rightarrow M$, such that $\gamma(0) = p$ and $\gamma(1) = q$. Since $K$ is constant along a neighborhood of $p$ it must be constant along this curve and thus $K(p) = K(q)$. Since $q \in M$ is arbitrary we have $K$ is constant on $M$.
\end{proof}



\end{document}
