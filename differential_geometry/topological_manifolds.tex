%!TEX TS-program = xelatex
%!TEX encoding = UTF-8 Unicode

\documentclass[a4paper]{article}

\usepackage{xltxtra}
\usepackage{amsfonts}
\usepackage{polyglossia}
\usepackage{fancyhdr}
\usepackage{geometry}
\usepackage{dsfont}
\usepackage{amsmath}
\usepackage{amsthm}
\usepackage{amssymb}

\geometry{a4paper,left=15mm,right=15mm,top=20mm,bottom=20mm}
\pagestyle{fancy}
\lhead{Devon Morris}
\chead{Differential Geometry}
\rhead{\today}
\cfoot{\thepage}

\setlength{\headheight}{23pt}
\setlength{\parindent}{0.0in}
\setlength{\parskip}{0.0in}

\newtheorem{prop}{Proposition}
\begin{document}

\section*{Topological Manifolds}
Recall t a topological space $X$ is called hausdorff if for every $p,q \in X$ there exist open neighborhoods $U$ and $V$ containing $p,q$ respectively s.t. $U \cap V = \varnothing$.

A basis for a topological space $X$ is a collection $\mathcal{B}$ of open subsets of $X$ such that for  every open set $U \subset X$ and every point $p \in U$, there exists a set $B \in \mathcal{B}$ such that $p \in B \subset U$. (Collection of open sets to build other open sets). $\Rightarrow$ all open sets in $X$ can be written as union of elements in $\mathcal{B}$.

Example: For any $n$, $\mathds{R}^n$, has a basis consisting of 
\[
    \mathcal{B} = \{\text{open balls with rational centers and rational radii}\}
\]
(Want properties second countable)

A space $X$ is second countable if it admits a basis with countable cardinality. $\mathds{R}^n$ is second countable with the basis defined above.

Let $M$ be a topological space. An $n$-dimensional coordinate chart is a pair $(U, \varphi)$, where $U \subset M$ and the map $\varphi: U \rightarrow \mathds{R}^n$ is a continuous map which is a homeomorphism onto its image. If $p \in U$ we call $\varphi$ a chart about $p$. $\varphi$ is called a coordinate map and if $(x^1, x^2, \dots, x^n)$ are the component functions of $\varphi$ (i.e. $\varphi(p) = (x^1(p), \dots, x^n(p))$), then I call $x^1, \dots, x^n$ local coordinates on $U$. (Think of the local coordinates at the pre-image of the coordinates in $\mathds{R}^n$.
An $n$-dimensional topological manifold $M$ is a Hausdorff second countable topological space, such that for every $p \in M$ there exists a $n$-dimensional coordinate chart about $p$. Every point of $M$ has a neighborhood that looks like $\mathds{R}^n$ (locally euclidean). We can think of $p$ being identified with the point $\varphi(p) = (x^1(p), \dots, x^n(p))$. Note: Do Carmo's maps go the other way and he calls them local parameterizations. He uses $\mathbf{x} : V \rightarrow \mathbf{x}(V)$ where $V \subset \mathds{R}^n$ and $\mathbf{x}(V) \subset M$.

\subsection*{Examples}
\begin{enumerate}
    \item
    $\mathds{R}^n$ is a topological manifold with $\text{id}_{\mathds{R}^n}: \mathds{R}^n \rightarrow \mathds{R}^n$. 
    \item
        The sphere, locally is like $\mathds{R}^2$. Let $S^n = \{(x^1, x^2, \dots, x^{n+1}) \in \mathds{R}^{n+1} | (x^1)^2 + \dots + (x^{n+1})^2 = 1\}$. Let $U_j^+  = \{(x_1, x_2, x_3) \in S^2 | x^j > 0\}$ and $U_j^- = \{(x^1, x^2, x^3) | x^j > 0\}$, make sure they cover the manifold. $\varphi_1^+: U_1^+ \rightarrow \mathds{R}^2, (x^1, x^2, x^3) \rightarrow (x^2, x^3)$. Note that the others are similar.
    \item Let $(r, \theta, z)$ be cylindrical "coordinates" on $\mathds{R}^3$. Define $T^2 = \{(r,\theta, z) \in \mathds{R}^3 | z^2 + (r-2)^2 = 1\} \subset \mathds{R}^3$. Find a local parameterization for $T^2$. $\mathbf{x}(y^1,y^2) = (\sin y^2 + 2,y^1, \cos y^2)$. We can rotate $\mathbf{x}$ to cover $T^2$ in 3 charts.
    \item Let $Z = \{(x,y,z) \in \mathds{R}^3 | z^2 = x^2 + y^2\}.$ The point $(0,0,0)$ has no coordinate chart, not a manifold.
    \item $M,N$ manifolds $U \subset M$  is open, $M \times N$. $U$ manifold  $\varphi_M|_u$. This is only true in general if $U$ is an open set.
\end{enumerate}

\subsection*{Graphs of manifolds}
Let $U \subset \mathds{R}^m$ be open, $f: U \rightarrow \mathds{R}^n$ be continous the graph $\Gamma(f) = \{(x, f(x)) | x \in U\}$. The claim is that $\Gamma(f)$ is a topological manifold. Let $\pi_m : \mathds{R}^{m+n} \rightarrow \mathds{R}$ be the projection to the first m coordinates. Let $\varphi_f = \pi_m |_{\Gamma(f)}$ this is a global coordinate chart.

\end{document}
