%!TEX TS-program = xelatex
%!TEX encoding = UTF-8 Unicode

\documentclass[a4paper]{article}

\usepackage{xltxtra}
\usepackage{amsfonts}
\usepackage{polyglossia}
\usepackage{fancyhdr}
\usepackage{geometry}
\usepackage{dsfont}
\usepackage{amsmath}
\usepackage{amsthm}
\usepackage{amssymb}

\geometry{a4paper,left=15mm,right=15mm,top=20mm,bottom=20mm}
\pagestyle{fancy}
\lhead{Devon Morris}
\chead{Differential Geometry}
\rhead{\today}
\cfoot{\thepage}

\setlength{\headheight}{23pt}
\setlength{\parindent}{0.0in}
\setlength{\parskip}{0.0in}

\newtheorem{prop}{Proposition}

\begin{document}

\section*{Topological Manifolds}
Recall t a topological space $X$ is called hausdorff if for every $p,q \in X$ there exist open neighborhoods $U$ and $V$ containing $p,q$ respectively s.t. $U \cap V = \varnothing$.

A basis for a topological space $X$ is a collection $\mathcal{B}$ of open subsets of $X$ such that for  every open set $U \subset X$ and every point $p \in U$, there exists a set $B \in \mathcal{B}$ such that $p \in B \subset U$. (Collection of open sets to build other open sets). $\Rightarrow$ all open sets in $X$ can be written as union of elements in $\mathcal{B}$.

Example: For any $n$, $\mathds{R}^n$, has a basis consisting of 
\[
    \mathcal{B} = \{\text{open balls with rational centers and rational radii}\}
\]
(Want properties second countable)

A space $X$ is second countable if it admits a basis with countable cardinality. $\mathds{R}^n$ is second countable with the basis defined above.

\end{document}
