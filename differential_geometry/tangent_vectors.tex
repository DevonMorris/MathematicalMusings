%!TEX TS-program = xelatex
%!TEX encoding = UTF-8 Unicode

\documentclass[a4paper]{article}

\usepackage{xltxtra}
\usepackage{amsfonts}
\usepackage{polyglossia}
\usepackage{fancyhdr}
\usepackage{geometry}
\usepackage{dsfont}
\usepackage{amsmath}
\usepackage{amsthm}
\usepackage{amssymb}
\usepackage{physics}

\geometry{a4paper,left=15mm,right=15mm,top=20mm,bottom=20mm}
\pagestyle{fancy}
\lhead{Devon Morris}
\chead{Differential Geometry}
\rhead{\today}
\cfoot{\thepage}

\setlength{\headheight}{23pt}
\setlength{\parindent}{0.0in}
\setlength{\parskip}{0.0in}

\newtheorem*{prop}{Proposition}
\newtheorem*{defn}{Definition}
\begin{document}

\section*{Tangent Vectors}
Basically we think of it as a vector sticking off the manifold in a "tangent direction". We can't define this on an ambient euclidean space, since we want to think of a manifold as an independent thing. This can be done as derivations on the space of smooth functions, but we will do it differently in this class. 

\begin{defn}
    A smooth curve in $M$ is a smooth map $\gamma: (-\epsilon, \epsilon) \rightarrow M$, for some $\epsilon > 0$. 
\end{defn}
If $\gamma_1$ and $\gamma_2$ are two curves such that $\gamma_1(0) = \gamma_2(0) = p \in M$, we write $\gamma_1 \sim_{p} \gamma_2$ if $(f \circ \gamma_1)'(0) = (f \circ \gamma_2)'(0)$ for all $f \in C^{\infty}(M)$. So if we think $f \circ \gamma_1: (-\epsilon, \epsilon) \rightarrow \mathds{R}$. So the derivative at $0$ makes sense. If we do this in coordinate representations we can define $\tilde{\gamma}(t) = \varphi \circ \gamma(t) = (\gamma^1(t), \dots, \gamma^n(t)) = (x^1(\gamma(t)), \dots, x^n(\gamma(t)))$ and $\tilde{f}(x^1, \dots, x^n) = f \circ \varphi^{-1}(x^1, \dots, x^n)$, which allows us to use the multivariate chain rule as we will show below.

\begin{defn}
    A tangent vector to $M$ at $p \in M$ is an equivalence class of curves $[\gamma]_p$ under the relation $\sim_p$. We let $T_pM$ denote the set of all tangent vectors to $M$ at $p$.
\end{defn}
Soon we will show that $T_pM$ that it is a vector space, and we will construct an isomorphism to $\mathds{R}^n$ using coordinate charts. Let $(U, \varphi)$ be a chart on $M$ about $p$ and let 
\[
    \varphi(q) = (x^1(q), \dots, x^n(q))
\]
Define a map $\theta_{\varphi}: T_pM \rightarrow \mathds{R}^n$ as follows. For any smooth function $f \in C^{\infty}(M)$ and curve $\gamma$ let $\tilde{f} = f \circ \varphi^{-1}: \varphi(U) \rightarrow \mathds{R}$, and let $\tilde{\gamma} = \varphi \circ \gamma: (-\epsilon, \epsilon) \rightarrow \mathds{R}^n$. We call these coordinate representation of $\gamma$ and $f$
\[
    \begin{aligned}
      \tilde{\gamma} &= (\tilde{\gamma}^1(t), \dots, \tilde{\gamma}^n(t)) = \left( x^1(\gamma(t)), \dots, x^n(\gamma(t)) \right)\\
      \tilde{f}&(x^1, \dots, x^n)
    \end{aligned}
\]
So we get 
\[
    \begin{aligned}
      (f \circ \gamma)'(0) &= (f \circ \varphi^{-1} \circ \varphi \circ \gamma)'(0) = \dv[]{}{t}(\tilde{f} \circ \tilde{\gamma})(\tilde{f} \circ \tilde{\gamma})(0) \\
                           &= \sum_{j=1}^n \frac{\partial \tilde{f}}{\partial x^j}\left(\tilde{\gamma}(0)\right) \dv[]{\tilde{\gamma}^j}{t}(0) \\
                           &= \sum_{j=1}^n \left(\dv[]{\tilde{\gamma}^j}{t}(0) \eval{\frac{\partial}{\partial x^j}}_{\tilde{\gamma}(0)}  \right)\tilde{f} \\
    \end{aligned}
\]
Now let $v^j = \dv[]{\gamma^j}{t}(0)$. Then
\[
  (f \circ \gamma)'(0) = \sum_{j=1}^n \left(v^j \eval{\frac{\partial}{\partial x^j}}_{\tilde{\gamma}(0)}  \right)\tilde{f} \\
\]
Now define $\theta_{\varphi}([\gamma]_p) = [v^1, v^2, \dots, v^n] = v$, where $v = \left[\dv[]{\tilde{\gamma}^1}{t}(0), \dots, \dv{\tilde{\gamma}^n}{t}(0)\right]$. We must observe well-definedness of $\theta$.
\begin{proof}
    Let $[\gamma_1]_p = [\gamma_2]_p$, then $(f \circ \gamma_1)'(0) = (f \circ \gamma_2)'(0)$ for $f \in C^{\infty}(M)$. Then we plug in $x^j: M \rightarrow \mathds{R}$, we get
    \[
      (f \circ \gamma_1)'(0) = \sum_j \dv[]{\tilde{\gamma}^j_1}{t}(0) \frac{\partial \tilde{f}}{\partial x^j}(\tilde{\gamma}(0))  = \sum_j \dv{\tilde{\gamma}^j_2}{t}(0) \frac{\partial \tilde{f}}{\partial x^j}(\tilde{\gamma}(0)) = (f \circ \gamma_2)'(0)
    \]
    Specifically, we can choose our parameterizations on the chart $f = x^i$, so $\tilde{f}(x^1, \dots, x^n) = \tilde{x}^i(x^1, \dots, x^n) = x^i$. Substituting this in our equation gives us that
    \[
      \sum_j \dv{\tilde{\gamma}^j_1}{t}(0) \frac{\partial x^i}{\partial x^j}(\tilde{\gamma}(0)) = \dv[]{\tilde{\gamma}^i_1}{t} = \dv[]{\tilde{\gamma}^i_2}{t} = \sum_j \dv{\tilde{\gamma}^j_2}{t}(0) \frac{\partial x^i}{\partial x^j}(\tilde{\gamma}(0))
    \]
    Since $i$ was arbitrary, we have that the all components must be equal. So we get $\theta_{\varphi}([\gamma_1]) = \theta_{\varphi}([\gamma_2])$ so $\theta_\varphi$ is well defined.
\end{proof}
We also claim that $\theta_{\varphi}: T_pM \rightarrow \mathds{R}^n$ is an isomorphism
\begin{proof}
    Injective: if $[\gamma_1] \neq [\gamma_2]$, then there exists some $f \in C^{\infty}(M)$ such that 
    \[
        (f \circ \gamma_1)'(0) \neq (f \circ \gamma_2)'(0)
    \]
    This show thats 
    \[
      \dv{\tilde{\gamma_1}^j}{t}(0) \neq \dv{\tilde{\gamma_2}^j}{t}
    \]
    for some $j$ so we have $\theta_{\varphi}([\gamma_1]) \neq \theta_{\varphi}([\gamma_2])$.

    Surjective: let $v = [v^1, \dots, v^n] \in \mathds{R}^n$, define $\tilde{\gamma}:(-\epsilon, \epsilon) \rightarrow \mathds{R}^n$ by 
    \[
        \tilde{\gamma}(t) = \varphi(p) + tv
    \]
    and $\gamma = \varphi^{-1} \circ \tilde{\gamma}: (-\epsilon, \epsilon) \rightarrow M$. Then $\theta_{\varphi}([\gamma]) = v$.
\end{proof}
So $\theta_p$ induces a vector space structure on $T_pM$, which we now call the tangent space to $M$ at $p$.

Now lets let $ \left\{ e_1, \dots, e_n \right\}$ be the standard basis for $\mathds{R}^n$
\[
  \eval{\pdv[]{}{x^j}}_p = \theta_{\varphi}^{-1}(e_j) \in T_pM
\]
Thus $\left\{ \eval{\pdv{}{x^1}}_p, \dots, \eval{\pdv{}{x^n}}_p\right\}$ is a basis for $T_pM$ called the coordinate basis induced by $(U, \varphi)$.
Write
\[
    [v]_p = v^1 \eval{\pdv{}{x^1}}_p + \dots + v^n \eval{\pdv{}{x^n}}_p
\]
where $v^j = \dv[]{\tilde{\gamma}^j}{t}$. Another important idea is that each tangent vector defines a map
\[
    D_v: C^{\infty} \rightarrow \mathds{R}
\]
satisfying leibnitz rule and linear on $C^{\infty}(M)$ by 
\[
    D_v(f) = (f \circ \gamma)'(0)
\]
which is called a derivation.

\end{document}
