%!TEX TS-program = xelatex
%!TEX encoding = UTF-8 Unicode

\documentclass[a4paper]{article}

\usepackage{xltxtra}
\usepackage{amsfonts}
\usepackage{polyglossia}
\usepackage{fancyhdr}
\usepackage{geometry}
\usepackage{dsfont}
\usepackage{amsmath}
\usepackage{amsthm}
\usepackage{amssymb}
\usepackage{physics}
\usepackage[shortlabels]{enumitem}
\usepackage{mathtools}

\geometry{a4paper,left=15mm,right=15mm,top=20mm,bottom=20mm}
\pagestyle{fancy}
\lhead{Devon Morris}
\chead{Differential Geometry - Homework 8}
\rhead{\today}
\cfoot{\thepage}

\setlength{\headheight}{23pt}
\setlength{\parindent}{0.0in}
\setlength{\parskip}{0.0in}

\newtheorem*{prop}{Proposition}
\newtheorem*{defn}{Definition}

\DeclarePairedDelimiterX{\inn}[2]{\langle}{\rangle}{#1, #2}

\begin{document}

\section*{Problem 1}%
Prove the following proposition
\begin{prop}
 The curvature $R$ of a Riemannian manifold is bilinear in $X,Y$, i.e.
 \[
   \begin{aligned}
     R(fX_1 + gX_2, Y_1) &= fR(X_1,Y_1) + gR(X_2, Y_1) \\
     R(X_1, fY_1 + gY_2) &= fR(X_1,Y_1) + gR(X_1,Y_2)
   \end{aligned}
 \]
 for $f,g \in C^{\infty}(M)$ and $X_1,X_2,Y_1,Y_2 \in \Gamma^{\infty}(TM)$.
\end{prop}

\begin{proof}
  Note by the definition of the Riemannian curvature and properties of the Levi-Civita connection we have
  \[
    \begin{aligned}
      R(fX_1 + gX_2, Y_1)Z &= \nabla_{Y_1} \nabla_{fX_1 + gX_2}Z - \nabla_{fX_1 + gX_2} \nabla_{Y_1} Z + \nabla_{[fX_1 + gX_2,Y]}Z \\
                           &= \nabla_{Y_1} \left( f\nabla_{X_1}Z + g\nabla_{X_2}Z\right) - f \nabla_{X_1}\nabla_{Y_1} Z  - g\nabla_{X_2}\nabla_{Y_1} Z + \nabla_{[fX_1,Y_1] + [gX_2, Y_1]} Z \\
                           &= f\nabla_{Y_1}\nabla_{X_1}Z + Y_1(f)\nabla_{X_1} Z + g\nabla_{Y_1}\nabla_{X_2}Z + Y_1(g)\nabla_{X_2}Z - f \nabla_{X_1}\nabla_{Y_1} Z  - g\nabla_{X_2}\nabla_{Y_1} Z \\
                           &\quad +  \nabla_{f[X_1, Y_1] + g_[X_2, Y_1] - Y_1(f)X_1 - Y_1(g)X_2}Z \\
                           &= f\nabla_{Y_1}\nabla_{X_1}Z + Y_1(f)\nabla_{X_1} Z + g\nabla_{Y_1}\nabla_{X_2}Z + Y_1(g)\nabla_{X_2}Z - f \nabla_{X_1}\nabla_{Y_1} Z  - g\nabla_{X_2}\nabla_{Y_1} Z \\
                           &\quad +  f\nabla_{[X_1,Y_1]}Z + g\nabla_{[X_2, Y_2]}Z - Y_1(f) \nabla_{X_1}Z - Y_1(g) \nabla_{X_2}Z \\
                           &= f \nabla_{Y_1}\nabla_{X_1} - f \nabla_{X_1} \nabla_{Y_1} Z + f \nabla_{[X_1,Y_1]} Z \\
                           &\quad + g \nabla_{Y_1}\nabla_{X_2} - g \nabla_{X_2} \nabla_{Y_1} Z + g \nabla_{[X_2,Y_1]} Z \\
                           &= f R(X_1, Y_1)Z + gR(X_2, Y_1)Z \\
                           &= \left( f R(X_1, Y_1) + gR(X_2, Y_2) \right)Z
    \end{aligned}
  \]
  At this point, we note that $R(X,Y)Z = -R(Y,X)Z$ and thus, we have
  \[
    R(X_1, fY_1 + gY_2) = -R(fY_1 + gY_2, X_1) = -fR(Y_1, X_1) - gR(Y_2, X_1) = fR(X_1, Y_1) + g R(X_1, Y_2)
  \]
\end{proof}

\section*{Problem 2}%
Let $S_r^2$ be the sphere of radius $r$ in $\mathds{R}^3$ centered at the origin. Equip $S_r^2$ with the metric induced by Euclidean space. Use spherical coordinates
\begin{enumerate}[(a)]
  \item  Compute the components of the Riemann curvature $R_{ijk}^s$ in these coordinates
  \item Use this to compute the sectional curvature $K(\sigma)$ at a point $p \in S_r^2$.
  \item Prove that $K(\sigma)$ is constant
  \item Compute the Ricci curvature and scalar curvature of $S_r^2$.
\end{enumerate}

\begin{proof}[Solution]
  \begin{enumerate}[(a)]
    \item In this case, it is easier to specify the inverse map $\varphi^{-1}: (0, 2\pi) \times (0, \pi) \rightarrow S_r^2$
      \[
        \varphi^{-1}(\theta, \phi) = (r \cos \theta \sin \phi, r \sin \theta \sin \phi, r \cos \phi)
      \]
  Now we can find the differential of the basis vectors $ \pdv{\theta}, \pdv{\phi}$ and compute the induce metric
  \[
    \begin{aligned}
      d\varphi^{-1} \left( \pdv{\theta} \right) &= -r \sin \theta \sin \phi \pdv{x} + r \cos \theta \sin \phi \pdv{y} \\
      d \varphi^{-1} \left( \pdv{\phi} \right) &= r \cos \theta \cos \phi \pdv{x} + r \sin \theta \cos \phi \pdv{y} - r \sin \phi \pdv{z}
    \end{aligned}
  \]
  Which gives us the components of our riemannian metric as
  \[
    \begin{aligned}
      g_{\theta \theta} &= r^2\sin^2 \phi \\
      g_{\phi \phi} &= r^2  \\
      g_{\theta \phi} &= 0
    \end{aligned}
  \]
  Now we can calculate the christoffel symbols. Due to symmetry we only need to calculate $\Gamma_{\theta \theta}^\theta, \Gamma_{\theta \phi}^\theta, \Gamma_{\phi \phi}^\theta, \Gamma_{\theta \theta}^\phi, \Gamma_{\phi \theta}^{\phi}, \Gamma_{\phi, \phi}^\phi$. First, let us calculate all possible derivatives
  \[
    \begin{aligned}
      \pdv{\theta} g_{\theta \theta} &= 0 \\
      \pdv{\theta} g_{\theta \phi} &= 0 \\
      \pdv{\theta} g_{\phi \phi} &= 0 \\
      \pdv{\phi} g_{\theta \theta} &= 2r^2 \sin \phi \cos \phi \\
      \pdv{\phi} g_{\theta \phi} &= 0 \\
      \pdv{\phi} g_{\phi \phi} &= 0
    \end{aligned}
  \]
  Furthermore we have
  \[
    \begin{aligned}
      g^{\theta \theta} &= \frac{1}{r^2 \sin^2 \phi} \\
      g^{\theta \phi} &= 0 \\
      g^{\phi \phi} &= \frac{1}{r^2}
    \end{aligned}
  \]
  Therefore, our Christoffel symbols are
  \[
    \begin{aligned}
      \Gamma_{\theta \theta}^\theta &= \frac{1}{2}\frac{1}{r^2 \sin^2 \phi} \left(0 +  0 - 0\right) = 0 \\
      \Gamma_{\theta \phi}^\theta &=  \frac{1}{2} \frac{1}{r^2 \sin^2 \phi} \left(2r^2 \sin \cos\phi + 0 - 0  \right) = \cot\phi \\
      \Gamma_{\phi \phi}^\theta &= \frac{1}{2} \frac{1}{r^2 \sin^2 \phi} \left(0 + 0 - 0  \right) = 0 \\
      \Gamma_{\theta \theta}^\phi &= \frac{1}{2} \frac{1}{r^2} \left(0 + 0 - 2r^2 \sin \phi \cos \phi \right) = -\sin \phi \cos \phi \\
      \Gamma_{\theta \phi}^\phi &= \frac{1}{2} \frac{1}{r^2} \left(0 + 0 - 0\right) = 0 \\
      \Gamma_{\phi \phi}^\phi &= \frac{1}{2} \frac{1}{r^2} \left( 0 + 0 - 0 \right) = 0
    \end{aligned}
  \]
  From this we see that
  \[
    \begin{aligned}
      \pdv{\phi} \Gamma_{\theta \theta}^\phi &= \sin^2 \phi - \cos^2 \phi \\
      \pdv{\phi} \Gamma_{\theta \phi}^\theta &= - \frac{1}{\sin^2 \phi} 
    \end{aligned}
  \]
  which are the only interesting derivatives. Now, we will calculate the components of the curvature tensor
  \[
    \begin{aligned}
      R_{\theta \theta \theta}^\theta &= \Gamma_{\theta \theta}^\theta \Gamma_{\theta \theta}^\theta + \Gamma_{\theta \theta}^\phi \Gamma_{\theta \phi}^{\theta} + \pdv{\theta} \Gamma_{\theta \theta}^\theta - \Gamma_{\theta \theta}^\theta \Gamma_{\theta \theta}^\theta - \Gamma_{\theta \theta}^\phi \Gamma_{\theta \phi}^\theta - \pdv{\theta} \Gamma_{\theta \theta}^\theta = 0\\
      R_{\theta \theta \theta}^\phi &= \Gamma_{\theta \theta}^\theta \Gamma_{\theta \theta}^\phi + \Gamma_{\theta \theta}^\phi \Gamma_{\theta \phi}^\phi + \pdv{\theta} \Gamma_{\theta \theta}^\phi - \Gamma_{\theta \theta}^\theta \Gamma_{\theta \theta}^\phi  - \Gamma_{\theta \theta}^\phi \Gamma_{\theta \phi}^\phi - \pdv{\theta} \Gamma_{\theta \theta}^\phi = 0 \\
      R_{\phi \theta \theta}^\theta &= \Gamma_{\phi \theta}^\theta \Gamma_{\theta \theta}^\theta + \Gamma_{\phi \theta}^\phi \Gamma_{\theta \phi}^\theta + \pdv{\theta} \Gamma_{\phi \theta}^\theta - \Gamma_{\theta \theta}^\theta \Gamma_{\phi \theta}^\theta  - \Gamma_{\theta \theta}^\phi \Gamma_{\phi \phi}^\theta - \pdv{\phi} \Gamma_{\theta \theta}^\theta = 0 \\
      R_{\phi \theta \theta}^\phi &= \Gamma_{\phi \theta}^\theta \Gamma_{\theta \theta}^\phi + \Gamma_{\phi \theta}^\phi \Gamma_{\theta \phi}^\phi + \pdv{\theta} \Gamma_{\phi \theta}^\phi - \Gamma_{\theta \theta}^\theta \Gamma_{\phi \theta}^\phi  - \Gamma_{\theta \theta}^\phi \Gamma_{\phi \phi}^\phi - \pdv{\phi} \Gamma_{\theta \theta}^\phi = -\sin^2 \phi \\
      R_{\theta \phi \theta}^\theta &= \Gamma_{\theta \theta}^\theta \Gamma_{\phi \theta}^\theta + \Gamma_{\theta \theta}^\phi \Gamma_{\phi \phi}^\theta + \pdv{\phi} \Gamma_{\theta \theta}^\theta - \Gamma_{\phi \theta}^\theta \Gamma_{\theta \theta}^\theta  - \Gamma_{\phi \theta}^\phi \Gamma_{\theta \phi}^\theta - \pdv{\theta} \Gamma_{\phi \theta}^\theta = 0 \\
      R_{\theta \phi \theta}^\phi &= \Gamma_{\theta \theta}^\theta \Gamma_{\phi \theta}^\phi + \Gamma_{\theta \theta}^\phi \Gamma_{\phi \phi}^\phi + \pdv{\phi} \Gamma_{\theta \theta}^\phi - \Gamma_{\phi \theta}^\theta \Gamma_{\theta \theta}^\phi  - \Gamma_{\phi \theta}^\phi \Gamma_{\theta \phi}^\phi - \pdv{\theta} \Gamma_{\phi \theta}^\phi = \sin^2 \phi \\
      R_{\phi \phi \theta}^\theta &= \Gamma_{\phi \theta}^\theta \Gamma_{\phi \theta}^\theta + \Gamma_{\phi \theta}^\phi \Gamma_{\phi \phi}^\theta + \pdv{\phi} \Gamma_{\phi \theta}^\theta - \Gamma_{\phi \theta}^\theta \Gamma_{\phi \theta}^\theta  - \Gamma_{\phi \theta}^\phi \Gamma_{\phi \phi}^\theta - \pdv{\phi} \Gamma_{\phi \theta}^\theta = 0 \\
      R_{\phi \phi \theta}^\phi &= \Gamma_{\phi \theta}^\theta \Gamma_{\phi \theta}^\phi + \Gamma_{\phi \theta}^\phi \Gamma_{\phi \phi}^\phi + \pdv{\phi} \Gamma_{\phi \theta}^\phi - \Gamma_{\phi \theta}^\theta \Gamma_{\phi \theta}^\phi  + \Gamma_{\phi \theta}^\phi \Gamma_{\phi \phi}^\phi - \pdv{\phi} \Gamma_{\phi \theta}^\phi = 0 \\
      R_{\theta \theta \phi}^\theta &= \Gamma_{\theta \phi}^\theta \Gamma_{\theta \theta}^\theta + \Gamma_{\theta \phi}^\phi \Gamma_{\theta \phi}^\theta + \pdv{\theta} \Gamma_{\theta \phi}^\theta - \Gamma_{\theta \phi}^\theta \Gamma_{\theta \theta}^\theta  - \Gamma_{\theta \phi}^\phi \Gamma_{\theta \phi}^\theta - \pdv{\theta} \Gamma_{\theta \phi}^\theta = 0 \\
      R_{\theta \theta \phi}^\phi &= \Gamma_{\theta \phi}^\theta \Gamma_{\theta \theta}^\phi + \Gamma_{\theta \phi}^\phi \Gamma_{\theta \phi}^\phi + \pdv{\theta} \Gamma_{\theta \phi}^\phi - \Gamma_{\theta \phi}^\theta \Gamma_{\theta \theta}^\phi  - \Gamma_{\theta \phi}^\phi \Gamma_{\theta \phi}^\phi - \pdv{\theta} \Gamma_{\theta \phi}^\phi = 0 \\
      R_{\phi \theta \phi}^\theta &= \Gamma_{\phi \phi}^\theta \Gamma_{\theta \theta}^\theta + \Gamma_{\phi \phi}^\phi \Gamma_{\theta \phi}^\theta + \pdv{\theta} \Gamma_{\phi \phi}^\theta - \Gamma_{\theta \phi}^\theta \Gamma_{\phi \theta}^\theta  - \Gamma_{\theta \phi}^\phi \Gamma_{\phi \phi}^\theta - \pdv{\phi} \Gamma_{\theta \phi}^\theta =  -1 \\
      R_{\phi \theta \phi}^\phi &= \Gamma_{\phi \phi}^\theta \Gamma_{\theta \theta}^\phi + \Gamma_{\phi \phi}^\phi \Gamma_{\theta \phi}^\phi + \pdv{\theta} \Gamma_{\phi \phi}^\phi - \Gamma_{\theta \phi}^\theta \Gamma_{\phi \theta}^\phi  - \Gamma_{\theta \phi}^\phi \Gamma_{\phi \phi}^\phi - \pdv{\phi} \Gamma_{\theta \phi}^\phi =  0 \\
      R_{\theta \phi \phi}^\theta &= \Gamma_{\theta \phi}^\theta \Gamma_{\phi \theta}^\theta + \Gamma_{\theta \phi}^\phi \Gamma_{\phi \phi}^\theta + \pdv{\phi} \Gamma_{\theta \phi}^\theta - \Gamma_{\phi \phi}^\theta \Gamma_{\theta \theta}^\theta  - \Gamma_{\phi \phi}^\phi \Gamma_{\theta \phi}^\theta - \pdv{\theta} \Gamma_{\phi \phi}^\theta =  1 \\
      R_{\theta \phi \phi}^\phi &= \Gamma_{\theta \phi}^\theta \Gamma_{\phi \theta}^\phi + \Gamma_{\theta \phi}^\phi \Gamma_{\phi \phi}^\phi + \pdv{\phi} \Gamma_{\theta \phi}^\phi - \Gamma_{\phi \phi}^\theta \Gamma_{\theta \theta}^\phi  - \Gamma_{\phi \phi}^\phi \Gamma_{\theta \phi}^\phi - \pdv{\theta} \Gamma_{\theta \phi}^\phi =  0 \\
      R_{\phi \phi \phi}^\theta &= \Gamma_{\phi \phi}^\theta \Gamma_{\phi \theta}^\theta + \Gamma_{\phi \phi}^\phi \Gamma_{\phi \phi}^\theta + \pdv{\phi} \Gamma_{\phi \phi}^\theta - \Gamma_{\phi \phi}^\theta \Gamma_{\phi \theta}^\theta  - \Gamma_{\phi \phi}^\phi \Gamma_{\phi \phi}^\theta - \pdv{\phi} \Gamma_{\phi \phi}^\theta =  0 \\
      R_{\phi \phi \phi}^\phi &= \Gamma_{\phi \phi}^\theta \Gamma_{\phi \theta}^\phi + \Gamma_{\phi \phi}^\phi \Gamma_{\phi \phi}^\phi + \pdv{\phi} \Gamma_{\phi \phi}^\phi - \Gamma_{\phi \phi}^\theta \Gamma_{\phi \theta}^\phi  - \Gamma_{\phi \phi}^\phi \Gamma_{\phi \phi}^\phi - \pdv{\phi} \Gamma_{\phi \phi}^\phi =  0 \\
    \end{aligned}
  \]
\item Let $p \in S_r^2$ we not that by definition $\pdv{\theta}, \pdv{\phi}$ are linearly independent vectors at this point. First let us calculate $R \left(\pdv{\theta}, \pdv{\phi} \right)\pdv{\theta}$ so we have
  \[
    \begin{aligned}
      R \left(\pdv{\theta}, \pdv{\phi} \right)\pdv{\theta} &= R_{ijk}^l \delta^{i}_\theta \delta^j_\phi \delta^k_\theta \pdv{x^l}\\
                                                           &= R_{\theta \phi \theta}^\theta \pdv{\theta} + R_{\theta \phi \theta}^\phi \pdv{\phi} \\
                                                           &= \sin^2 \phi \pdv{\phi}
    \end{aligned}
  \]
  and so we have
  \[
    \left( \pdv{\theta}, \pdv{\phi}, \pdv{\theta}, \pdv{\phi} \right) = \sin^2 \phi \inn*{\pdv{\phi}}{\pdv{\phi}} = r^2 \sin^2 \phi
  \]
  likewise
  \[
    \norm{\pdv{\theta}}^2 \norm{\pdv{\phi}}^2 - \inn*{\pdv{\theta}}{\pdv{\phi}} = r^4 \sin^2 \phi
  \]
  implying that
  \[
    K(\sigma) = \frac{1}{r^2}
  \]
  \item We note that in the calculation of $K(\sigma)$ in part b, we got something that wasn't dependent on the point $\varphi^{-1}(\sigma, \phi)$ so we have that $S_r^2$ has constant curvature.
  \end{enumerate}
\end{proof}

\section*{Problem 3}%
Recall the embeddings of the torus $T = \mathds{R}/2\pi \mathds{Z}$ in $\mathds{R}^3$ and $\mathds{R}^4$ given by the maps
\[
  \omega(\alpha, \beta) = ((\cos \beta + 4) \cos \alpha, (\cos \beta + 4) \sin \alpha, \sin \beta)
\]
and
\[
  \psi(\alpha, \beta) = (\cos \alpha, \sin \alpha, \cos \beta, \sin \beta)
\]
respectively. Let $T_3$ be the torus equipped with the metric induced from $\mathds{R}^3$ by the map $\omega$, and $T_4$ denote the torus equipped with the metric induced from $\mathds{R}^4$ by the map $\psi$. Compute the components of the curvature of $T_3$ and $T_4$

\begin{proof}[Solution]
  First we will start with $T_3$, we will start by computing the differential of this map
  \[
    d \omega \left( \pdv{\alpha} \right) = -(\cos \beta + 4) \sin \alpha \pdv{x^1} + (\cos \beta + 4) \cos \alpha \pdv{x^2}
  \]
  and 
  \[
    d \omega \left( \pdv{\beta} \right) = - \sin \beta \cos \alpha \pdv{x^1} - \sin \beta \sin \alpha \pdv{x^2} - \cos \beta \pdv{x^3}
  \]
  from this we can see that the components of the metric tensor are
  \[
    \begin{aligned}
      g_{\alpha \alpha} &= (\cos \beta + 4)^2 \\
      g_{\beta \beta} &=  1 \\
      g_{\alpha \beta} &= 0
    \end{aligned}
  \]
  This gives us the following christoffel symbols (of the first kind)
  \[
    \begin{aligned}
      \Gamma_{\alpha \alpha \alpha} &= \frac{1}{2}\left(0 + 0 + 0 \right) = 0 \\
      \Gamma_{\beta \alpha \alpha} &= \frac{1}{2} \left(0 + 0 + 2(\cos\beta + 4)\sin \beta \right) = (\cos \beta + 4) \sin \beta \\
      \Gamma_{\alpha \beta \alpha} &= \frac{1}{2} \left(0 - 2(\cos \beta + 4)\sin \beta - 0\right) = -(\cos \beta + 4) \sin \beta \\
      \Gamma_{\beta \beta \alpha} &= \frac{1}{2} \left(0 + 0 - 0 \right) = 0 \\
      \Gamma_{\alpha \beta \beta} &= \frac{1}{2} \left(0 + 0 - 0 \right) = 0 \\
      \Gamma_{\beta \beta \beta} &= \frac{1}{2} \left(0 + 0 - 0 \right) = 0 
    \end{aligned}
  \]
  raising the indices with the metric tensor gives us
  \[
    \begin{aligned}
      \Gamma^\alpha_{\alpha \alpha} &= g^{\alpha \alpha} \Gamma_{\alpha \alpha \alpha} + g^{\alpha \beta} \Gamma_{\beta \alpha \alpha} = 0 \\
    \Gamma^\beta_{\alpha \alpha} &= g^{\beta \beta} \Gamma_{\beta \alpha \alpha} = (\cos \beta + 4)\sin \beta \\
    \Gamma^\alpha_{\beta \alpha} &= g^{\alpha \alpha} \Gamma_{\alpha \beta \alpha} = - \frac{\sin \beta}{(\cos \beta + 4)} \\
    \Gamma^\beta_{\beta \alpha} &= g^{\beta \beta} \Gamma_{\beta \beta \alpha} = 0 \\
    \Gamma^\alpha_{\beta \beta} &= g^{\alpha \alpha} \Gamma_{\alpha \beta \beta} = 0 \\
    \Gamma^\beta_{\beta \beta} &= g^{\beta \beta} \Gamma_{\beta \beta \beta} = 0
    \end{aligned}
  \]
  Note that he only interesting second derivative is
  \[
    \pdv[2]{g_{\alpha \alpha}}{\beta} = 2(\sin^2 \beta - \cos^2 \beta - 4 \cos \beta)
  \]
  So we can write down the components of the lowered curvature tensor fairly easily using some identities
  \[
    \begin{aligned}
      R_{\alpha \alpha \alpha \alpha} &= 0 \\
      R_{\beta \alpha \alpha \alpha} &= 0 \\
      R_{\alpha \beta \alpha \alpha} &= 0 \\
      R_{\beta \beta \alpha \alpha} &= 0 \\
      R_{\alpha \alpha \beta \alpha} &= 0 \\
      R_{\beta \alpha \beta \alpha} &= \frac{1}{2} \left(0 + 0 - 0 + 2(-\sin^2 \beta + \cos^2 \beta + 4 \cos \beta) \right) + \left( \cos \beta + 4\right)^2\left(\frac{\sin^2 \beta}{(\cos \beta + 4)^2}\right)  \\
                                    &= \cos^2 \beta + 4 \cos \beta \\
      R_{\alpha \beta \beta \alpha} &= -\cos^2 \beta - 4 \cos \beta \\
      R_{\beta \beta \beta \alpha} &= 0 \\
      R_{\alpha \alpha \alpha \beta} &= 0 \\
      R_{\beta \alpha \alpha \beta} &= - \cos^2 \beta - 4 \cos \beta \\
      R_{\alpha \beta \alpha \beta} &= \cos^2 \beta + 4 \cos \beta \\
      R_{\beta \beta \alpha \beta} &= 0 \\
      R_{\alpha \alpha \beta \beta} &= 0 \\
      R_{\beta \alpha \beta \beta} &= 0 \\
      R_{\alpha \beta \beta \beta} &= 0 \\
      R_{\beta \beta \beta \beta} &= 0
    \end{aligned}
  \]
  Raising the indices is fairly straightforward, the nonzero components of the tensor are
  \[
    \begin{aligned}
      R^\beta_{\alpha \beta \alpha} &= \cos^2 \beta + 4 \cos \beta \\
      R^\alpha_{\beta \beta \alpha} &= \frac{-\cos^2 \beta - 4 \cos \beta}{(\cos \beta + 4)^2} \\
      R^\beta_{\alpha \alpha \beta} &= - \cos^2 \beta - 4 \cos \beta \\
      R^\alpha_{\beta \alpha \beta} &= \frac{\cos^2 \beta + 4 \cos \beta}{(\cos \beta + 4)^2}
    \end{aligned}
  \]
\end{proof}

\end{document}
