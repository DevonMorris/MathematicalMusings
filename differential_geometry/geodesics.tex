%!TEX TS-program = xelatex
%!TEX encoding = UTF-8 Unicode

\documentclass[a4paper]{article}

\usepackage{xltxtra}
\usepackage{amsfonts}
\usepackage{polyglossia}
\usepackage{fancyhdr}
\usepackage{geometry}
\usepackage{dsfont}
\usepackage{amsmath}
\usepackage{amsthm}
\usepackage{amssymb}
\usepackage{physics}
\usepackage{mathtools}

\geometry{a4paper,left=15mm,right=15mm,top=20mm,bottom=20mm}
\pagestyle{fancy}
\lhead{Devon Morris}
\chead{Differential Geometry}
\rhead{\today}
\cfoot{\thepage}

\setlength{\headheight}{23pt}
\setlength{\parindent}{0.0in}
\setlength{\parskip}{0.0in}

\newtheorem*{prop}{Proposition}
\newtheorem*{defn}{Definition}
\newtheorem*{thm}{Theorem}
\newtheorem*{cor}{Corollary}
\newtheorem*{lem}{Lemma}
\newtheorem*{rem}{Remark}

\DeclarePairedDelimiterX{\inn}[2]{\langle}{\rangle}{#1, #2}

\begin{document}
\section*{Geodesics}%
Let $(M,g)$ be a Riemannian manifold with $\nabla$ the Levi-Civita connection.

\begin{defn}
  A smooth curve $\gamma: [a,b] \rightarrow M$ is a geodesic if $\frac{D}{dt} \left( \dv{\gamma}{t} \right) =0$ at every $t \in I$. In other words $\dv{\gamma}{t}$ is parallel along $\gamma$.
\end{defn}
These are the analog of ``straight-lines'' on a manifold. We can think of this parallel requirement in terms of not allowing the velocity to change. Equivalently we could say these curves have 0 acceleration and the entire trajectory is determined by initial position and velocity. 

\begin{rem}
 By the compatibility of $\nabla$ with $g$ we have 
 \[
   \dv{t} g \left( \dv{\gamma}{t}, \dv{\gamma}{t} \right) = g \left( \frac{D}{dt} \left( \dv{\gamma}{t} \right),  \dv{\gamma}{t} \right) + g \left(\dv{\gamma}{t}, \frac{D}{dt} \left( \dv{\gamma}{t} \right)\right) = 0
 \]
 So the length of our velocity vector $\norm{\dv{\gamma}{t}}^2 = c^2$ is constant. Thus the length of a geodesic $\gamma: [a,b] \rightarrow M$
 \[
   \ell_a^b(\gamma) = \int_a^b \sqrt{ g \left( \dv{\gamma}{t}, \dv{\gamma}{t} \right)}\ dt = c(b-a)
 \]
 So the arclength of a geodesic is proportional to parameter length.
\end{rem}

Let $(U,\varphi) = (U, (x^j))$ be a coordinate chart on $M$ and $\gamma([a,b]) \cap U \neq \varnothing$, Then $\gamma(t) = (\gamma^1(t), \dots, \gamma^n(t))$ and 
\[
  \dv{\gamma}{t} = \dv{\gamma^j}{t} \pdv{x^j}
\]
hence 
\[
  0 = \frac{D}{dt} \left( \dv{\gamma}{t} \right) = \left(\dv[2]{t} \gamma^j + \dv{\gamma^j}{t}\dv{\gamma^i}{t} \Gamma_{ij}^k\right) \pdv{x^k}
\]
for all $k$. To solve $\dv{\gamma^k}{t} = y^k$ and solve the resulting first order system
\[
  \begin{aligned}
    \dv{\gamma^k}{t} &= y^k  \\
    \dv{y^k}{t} &= -\Gamma_{ij}^ky^jy^i
  \end{aligned}
\]
We want to solve for $(\gamma^1, \dots, \gamma^n, y^1, \dots, y^n) = \left( \gamma^1, \dots, \gamma^n, \dv{\gamma^1}{t}, \dots, \dv{\gamma^n}{t} \right)$. Recall that if $(U, \varphi) = (U, (x^j))$ are coordinates on $M$ and $\pi: TM \rightarrow M$ is the projection, then $\varphi$ induces a chart $\bar{\phi}$ on $\pi^{-1}(U)$ by 
\[
  (p,v) \mapsto (x^1(p), \dots, x^n(p), v^1, \dots, v^n)
\]
where $v \in T_pM$ is $v = v^j \pdv{x^j}$. Then for any $(p,v) = \left(\gamma^1(t_0), \dots, \gamma^n(t_0),\dv{\gamma^1}{t}(t_0), \dots, \dv{\gamma^n}{t}(t_0)\right)$, then the system above has a unique solution. In other words, for any $(p,v) \in TM$, there is a unique geodesic $\gamma$ passing through $p$ with $\dv{\gamma}{t} = v$.

\begin{lem}
  There exists a unique vector field $G$ on $TM$, $G \in \Gamma^{\infty}(TTM)$, whose integral curves are all of the form $t \mapsto (\gamma(t), \dv{\gamma}{t}(t))$, where $\gamma$ is a geodesic.
\end{lem}

Recall that $X \in \Gamma^{\infty}(N)$, for all $p \in N$, there exists $U$ containing $p$, and $\delta >0$, an a map $\theta: (-\delta, \delta) \times U \rightarrow N$ such that for all $q \in U$, the curve $\theta_{q}(t) = \theta(t,q)$ with 
\[
  \dv{\theta_q}{t} = X_{\theta_q(t)}
\]
with $\theta(0,q) = q$. $\theta$ is the local flow generated by $X$.

\begin{defn}
  The flow on $TM$ generated by $G$ is called the geodesic flow.
\end{defn}

\begin{lem}
  If the geodesic $\gamma(t,p,v)$ is defined for $t \in (-\delta, \delta)$ and $a > 0$ thenthe geodesic $\gamma(at, p,v)$ is defined on the interval $(-\frac{\delta}{a}, \frac{\delta}{a})$ and $\gamma(a t, p, v) = \gamma(t, p, av)$.
\end{lem}

\begin{proof}
  Let $h : (-\frac{\delta}{a}, \frac{\delta}{a}) \rightarrow M$ be defined by $h(t) = \gamma(at, p, v)$. Then $h(0) = \gamma(0, p,v) = p$, 
  \[
    \dv{h}{t}(0) =  a \dv{\gamma}{t} (at, p, v) = av
  \]
  We should check that it is a geodesic. So locally
  \[
    \frac{D}{dt} \left(\dv{h}{t}\right) = \nabla_{\dv{h}{t}} \dv{h}{t} =  \nabla_{a \dv{\gamma}{t}} \dv{\gamma}{t} = a^2 \nabla_{\dv{\gamma}{t}} \dv{\gamma}{t} = 0
  \]
  since $\gamma$ is a geodesic. Thus $h(t)$ is a geodesic passing through $p$. Thus by uniqueness we must have that
  \[
    h(t) = \gamma(at, p, v) = \gamma(t, p, av)
  \]
\end{proof}

\begin{defn}
  Let $M_0$ be the zero section of the tangent bundle $TM$, 
  \[
    M_0 = \coprod_{p \in M} \left\{ 0_p \right\}
  \]
  $0_p \in T_pM$.
\end{defn}

So using the zero section we can identify the zero section with the manifold itself. We can use this lemma.

\begin{cor}
  There is an open neighborhood $U$ of $M_0$ in $TM$ so that for every $(p,v) \in U$ the geodesic $t \mapsto \gamma(t,p,v)$ is defined for all $t \in (-2,2)$.
\end{cor}

\begin{proof}
  Around each $p \in M_0$, pick a neighborhood $U_p$ on which the geodesicsare defined for all $(-\delta_p, \delta_p)$. We can use the above lemma to extend $(-\delta_p, \delta_p)$ to $(-2, 2)$ at the expense of shrinking down the neighborhood $U_p$. The let $U = \cup_{p \in M} U_p$.
\end{proof}

\begin{defn}
  The map $\exp: U \rightarrow M$ defined by 
  \[
    \exp: (p,v) \mapsto \gamma(1, p, v) =  \gamma(|v|, p, v/|v|)
  \]
  is called the exponential map.
\end{defn}
We'll often restrict $\exp$ to a neighborhood of $0_p$ in some $T_pM$. Let $B_{\epsilon}(0)$ be the ball of radius $\epsilon$ about $0_p \in T_pM$. For $\epsilon$ sufficiently small define 
\[
  \begin{aligned}
    \exp_p: B_{\epsilon}(0) \rightarrow M \\
    (p,v) \mapsto \exp(p,v)
  \end{aligned}
\]

\begin{prop}
  Given $p \in M$, there exists $\epsilon > 0$, such that $\exp_p: B_{\epsilon}(0) \rightarrow M$ is a diffeomorphism from $B_{\epsilon} \subset T_pM$ onto an open subset of $M$. 
\end{prop}

Before we do the proof we state the following remark.

\begin{rem}
  For any finite dimensional vector space $V$ any $v \in V$, there is a canonical identification $ V \cong T_vV$. It is given by $w \in V$ is given by $v + tw$ and we think of $w$ as $[v + tw] = \dv{v}{t} (v + tw)$ 
\end{rem}

\begin{proof}
  We will show that the differential $d (\exp_p)_0$ is an isomorphism
  \[
    d(\exp_p)_0 : T_0T_pM \rightarrow T_pM
  \]
  let $\tilde{w} \in T_0T_pM$, we can identify $\tilde{w}$ with $[tw] = \dv{t}(tw)$ for some $w \in T_pM$. So we have
  \[
    d(\exp_p)_0(\tilde{w}) = \dv{t}\eval{(\exp_0(tw))}_{t=0} = \dv{t} \eval{\left( \gamma(1, p, tw)\right)}_{t=0} = \dv{t} \eval{\left( \gamma(t, p, w) \right)}_{t=0} = w
  \]
  So $d(\exp_p)_0$ is an isormorphism, and $\exp_p$ is a diffeomorphism on a neighborhood of $0 \in T_pM$ by the invertible mapping theorem.
\end{proof}

\begin{defn}
  Let $p \in M$ and let $B_{\epsilon}(0) \subset T_pM$ be an open ball such that $\exp_p$ restricts to a diffeomorphism on some open set $V$ with the closure of the $\overline{B_{\epsilon}(0)} \subset V$. Then, 
  \[
    B_{\epsilon}(p) = \exp_p (B_{\epsilon}(0))
  \]
  is called the normal ball (or geodesic ball) center at $p$ with radius $\epsilon$.
\end{defn}
 
So $\exp_p(v)$ is called a normal neighborhood of $p$. We call $\partial B_{\epsilon}(p)$ is calld normal (or geodesic) sphere. Geodesics in $B_{\epsilon}(p)$ which start at $p$ are radial geodesics.

\begin{defn}
  A piecewise differentiable or $C^{\infty}$ curve is a continuous map $c:[a,b] \rightarrow M$ such that there exists $a= t_0 < t_1 < \dots < t_n = b$ with $\eval{c}_{(t_i, t_{i+1})}$ is smooth for all $i$. The points $c(t_i)$ are called vertices of $c$ and the angle
  \[
    \lim_{t \rightarrow t_i^{-}} \dv{c}{t} \text{ and } \lim_{t \rightarrow t_i^+} \dv{c}{t}
  \]
  is called the vertex angle of $c(t_i)$. 
\end{defn}
This assumes that we have a connection (parallel transport). We want to prove the following.

\begin{lem}
  Let $p \in M$, $U$ a normal neighborhood of $p$ and $B \subset M$ a normal ball centered at $p$. Let $\gamma: [0,1] \rightarrow B$ be a geodesic with $\gamma(0) = p$ if $c:[0,1] \rightarrow M$ is some other piecewise differentiable curve with $c(0) = \gamma(0)$ and $c(1) = \gamma(1)$ then $\ell(c) \geq \ell(\gamma)$ and if equality holds then $\gamma([0,1]) = c([0,1])$.
\end{lem}

\begin{lem}[Gauss' Lemma]
  Let $p \in M$ and let $v \in T_pM$ such that $\exp_p(v)$ is defined. Let $w \in T_pM \cong T_v(T_pM)$. Then 
  \[
    \inn*{d(\exp_p)_v(v)}{d(\exp_p)_v(w)} = \inn*{v}{w}
  \]
  (on left hand we think $v,w \in T_v(T_pM)$ and on right hand $v,w \in T_pM$).
\end{lem}
We can think of this as saying that the radial geodesics are orthogonal to the normal spheres.

\begin{proof}[proof: of previous lemma]
  Suppose first that $c([0,1]) \subset B$. For all $t \in (0,1]$, we can write $c(t) = \exp_p(r(t)v(t))$, where $t \mapsto v(t) \in T_pM$ is a curve in $T_pM$ with $|v(t)| = 1$ for all $t$, and $r: (0,1] \rightarrow \mathds{R}$ is some piecewise $C^{\infty}$ function with $r(t) > 0$ for all $t$. We can assume that $c(t) \neq p$ for $t \in (0,1]$. We write $f(r,t) = \exp_p(r v(t))$, then $c(t) = f(r(t),t)$, then except at a finite number of points we have 
  \[
    \dv{c}{t} = \dv{f(r(t), t))}{t} = \pdv{f}{r}\dv{r}{t} + \pdv{f}{t}
  \]
  where we define
  \[
    \pdv{f}{r} = df \left( \pdv{r} \right) \text{ and }  \pdv{f}{t} = df \left( \pdv{t} \right)
  \]
  $\pdv{r}$ is a radial vector field on $T_pM$ and $\pdv{f}{r} = df \left( \pdv{r} \right) = d\left(\exp_p(r v(t))\right) \left( \pdv{r} \right)$ will be orthogonal to $\pdv{f}{t}$ which is
  \[
    \pdv{f}{t} = df \left( \dv{v}{t} \right)
  \]
  since $|v(t)| = 1$, by gauss's lemma. So
  \[
    \inn*{\pdv{f}{r}}{\pdv{f}{t}} = 0
  \]
  also $|\pdv{f}{r}| = 1$ since 
  \[
    \pdv{f}{r} = \pdv{r} \left( \exp_p \left( r(t) v(t) \right) \right) = \pdv{r} \left( \gamma(1, p, r(t)v(t) \right) = \pdv{r} \left( \gamma(r(t), p, v(t)) \right)  = v(t)
  \]
  and $|v(t)| = 1$. So we have that
  \[
    \left| \dv{c}{t} \right|^2 = \left| \pdv{f}{r} \right| \left| \dv{r}{t} \right| + \left| \dv{r}{t} \right| + \left| \pdv{f}{t} \right|^2 \geq \left| \dv{r}{t} \right|^2
  \]
  So
  \[
    \int_{\epsilon}^1 \left| \dv{c}{t} \right| dt \geq \int_{\epsilon}^1 \left| \dv{r}{t} \right| dt \geq \int_{\epsilon}^1 \dv{r}{t} dt = r(1) - r(\epsilon)
  \]
  So we have that
  \[
    r(1) = \ell(\gamma)
  \]
\end{proof}
  
\end{document}
