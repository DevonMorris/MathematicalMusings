%!TEX TS-program = xelatex
%!TEX encoding = UTF-8 Unicode

\documentclass[a4paper]{article}

\usepackage{xltxtra}
\usepackage{amsfonts}
\usepackage{polyglossia}
\usepackage{fancyhdr}
\usepackage{geometry}
\usepackage{dsfont}
\usepackage{amsmath}
\usepackage{amsthm}
\usepackage{amssymb}
\usepackage{physics}
\usepackage{mathtools}

\geometry{a4paper,left=15mm,right=15mm,top=20mm,bottom=20mm}
\pagestyle{fancy}
\lhead{Devon Morris}
\chead{Differential Geometry}
\rhead{\today}
\cfoot{\thepage}

\setlength{\headheight}{23pt}
\setlength{\parindent}{0.0in}
\setlength{\parskip}{0.0in}

\newtheorem*{prop}{Proposition}
\newtheorem*{defn}{Definition}
\newtheorem*{thm}{Theorem}
\newtheorem*{cor}{Corollary}
\newtheorem*{lem}{Lemma}
\newtheorem*{rem}{Remark}

\DeclarePairedDelimiterX{\inn}[2]{\langle}{\rangle}{#1, #2}

\begin{document}
\section*{Geodesics}%
Let $(M,g)$ be a Riemannian manifold with $\nabla$ the Levi-Civita connection.

\begin{defn}
  A smooth curve $\gamma: [a,b] \rightarrow M$ is a geodesic if $\frac{D}{dt} \left( \dv{\gamma}{t} \right) =0$ at every $t \in I$. In other words $\dv{\gamma}{t}$ is parallel along $\gamma$.
\end{defn}
These are the analog of ``straight-lines'' on a manifold. We can think of this parallel requirement in terms of not allowing the velocity to change. Equivalently we could say these curves have 0 acceleration and the entire trajectory is determined by initial position and velocity. 

\begin{rem}
 By the compatibility of $\nabla$ with $g$ we have 
 \[
   \dv{t} g \left( \dv{\gamma}{t}, \dv{\gamma}{t} \right) = g \left( \frac{D}{dt} \left( \dv{\gamma}{t} \right),  \dv{\gamma}{t} \right) + g \left(\dv{\gamma}{t}, \frac{D}{dt} \left( \dv{\gamma}{t} \right)\right) = 0
 \]
 So the length of our velocity vector $\norm{\dv{\gamma}{t}}^2 = c^2$ is constant. Thus the length of a geodesic $\gamma: [a,b] \rightarrow M$
 \[
   \ell_a^b(\gamma) = \int_a^b \sqrt{ g \left( \dv{\gamma}{t}, \dv{\gamma}{t} \right)}\ dt = c(b-a)
 \]
 So the arclength of a geodesic is proportional to parameter length.
\end{rem}

Let $(U,\varphi) = (U, (x^j))$ be a coordinate chart on $M$ and $\gamma([a,b]) \cap U \neq \varnothing$, Then $\gamma(t) = (\gamma^1(t), \dots, \gamma^n(t))$ and 
\[
  \dv{\gamma}{t} = \dv{\gamma^j}{t} \pdv{x^j}
\]
hence 
\[
  0 = \frac{D}{dt} \left( \dv{\gamma}{t} \right) = \left(\dv[2]{t} \gamma^j + \dv{\gamma^j}{t}\dv{\gamma^i}{t} \Gamma_{ij}^k\right) \pdv{x^k}
\]
for all $k$. To solve $\dv{\gamma^k}{t} = y^k$ and solve the resulting first order system
\[
  \begin{aligned}
    \dv{\gamma^k}{t} &= y^k  \\
    \dv{y^k}{t} &= -\Gamma_{ij}^ky^jy^i
  \end{aligned}
\]
We want to solve for $(\gamma^1, \dots, \gamma^n, y^1, \dots, y^n) = \left( \gamma^1, \dots, \gamma^n, \dv{\gamma^1}{t}, \dots, \dv{\gamma^n}{t} \right)$. Recall that if $(U, \varphi) = (U, (x^j))$ are coordinates on $M$ and $\pi: TM \rightarrow M$ is the projection, then $\varphi$ induces a chart $\bar{\phi}$ on $\pi^{-1}(U)$ by 
\[
  (p,v) \mapsto (x^1(p), \dots, x^n(p), v^1, \dots, v^n)
\]
where $v \in T_pM$ is $v = v^j \pdv{x^j}$. Then for any $(p,v) = \left(\gamma^1(t_0), \dots, \gamma^n(t_0),\dv{\gamma^1}{t}(t_0), \dots, \dv{\gamma^n}{t}(t_0)\right)$, then the system above has a unique solution. In other words, for any $(p,v) \in TM$, there is a unique geodesic $\gamma$ passing through $p$ with $\dv{\gamma}{t} = v$.

\begin{lem}
  There exists a unique vector field $G$ on $TM$, $G \in \Gamma^{\infty}(TTM)$, whose integral curves are all of the form $t \mapsto (\gamma(t), \dv{\gamma}{t}(t))$, where $\gamma$ is a geodesic.
\end{lem}

Recall that $X \in \Gamma^{\infty}(N)$, for all $p \in N$, there exists $U$ containing $p$, and $\delta >0$, an a map $\theta: (-\delta, \delta) \times U \rightarrow N$ such that for all $q \in U$, the curve $\theta_{q}(t) = \theta(t,q)$ with 
\[
  \dv{\theta_q}{t} = X_{\theta_q(t)}
\]
with $\theta(0,q) = q$. $\theta$ is the local flow generated by $X$.

\begin{defn}
  The flow on $TM$ generated by $G$ is called the geodesic flow.
\end{defn}
  
\end{document}
