%!TEX TS-program = xelatex
%!TEX encoding = UTF-8 Unicode

\documentclass[a4paper]{article}

\usepackage{xltxtra}
\usepackage{amsfonts}
\usepackage{polyglossia}
\usepackage{fancyhdr}
\usepackage{geometry}
\usepackage{dsfont}
\usepackage{amsmath}
\usepackage{amsthm}
\usepackage{amssymb}
\usepackage{physics}
\usepackage{mathtools}

\geometry{a4paper,left=15mm,right=15mm,top=20mm,bottom=20mm}
\pagestyle{fancy}
\lhead{Devon Morris}
\chead{Differential Geometry}
\rhead{\today}
\cfoot{\thepage}

\setlength{\headheight}{23pt}
\setlength{\parindent}{0.0in}
\setlength{\parskip}{0.0in}

\newtheorem*{prop}{Proposition}
\newtheorem*{defn}{Definition}
\newtheorem*{thm}{Theorem}
\newtheorem*{cor}{Corollary}
\newtheorem*{lem}{Lemma}
\newtheorem*{rem}{Remark}

\DeclarePairedDelimiterX{\inn}[2]{\langle}{\rangle}{#1, #2}

\begin{document}
\section*{Tensors}%

\subsection*{Covectors}%
Let $V$ be a finite dimensional real vector space. The dual space 
\[
  V^*  = \left\{ f: f:V \rightarrow \mathds{R} \text{ linear } \right\}
\]
\begin{defn}
  Let $\left\{ e_1, \dots, e_n \right\}$ is a basis for $V$. Let $\epsilon^j \in V^*$ by
  \[
    \epsilon^j(e_i) = \delta_i^j
  \]
  and we extendend linearly to $V$.
  \[
    \epsilon^j(x^ie_i) = x^j
  \]
\end{defn}

\begin{rem}
  Recall that
  \begin{enumerate}
    \item $\left\{ \epsilon^1,  \dots, \epsilon^n \right\}$ is a basis for $V^*$
    \item $V \cong V^*$ though not canonically.
    \item Specifying a basis on $V$, $\left\{e_1, \dots, e_n  \right\}$ determines an isomorphism by sending $e_j \mapsto \epsilon^j$ and extending linearly and specifying an inner product $\inn*{\cdot}{\cdot}$ on $V$ determines an isomorphism by
      \[
        x \mapsto \inn*{x_j}{\cdot}
      \]
    \item $V \cong V^{**}$ canonically via the isomorphism
      \[
        x \mapsto (f \mapsto f(x))
      \]
  \end{enumerate}
\end{rem}

\subsection*{Cotangent Space}%
\begin{defn}
  Let $M$ be a smooth manifold, $p \in M$. 
  \[
    T_p^*M = (T_pM)^*
  \]
  is called the cotangent space to $M$ at $p$. 
\end{defn}
The elements of $T_p^*M$ are linear functionals $T_pM \rightarrow \mathds{R}$ and are called covectors. 
\begin{defn}
 The disjoint union of our cotangent spaces
 \[
   T^*M = \coprod_{p \in M} T_p^*M
 \]
 is called the cotangent bundle of $M$.
\end{defn}
Like the tangent bundle we can define a projection map $\pi: T^*M \rightarrow M$. Let $(U, (x^j)$ be local coordinates on the manifold $M$. Then $\left\{ \pdv{x^1}, \dots, \pdv{x^n} \right\}$ is a basis for $T_pM$, $p \in U$. Then we create the dual basis $ \left\{ dx^1, \dots, dx^n \right\}$ for $T_p^*M$ for $p \in U$.
\[
  dx^i\left(\pdv{x^j}\right)= \delta_j^i
\]
This allows us to define a local product structure on $T^*M$, giving it a topology and smooth structure, $\dim(T^*M) = 2n$ and $\pi: T^*M \rightarrow M$ smooth. 
\begin{defn}
  A covector field is $\omega$ is a map $\omega: M \rightarrow T^*M$ such that $\pi \circ \omega = \text{id}_M$.
\end{defn}
\begin{rem}
  $\omega$ is smooth if and only if $\omega(X) \in C^\infty(M)$ for all $X \in \Gamma^{\infty}(TM)$. We denote this by $\Gamma^{\infty}(T^*M)$ is the space of all smooth covector fields on $M$.
\end{rem}

Let $f: M \rightarrow \mathds{R}$ be smooth. Then
\[
  df_p: T_pM \rightarrow T_{f(p)}\mathds{R} \cong  \mathds{R}
\] 
So we think of $df_p$ as a linear functional at each $p \in M$. In coordinates
\[
  df_p = \left[ \pdv{f}{x^1}, \dots, \pdv{f}{x^n} \right]
\]
or more explicitly
\[
  df_p = \pdv{f}{x^1} dx^1 + \dots + \pdv{f}{x^n} dx^n
\]
is smooth in $p \in M$. Thus $df \in \Gamma^{\infty}(T^*M)$. So if $X \in \Gamma^{\infty}(TM)$ then
\[
  df(X) = df \left( X^i \pdv{x^i} \right) = \pdv{f}{x^j} dx^j \left(X^i \pdv{x^i}\right) = X^i \pdv{f}{x^j} \delta_i^j = X^i \pdv{f}{x^i}
\]
Recall that $(T^*M)^* \cong TM$, so for $X = X^i \pdv{x^i} \in \Gamma^{\infty}(TM)$. The isomorphism $V^{**} \cong V$ is made explicit by
\[
  v \mapsto (f \mapsto f(v))
\]
so making this identification we get
\[
  X(df) = df(X) = X^i \pdv{f}{x^i} = X(f)
\]
So if we analyze
\[
  dx^i(X) = X(x^i) = X^j \pdv{x^i}{x^j} = X^i
\]

\subsection*{Tensors}%
Let $V$ be a finite dimensional real vector space

\begin{defn}
  A tensor of type $(k,l)$ is a multilinear map of the form 
  \[
    F: V^* \times \dots \times V^* \times V \times \dots \times V \rightarrow \mathds{R}
  \]
  where there are $l$ copies of $V^*$ and $k$ copies of $V$.
\end{defn}
If $l = 0$ we call it a $k$ - covariant tensor, and if $k$ - covariant tensor. The rank of $F$ is $k + l$. The space of all tesnors of type $(k,l)$ is $T^k_l(V)$ if $k=0$ or $l=0$ we sometimes write $T^k(V) = T^k_0(V)$ or $T_l(V) = T_l^0(V)$.

\begin{defn}
  If $F \in T_k^l(v)$ and $G \in T_s^r(V)$, then the tensor product of $F$ and $G$ is $F \otimes G \in T_{l + s}^{k+r}(V)$.
  \[
    F \otimes G(\omega^1, \dots, \omega^{l+s}, v_1, \dots, v_{k+r}) = F(\omega^1, \dots \omega^l, v_1, \dots, v_k)G(\omega^{l+1}, \dots, \omega^{l+s}, v_{k+1}, \dots, v_{k+r})
  \]
\end{defn}

If $\left\{ e_1, \dots, e_n \right\}$ is a basis for $V$ and $\left\{ \epsilon^1, \dots, \epsilon^n \right\}$ for $V^*$, then
\[
  \left\{ e_{j_1} \otimes \dots \otimes e_{j_l} \otimes \epsilon^{i_1} \otimes \dots \otimes \epsilon^{i_k} | 1 \leq i_1, \dots, i_k, j_1, \dots, j_l \leq n \right\}
\]
Then we have that any $F \in T_l^k(V)$ can be written as
\[
  F = F^{j_1, \dots, j_l}_{i_1, \dots, i_k} e_{j_1} \otimes \dots \otimes e_{j_l} \otimes \epsilon^{i_1} \otimes \dots \otimes \epsilon^{i_k}
\]
Let $V = T_pM$, $(U, (x^j)$ be local coordinates at each $p \in U$ we get the basis
\[
  \left\{ \pdv{x^{j_1}} \otimes \dots \otimes \pdv{x^{j_l}} \otimes dx^{i_1} \otimes \dots \otimes dx^{i_k} | 1 \leq i_1, \dots, i_k, j_1, \dots, j_l \leq n \right\}
\]
giving a topology and smooth structure on
\[
  T_l^kM = \coprod_{p \in M} T_l^k (T_pM)
\]
as before and $\pi:T_k^lM \rightarrow M$ be the projection.

\begin{defn}
  A tensor field of type $(k,l)$ is a map $T: M \rightarrow T_l^kM$ such that $\pi circ T = \text{id}_M$. Often we just call $T$ a tensor. 
\end{defn}

We check if this vector field is smooth if the function $f = T(\omega^1, \dots, \omega^l, X_1, \dots, X_k)$ for smooth covector and vector fields. We say $T \in \Gamma^{\infty}(T_l^kM)$ is the smooth tensor fields of type $(k,l)$. So any $T \in \Gamma^{\infty}(T_l^kM)$ can be written locally as
\[
  T = T_{i_1 \dots i_k}^{j_1 \dots j_l} \pdv{x^{j_1} } \otimes \dots \otimes \pdv{x^{j_l}} \otimes dx^{i_1} \otimes \dots \otimes dx^{i_k} 
\]
for some $T_{i_1 dots i_k}^{j_1 \dots j_l} \in C^{\infty}(U)$. So we have
\[
  T(\omega^1, \dots, \omega^l, X_{1}, \dots, X_k) = T_{i_1 \dots i_k}^{j_1 \dots j_l} X_1^{i_1} \dots X_k^{i_k} \omega_{j_1}^1 \dots \omega_{j_l}^l
\]
So we talk about $T_k^l(M) = T_l^k(TM)$ as pointwise things. So $T_0^0 M \cong \mathds{R}$ and $\Gamma^{\infty}(T_0^0M) = C^{\infty}(M)$. Another example is the curvature $R$ of a connection $\nabla$, $R \in \Gamma^{\infty}(T_1^3M)$
\[
  R(X,Y,Z,\omega) = \omega(R(X,Y)Z)
\]

\subsection*{Covariant Derivative of Tensor Fields}%
Given a connection $\nabla: \Gamma^{\infty}(TM) \times \Gamma^{\infty} \rightarrow \Gamma^{\infty}(TM)$, we can extend $\nabla$ to a map
\[
  \nabla: \Gamma^{\infty} \times \Gamma^\infty(T_l^k M) \rightarrow \Gamma^\infty (T_l^k M)
\]
by the following rules
\begin{enumerate}
  \item $\nabla_Y f = df(Y) = Y(f)$ for all $Y \in \Gamma^\infty (TM)$, $f \in C^\infty(M)$
  \item 
    \[
      \begin{aligned}
        \nabla_Y(T(\omega^1, \dots, \omega^l, X_1, \dots, X_k)) =& \nabla_Y T(\omega, \dots, \omega^l, X_1, \dots, X_k) \\
      &+ T(\nabla_Y \omega^1, \omega^2, \dots, \omega^l, X_1, \dots, X_k) \\ 
      &+ \cdots \\
      &+ T(\omega^1, \dots, \omega^l, X_1, \dots, \nabla_Y X_k)
      \end{aligned}
  \]
  for all $T \in \Gamma^\infty(T_l^k M)$
\end{enumerate}
So for the special case when we are looking at covector fields we have
\[
  \nabla_Y \omega(X) = Y(\omega(X)) - \omega(\nabla_Y X)
\]
for all $X \in \Gamma^\infty(TM)$. We can also think of $Y$ in $\nabla_Y T$ as being an extra argument to be filled making $\nabla T$ a $(k+1, l)$ tensor field which is defined by
\[
  \nabla T(\omega^1, \dots, \omega^p, X_1, \dots, X_k, Y) = \nabla_Y T(\omega^1, \dots, \omega^l, X_1, \dots, X_k)
\]
\subsection*{Differential forms}%
\begin{defn}
  A tensor $\omega \in \Gamma^\infty(T^kM)$ is alternating if 
\[
  \omega(X_1, \dots X_i, \dots, X_j, \dots, X_k) = -\omega(X_1, \dots, X_j, \dots, X_i, \dots, X_k) 
\]
\end{defn}
We note that $\omega$ alternating is equivalent to saying $\omega$ vanishes when two indices are repeated. Let $\Omega^kM$ denote the space of alternating $(k,0)$ smooth tensor fields. We call elements of $\Omega^kM$ differential $k$ forms.

Let $\Sigma_k$ be the symmetric group on $\left\{ 1, \dots, k \right\}$, so if $ \omega \in \Gamma^\infty T^kM$, define $^\sigma \omega \in \Gamma^{\infty} (T^kM)$ by
\[
  ^\sigma \omega(X_1, \dots, X_k) = \omega (X_{\sigma(1)}, \dots, X_{\sigma(k)})
\]
this is the same as setting
\[
  ^\sigma \left( dx^{i_1} \otimes \cdots \otimes dx^{i_k} \right) = dx^{i_{\sigma(1)}} \otimes \dots \otimes dx^{i_{\sigma(k)}}
\]
then extending linearly to all of $\Gamma^\infty (T^kM)$

\begin{defn}
  The alternating map 
  \[
    \begin{aligned}
      \text{Alt}_k&: T^k M \rightarrow \Omega^k M \\
                  &\omega \mapsto \frac{1}{k!} \sum_{\sigma \in \Sigma_k} \text{sgn}(\sigma)\ ^\sigma\omega
    \end{aligned}
  \]
\end{defn}

\begin{prop}
  $\text{Alt}_k$ is surjective
\end{prop}
\begin{proof}
  If $\omega \in \Omega^k M$ then $\text{Alt}_k (\omega) = \omega$.
\end{proof}

\begin{defn}[Wedge Product]
  If $\omega \in \Omega^kM$, $\rho \in \Omega^lM$, then the wedge product of $\omega, \rho$ is defined by
  \[
    \omega \wedge \rho = \frac{(k+l)!}{k!l!} \text{Alt}(\omega \otimes \rho) \in \Omega^{k+l}M
  \]
\end{defn}

\end{document}
