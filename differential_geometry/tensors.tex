%!TEX TS-program = xelatex
%!TEX encoding = UTF-8 Unicode

\documentclass[a4paper]{article}

\usepackage{xltxtra}
\usepackage{amsfonts}
\usepackage{polyglossia}
\usepackage{fancyhdr}
\usepackage{geometry}
\usepackage{dsfont}
\usepackage{amsmath}
\usepackage{amsthm}
\usepackage{amssymb}
\usepackage{physics}
\usepackage{mathtools}

\geometry{a4paper,left=15mm,right=15mm,top=20mm,bottom=20mm}
\pagestyle{fancy}
\lhead{Devon Morris}
\chead{Differential Geometry}
\rhead{\today}
\cfoot{\thepage}

\setlength{\headheight}{23pt}
\setlength{\parindent}{0.0in}
\setlength{\parskip}{0.0in}

\newtheorem*{prop}{Proposition}
\newtheorem*{defn}{Definition}
\newtheorem*{thm}{Theorem}
\newtheorem*{cor}{Corollary}
\newtheorem*{lem}{Lemma}
\newtheorem*{rem}{Remark}

\DeclarePairedDelimiterX{\inn}[2]{\langle}{\rangle}{#1, #2}

\begin{document}
\section*{Tensors}%

\subsection*{Covectors}%
Let $V$ be a finite dimensional real vector space. The dual space 
\[
  V^*  = \left\{ f: f:V \rightarrow \mathds{R} \text{ linear } \right\}
\]
\begin{defn}
  Let $\left\{ e_1, \dots, e_n \right\}$ is a basis for $V$. Let $\epsilon^j \in V^*$ by
  \[
    \epsilon^j(e_i) = \delta_i^j
  \]
  and we extendend linearly to $V$.
  \[
    \epsilon^j(x^ie_i) = x^j
  \]
\end{defn}

\begin{rem}
  Recall that
  \begin{enumerate}
    \item $\left\{ \epsilon^1,  \dots, \epsilon^n \right\}$ is a basis for $V^*$
    \item $V \cong V^*$ though not canonically.
    \item Specifying a basis on $V$, $\left\{e_1, \dots, e_n  \right\}$ determines an isomorphism by sending $e_j \mapsto \epsilon^j$ and extending linearly and specifying an inner product $\inn*{\cdot}{\cdot}$ on $V$ determines an isomorphism by
      \[
        x \mapsto \inn*{x_j}{\cdot}
      \]
    \item $V \cong V^{**}$ canonically via the isomorphism
      \[
        x \mapsto (f \mapsto f(x))
      \]
  \end{enumerate}
\end{rem}

\subsection*{Cotangent Space}%
\begin{defn}
  Let $M$ be a smooth manifold, $p \in M$. 
  \[
    T_p^*M = (T_pM)^*
  \]
  is called the cotangent space to $M$ at $p$. 
\end{defn}
The elements of $T_p^*M$ are linear functionals $T_pM \rightarrow \mathds{R}$ and are called covectors. 
\begin{defn}
 The disjoint union of our cotangent spaces
 \[
   T^*M = \coprod_{p \in M} T_p^*M
 \]
 is called the cotangent bundle of $M$.
\end{defn}
Like the tangent bundle we can define a projection map $\pi: T^*M \rightarrow M$. Let $(U, (x^j)$ be local coordinates on the manifold $M$. Then $\left\{ \pdv{x^1}, \dots, \pdv{x^n} \right\}$ is a basis for $T_pM$, $p \in U$. Then we create the dual basis $ \left\{ dx^1, \dots, dx^n \right\}$ for $T_p^*M$ for $p \in U$.
\[
  dx^i\left(\pdv{x^j}\right)= \delta_j^i
\]
This allows us to define a local product structure on $T^*M$, giving it a topology and smooth structure, $\dim(T^*M) = 2n$ and $\pi: T^*M \rightarrow M$ smooth. 
\begin{defn}
  A covector field is $\omega$ is a map $\omega: M \rightarrow T^*M$ such that $\pi \circ \omega = \text{id}_M$.
\end{defn}
\begin{rem}
  $\omega$ is smooth if and only if $\omega(X) \in C^\infty(M)$ for all $X \in \Gamma^{\infty}(TM)$. We denote this by $\Gamma^{\infty}(T^*M)$ is the space of all smooth covector fields on $M$.
\end{rem}

Let $f: M \rightarrow \mathds{R}$ be smooth. Then
\[
  df_p: T_pM \rightarrow T_{f(p)}\mathds{R} \cong  \mathds{R}
\] 
So we think of $df_p$ as a linear functional at each $p \in M$. In coordinates
\[
  df_p = \left[ \pdv{f}{x^1}, \dots, \pdv{f}{x^n} \right]
\]
or more explicitly
\[
  df_p = \pdv{f}{x^1} dx^1 + \dots + \pdv{f}{x^n} dx^n
\]
is smooth in $p \in M$. Thus $df \in \Gamma^{\infty}(T^*M)$. So if $X \in \Gamma^{\infty}(TM)$ then
\[
  df(X) = df \left( X^i \pdv{x^i} \right) = \pdv{f}{x^j} dx^j \left(X^i \pdv{x^i}\right) = X^i \pdv{f}{x^j} \delta_i^j = X^i \pdv{f}{x^i}
\]
Recall that $(T^*M)^* \cong TM$, so for $X = X^i \pdv{x^i} \in \Gamma^{\infty}(TM)$. The isomorphism $V^{**} \cong V$ is made explicit by
\[
  v \mapsto (f \mapsto f(v))
\]
so making this identification we get
\[
  X(df) = df(X) = X^i \pdv{f}{x^i} = X(f)
\]
So if we analyze
\[
  dx^i(X) = X(x^i) = X^j \pdv{x^i}{x^j} = X^i
\]

\subsection*{Tensors}%
Let $V$ be a finite dimensional real vector space

\begin{defn}
  A tensor of type $(k,l)$ is a multilinear map of the form 
  \[
    F: V^* \times \dots \times V^* \times V \times \dots \times V \rightarrow \mathds{R}
  \]
  where there are $l$ copies of $V^*$ and $k$ copies of $V$.
\end{defn}
If $l = 0$ we call it a $k$ - covariant tensor, and if $k$ - covariant tensor. The rank of $F$ is $k + l$. The space of all tesnors of type $(k,l)$ is $T^k_l(V)$ if $k=0$ or $l=0$ we sometimes write $T^k(V) = T^k_0(V)$ or $T_l(V) = T_l^0(V)$.

\begin{defn}
  If $F \in T_k^l(v)$ and $G \in T_s^r(V)$, then the tensor product of $F$ and $G$ is $F \otimes G \in T_{l + s}^{k+r}(V)$.
  \[
    F \otimes G(\omega^1, \dots, \omega^{l+s}, v_1, \dots, v_{k+r}) = F(\omega^1, \dots \omega^l, v_1, \dots, v_k)G(\omega^{l+1}, \dots, \omega^{l+s}, v_{k+1}, \dots, v_{k+r})
  \]
\end{defn}

If $\left\{ e_1, \dots, e_n \right\}$ is a basis for $V$ and $\left\{ \epsilon^1, \dots, \epsilon^n \right\}$ for $V^*$, then
\[
  \left\{ e_{j_1} \otimes \dots \otimes e_{j_l} \otimes \epsilon^{i_1} \otimes \dots \otimes \epsilon^{i_k} | 1 \leq i_1, \dots, i_k, j_1, \dots, j_l \leq n \right\}
\]
Then we have that any $F \in T_l^k(V)$ can be written as
\[
  F = F^{j_1, \dots, j_l}_{i_1, \dots, i_k} e_{j_1} \otimes \dots \otimes e_{j_l} \otimes \epsilon^{i_1} \otimes \dots \otimes \epsilon^{i_k}
\]
Let $V = T_pM$, $(U, (x^j)$ be local coordinates at each $p \in U$ we get the basis
\[
  \left\{ \pdv{x^{j_1}} \otimes \dots \otimes \pdv{x^{j_l}} \otimes dx^{i_1} \otimes \dots \otimes dx^{i_k} | 1 \leq i_1, \dots, i_k, j_1, \dots, j_l \leq n \right\}
\]
giving a topology and smooth structure on
\[
  T_l^kM = \coprod_{p \in M} T_l^k (T_pM)
\]
as before and $\pi:T_k^lM \rightarrow M$ be the projection.

\begin{defn}
  A tensor field of type $(k,l)$ is a map $T: M \rightarrow T_l^kM$ such that $\pi circ T = \text{id}_M$. Often we just call $T$ a tensor. 
\end{defn}

We check if this vector field is smooth if the function $f = T(\omega^1, \dots, \omega^l, X_1, \dots, X_k)$ for smooth covector and vector fields. We say $T \in \Gamma^{\infty}(T_l^kM)$ is the smooth tensor fields of type $(k,l)$. So any $T \in \Gamma^{\infty}(T_l^kM)$ can be written locally as
\[
  T = T_{i_1 \dots i_k}^{j_1 \dots j_l} \pdv{x^{j_1} } \otimes \dots \otimes \pdv{x^{j_l}} \otimes dx^{i_1} \otimes \dots \otimes dx^{i_k} 
\]
for some $T_{i_1 \dots i_k}^{j_1 \dots j_l} \in C^{\infty}(U)$. So we have
\[
  T(\omega^1, \dots, \omega^l, X_{1}, \dots, X_k) = T_{i_1 \dots i_k}^{j_1 \dots j_l} X_1^{i_1} \dots X_k^{i_k} \omega_{j_1}^1 \dots \omega_{j_l}^l
\]
So we talk about $T_k^l(M) = T_l^k(TM)$ as pointwise things. So $T_0^0 M \cong \mathds{R}$ and $\Gamma^{\infty}(T_0^0M) = C^{\infty}(M)$. Another example is the curvature $R$ of a connection $\nabla$, $R \in \Gamma^{\infty}(T_1^3M)$
\[
  R(X,Y,Z,\omega) = \omega(R(X,Y)Z)
\]

\subsection*{Covariant Derivative of Tensor Fields}%
Given a connection $\nabla: \Gamma^{\infty}(TM) \times \Gamma^{\infty} \rightarrow \Gamma^{\infty}(TM)$, we can extend $\nabla$ to a map
\[
  \nabla: \Gamma^{\infty} \times \Gamma^\infty(T_l^k M) \rightarrow \Gamma^\infty (T_l^k M)
\]
by the following rules
\begin{enumerate}
  \item $\nabla_Y f = df(Y) = Y(f)$ for all $Y \in \Gamma^\infty (TM)$, $f \in C^\infty(M)$
  \item 
    \[
      \begin{aligned}
        \nabla_Y(T(\omega^1, \dots, \omega^l, X_1, \dots, X_k)) =&  Y(T(\omega, \dots, \omega^l, X_1, \dots, X_k)) \\
      &- T(\nabla_Y \omega^1, \omega^2, \dots, \omega^l, X_1, \dots, X_k) \\ 
      &- \cdots \\
      &- T(\omega^1, \dots, \omega^l, X_1, \dots, \nabla_Y X_k)
      \end{aligned}
  \]
  for all $T \in \Gamma^\infty(T_l^k M)$
\end{enumerate}
So for the special case when we are looking at covector fields we have
\[
  \nabla_Y \omega(X) = Y(\omega(X)) - \omega(\nabla_Y X)
\]
for all $X \in \Gamma^\infty(TM)$. We can also think of $Y$ in $\nabla_Y T$ as being an extra argument to be filled making $\nabla T$ a $(k+1, l)$ tensor field which is defined by
\[
  \nabla T(\omega^1, \dots, \omega^p, X_1, \dots, X_k, Y) = \nabla_Y T(\omega^1, \dots, \omega^l, X_1, \dots, X_k)
\]
\subsection*{Differential forms}%
\begin{defn}
  A tensor $\omega \in \Gamma^\infty(T^kM)$ is alternating if 
\[
  \omega(X_1, \dots X_i, \dots, X_j, \dots, X_k) = -\omega(X_1, \dots, X_j, \dots, X_i, \dots, X_k) 
\]
\end{defn}
We note that $\omega$ alternating is equivalent to saying $\omega$ vanishes when two indices are repeated. Let $\Omega^kM$ denote the space of alternating $(k,0)$ smooth tensor fields. We call elements of $\Omega^kM$ differential $k$ forms.

Let $\Sigma_k$ be the symmetric group on $\left\{ 1, \dots, k \right\}$, so if $ \omega \in \Gamma^\infty T^kM$, define $^\sigma \omega \in \Gamma^{\infty} (T^kM)$ by
\[
  ^\sigma \omega(X_1, \dots, X_k) = \omega (X_{\sigma(1)}, \dots, X_{\sigma(k)})
\]
this is the same as setting
\[
  ^\sigma \left( dx^{i_1} \otimes \cdots \otimes dx^{i_k} \right) = dx^{i_{\sigma(1)}} \otimes \dots \otimes dx^{i_{\sigma(k)}}
\]
then extending linearly to all of $\Gamma^\infty (T^kM)$

\begin{defn}
  The alternating map 
  \[
    \begin{aligned}
      \text{Alt}_k&: T^k M \rightarrow \Omega^k M \\
                  &\omega \mapsto \frac{1}{k!} \sum_{\sigma \in \Sigma_k} \text{sgn}(\sigma)\ ^\sigma\omega
    \end{aligned}
  \]
\end{defn}

\begin{prop}
  $\text{Alt}_k$ is surjective
\end{prop}
\begin{proof}
  If $\omega \in \Omega^k M$ then $\text{Alt}_k (\omega) = \omega$.
\end{proof}

\begin{defn}[Wedge Product]
  If $\omega \in \Omega^kM$, $\rho \in \Omega^lM$, then the wedge product of $\omega, \rho$ is defined by
  \[
    \omega \wedge \rho = \frac{(k+l)!}{k!l!} \text{Alt}(\omega \otimes \rho) \in \Omega^{k+l}M
  \]
\end{defn}
The simplest example is 
\[
  dx \wedge dy = 2 \text{Alt}(dx \otimes dy) = dx \otimes dy - dy \otimes dx
\]
another example is
\[
  \begin{aligned}
    (dx \wedge dy) \wedge dz &= \frac{3!}{2!} \text{Alt}((dx \otimes dy - dy \otimes dx) \otimes dz) \\
                             &= dx \otimes dy \otimes dz - dy \otimes dx \otimes dz + \dots (fill this in)
  \end{aligned}
\]

\begin{prop}
  The wedge product has the following properties
  \begin{enumerate}
    \item Bilinearity $(a \omega + b \rho) \wedge \eta = a(\omega \wedge \eta) + b (\rho \wedge \eta)$
    \item Associative $(\omega \wedge \rho) \wedge \eta = \omega \wedge (\rho \wedge \eta)$
    \item Anticommutative $\omega \wedge \rho = (-1)^{kl} \rho \wedge \omega$
    \item For $\omega^1, \dots, \omega^k \in \Omega^1M = \Gamma^\infty (T^*M)$ and $v_1,\dots,v_k \in \Gamma^\infty(TM)$ then
      \[
        \omega^1 \wedge \cdots \wedge \omega^k(v_1, \dots, v_k) = \det(w^j(v_i))
      \]
  \end{enumerate}
\end{prop}
We also note that $\eta \wedge \eta = 0$ for $\eta \in \Omega^{2k+1}M$, but even forms don't necessarily have to be zero.

If $M$ has local coordinates $(U,(x^i))$, then locally we can write $\Omega^kU$ as a $C^\infty(U)$ module generated by 
\[
  \left\{ dx^{i_1} \wedge \cdots \wedge dx^{i_k} | 1 \leq i_1 < \dots < i_k \leq n\right\}
\]
rank $\Omega^k U = \binom{n}{k}$. Locally then we write
\[
  \omega = \omega_{i_1 \dots i_k} dx^{i_1} \wedge \cdots \wedge dx^{i_k} = \omega_I dx^I
\]
so we have
\[
  \omega \wedge  \eta = (\omega_I dx^I) \wedge (\eta_J dx^J) = \omega_I \eta_J dx^I \wedge dx^J
\]
and $dx^I \wedge dx^J = 0$ if $I$ and $J$ share an index. So an example is
\[
  \begin{aligned}
    \omega &= \sin x dx \wedge dy + y dz \wedge dy \\
    \eta = dx + e^z dz
  \end{aligned}
\]
and then we get
\[
  \omega \wedge \eta = (\sin x dx \wedge dy + y dz \wedge dy) \wedge (dz + e^zdz) = (\sin x e^z - y) dz \wedge dy \wedge dz
\]

\begin{defn}
  Suppose $F: M \rightarrow N$ smooth. If $T \in \Gamma^\infty(T^kN)$ the pull-back, $F^* T \in \Gamma^\infty(T^kM)$ defined by 
  \[
    F^*T(X_1, \dots, X_k) = T(dF(X_1), \dots, dF(X_k))
  \]
  for all $X_1, \dots, X_k \in \Gamma^\infty(TM)$
\end{defn}

\begin{defn}
  Suppose $F: M \rightarrow N$ smooth. If $T \in \Gamma^\infty(T_lM)$ the push-forward, $F_* T \in \Gamma^\infty(T^lN)$ defined by 
  \[
    F_*T(\omega^1, \dots, \omega^l) = T(F^*\omega^1, \dots, F^*\omega^l)
  \]
  for all $\omega^1, \dots, \omega^l \in \Gamma^\infty(T^*N)$.
\end{defn}
Example is $\omega \in \Omega^kN$ in local coordinates
\[
  \begin{aligned}
    F^*\omega &= F^*(\Omega_{i_1\dotsi_k} dx^{i_1} \wedge \cdots \wedge dx^{i_k}) \\
              &= (\omega_{i_1 \dots i_k} \circ F) d(x^{i_1} \circ F) \wedge \cdots \wedge d(x^{i_k} \circ F) \\
  \end{aligned}
\]
Another fact is that
\[
  F^*(\omega \wedge \eta) = (F^*\omega) \wedge (F^*\eta)
\]
for top dim forms i.e. $\omega \in \Omega^n$ where $\dim M = n$ is $(U, (x^i))$, $(\tilde{U}, (\tilde{x}^i))$ local coordinates, then
\[
  d\tilde{x}^1 \wedge \dots \wedge d\tilde{x}^n = \det\left( \pdv{\tilde{x}^j}{x^i}\right) dx^1 \wedge \cdots \wedge dx^n
\]

\subsection*{Exterior Derivative}%

\begin{thm}
  Suppose $M$ is smooth then there exists a unique operator $d: \Omega^kM \rightarrow \Omega^{k}M$ which satisfies the following 
  \begin{enumerate}
    \item $d$ is linear over $\mathds{R}$
    \item For $\omega \in \Omega^kM$, $\eta \in \Omega^lM$ then
      \[
        d(\omega \wedge \eta) = d\omega \wedge d\eta + (-1)^k \omega \wedge d\eta
      \]
    \item $d \circ d = 0$
    \item $f \in \Omega^0M = C^\infty M$ then $df \in \Omega^1M$ is defined as $df(X) = X(f)$.
  \end{enumerate}
\end{thm}

\begin{prop}
 Let $F: M \rightarrow N$, $\omega \in \Omega^kN$, then 
 \[
   F^*(d\omega) = d(F^*\omega)
 \]
\end{prop}

An example is $\omega = (x^2 + y) dx \wedge dy$. so we have that
\[
  d\omega = (2x dx + dy) \wedge dx \wedge dy + (x^2 + y) \wedge ddx \wedge dy - (x^2 + y) \wedge dx \wedge ddy = (2x dx + dy) \wedge dx \wedge dy
\]

\subsection*{First and Second Fundamental Forms for surfaces}%
Let $S \subset \mathds{R}^3$ be a regular, parametrized surface
\[
  r(u,v) \rightarrow S
\]
The first fundamental form is given by
\[
  \begin{aligned}
    I&: T_pS \times T_pS \rightarrow \mathds{R}\\
     &(x,y) \mapsto \inn*{x}{y}_{\mathds{R}^3}
  \end{aligned}
\]
If we take the basis $r_u = \pdv{r}{u}$, $r_v = \pdv{r}{v}$ gives us a basis for $T_pS$, In this basis $I$ is represented by
\[
  \begin{bmatrix}
    r_u r_u & r_u r_v \\
    r_v r_u & r_v r_v
  \end{bmatrix}
\]
If you just think of this as a weighted inner product. The second fundamental form is given by
\[
  \mathds{I}: T_pS \times T_pS \rightarrow \mathds{R}
\]
We pick coordinates $(x,y,z)$ about $p \in S$ such that $p = (0,0,0)$ and $T_pS = \left\{ z = 0 \right\}$, the $z$-plane tagent to $S$ and $p$. So that $S$ is locally given by $z = f(x,y)$ for some smooth $f$. So this tells us that $f(0,0) = 0$ and $f_u(0,0) = 0$,$f_y(0,0) = 0$. So we take the maclaurin series and get
\[
  f \rightarrow \frac{Lx^2}{2} + Mxy + \frac{Ny^2}{2} + \mathcal{O}(\text{cubic terms})
\]
so in the basis $\left\{ \pdv{x}, \pdv{y} \right\}$, for $T_{(0,0,0)}S$ the second fundamental form is given by
\[
  \begin{bmatrix}
    L & M \\
    M & N
  \end{bmatrix}
  = 
  \begin{bmatrix}
    f_{xx}(0,0) & f_{xy}(0,0) \\
    f_{yx}(0,0) & f_{yy}(0,0)
  \end{bmatrix}
\]
in this basis $\mathds{I}$ is represented by
\[
  \begin{bmatrix}
    r_{uu} \cdot n & r_{uv} \cdot n \\
    r_{vu} \cdot n & r_{vv} \cdot n
  \end{bmatrix}
\]
where $n = r_u \times r_v / |r_u \times r_v|$.  This implies that 
\[
  \mathds{I} = (r_{uu} \cdot n) du^2 + 2(r_{uu} \cdot n) dudv + (r_{vv} \cdot n) dv^2
\]
Where $dudv = du \otimes dv$. 

\subsection*{Second Fundamental form Riemannnian Manifolds}%
In these sections we will let $f: M^n \rightarrow \bar{M}^{m+n}$ be an immersion and let $\bar{M}$ be riemannian with metric $g$. Equip $M$ witht he metric induced by $g$. For every $p \in M$ there exists an open $U$ containing $p$ and $\bar{U}$ containing $f(p)$ so that $(f(U), \bar{U})$ is mapped diffeomorphically on $(B^n, B^{n+m})$ (think of it like a slice chart). Identify $U$ with $f(U)$ and $v \in T_pM$ with their image under $df_p(v) \in T_{f(p)}\bar{M}$. 

For each $p \in M$, using $g$ we can split the tangent space at $T_p\bar{M} = T_pM \oplus (T_pM)^\perp$. For $v \in T_p\bar{M}$ write $v = v^T + v^N$ (this splitting is smooth) meaning that
\[
  \begin{aligned}
    T\bar{M} &\rightarrow T\bar{M} \\
    (p,v) &\mapsto (p, v^T) \\
    (p,v) &\mapsto (p,v^N) 
  \end{aligned}
\]
now let $\bar{\nabla}$ be the levi civita connection on $\bar{M}$. Let $X,Y \in \Gamma^{\infty}(TU)$ and let $\bar{X},\bar{Y} \in \Gamma^\infty(T\bar{U})$ be local extensions. So $\nabla_X Y = (\bar{\nabla}_{\bar{X}}\bar{Y})^T$. Is the levi-civita connection on $U$. 
\begin{defn}
  Let $X,Y$ be local vector fields on $M$, with local extensions $\bar{X}, \bar{Y}$ to $\bar{M}$. Define
  \[
    \begin{aligned}
      B(X,Y) &= \bar{\nabla}_{\bar{X}}\bar{Y} - \nabla_X Y \\
             &= (\bar{\nabla}_{\bar{X}}\bar{Y})^N
    \end{aligned}
  \]
  Is a local vectorfield on $\bar{M}$, defined locally on $M$.
\end{defn}

\begin{lem}
  $B(X,Y)$ doesn't depend on $\bar{X}, \bar{Y}$
\end{lem}
\begin{proof}
  $\bar{X}_1$ and $\bar{X}_2$ are local extension of $X$. Consider
  \[
    (\bar{\nabla}_{\bar{X}_1}\bar{Y} - \nabla_X Y) - (\bar{\nabla}_{\bar{X}_2}\bar{Y} - \nabla_X Y) = \bar{\nabla}_{\bar{X}_1 - \bar{X}_2} \bar{Y} = 0
  \]
  on $M$. By what was proved it also doesn't depend on $\bar{Y}$.
\end{proof}

\begin{prop}
  The map $B: \Gamma^\infty(TU) \times \Gamma^\infty(TU) \rightarrow \Gamma^\infty(TU)^\perp$ is symmetric and $C^\infty(U)$-bilinear
\end{prop}

\begin{proof}
  Clearly $B(X,Y)$ is additive in both arguments by definition of the Levi-Civita connection. Also $B(fX,Y) = fB(X,Y)$ because $\nabla$ and $\bar{\nabla}$ are $C^\infty$ linear in $X$. Show $B(X,fY) = fB(X,Y)$ so let $\bar{f} \in C^\infty(\bar{U})$ be a local extension of $f$. 
  \[
    \begin{aligned}
      B(X,fY) &= \bar{\nabla}_{\bar{X}} (\bar{f}\bar{Y}) - \nabla_X(fY) \\
              &= \bar{X} (\bar{f}) + \bar{f}\nabla_{\bar{X}}\bar{Y} - X(f) Y - f \nabla_X Y
    \end{aligned}
  \]
  since $X \in \Gamma^\infty(TU)$ so $\bar{X}(\bar{f}) = X(\bar{f}) = X(f)$. So along $U$, $\bar{Y} = Y$ we have that
  \[
    B(X,fY) = fB(X,Y)
  \]
  Note by the symmetry of the connection we have
  \[
    \left[ \bar{X}, \bar{Y} \right] = \bar{\nabla}_{\bar{X}}\bar{Y} - \bar{\nabla}{\bar{Y}}\bar{X}
  \]
  So we have that
  \[
    B(X,Y) = \bar{\nabla}_{\bar{X}}\bar{Y} - \nabla_X Y = \left[ \bar{X}, \bar{Y} \right] + \bar{\nabla}_{\bar{Y}}\bar{X} - [X,Y] - \nabla_Y X = \bar{\nabla}_{\bar{Y}}\bar{X} - \nabla_Y \nabla_X = B(Y,X)
  \]
  We further note that $B$ is a tensor.
\end{proof}

\begin{defn}
  Let $p \in M$, $\eta \in (T_pM)^\perp$. Define $H_\eta : T_pM \times T_pM \rightarrow \mathds{R}$ by
  \[
    H_\eta(X,Y) = \inn*{B(X,Y)}{\eta}
  \]
\end{defn}

\begin{defn}
  Let $p \in M$, $\eta \in (T_pM)^\perp$. Then the second fundamental form of an immersion $f:M \rightarrow \bar{M}$ at $p$ along the normal vector $\eta$ is
  \[
    \mathds{I}_\eta(X) = H_\eta(X,X) = \inn*{B(X,X)}{\eta}
  \]
\end{defn}
Note: Other books define the 2nd fundamental form to be $B(X,Y)$ and call $H(X,Y)$ the scalar second fundamental form.

As $H_\eta(X,Y)$ is bilinear and symmetric, there exists a self-adjoint operator $S_\eta: T_pM \rightarrow T_pM$, such that
\[
  \inn*{S_\eta(X)}{Y} = \inn*{X}{S_\eta(Y)} = H_\eta(X,Y) = \inn*{B(X,Y)}{\eta}
\]
$S_\eta$ is called the shape operator.

\begin{lem}[Weingarten equation]
  Let $X,Y \in \Gamma^\infty(TM)$ and $N \in \Gamma^\infty(TM)^\perp$. Then 
  \[
    \inn*{\bar{\nabla}_X N}{Y} = - \inn*{N}{B(X,Y)}
  \]
\end{lem}

\begin{proof}
  So we have that $X \left( \inn*{N}{Y} \right) = 0$ since $\inn*{N}{Y}$. Due to compatilibity  with the metric we have
  \[
    \begin{aligned}
      0 &= \inn*{\bar{\nabla}_{\bar{X}} N}{Y} + \inn*{N}{\bar{\nabla}_{\bar{X}}\bar{Y}} \\
        &= \inn*{\bar{\nabla}_{\bar{X}} N}{Y} + \inn*{N}{B(X,Y) + \nabla_X Y} \\
        &= \inn*{\bar{\nabla}_{\bar{X}} N}{Y} + \inn*{N}{B(X,Y)}
                                                                                 &
    \end{aligned}
  \]
  which gives us that $\inn*{\bar{\nabla}_{\bar{X}} N}{Y} = - \inn*{N}{B(X,Y)}$
\end{proof}

\begin{prop}
  Let $p \in M$, $\eta \in (T_pM)^\perp$, $x \in T_pM$. Let $N$ be a local extension of $\eta$ to a normal vector field along $M$,  then 
  \[
    S_\eta(X) = - (\bar{\nabla}_{\bar{X}} N)^T
  \]
\end{prop}
This shows that the shape object mirrors the gauss map for embedded surfaces.
\begin{proof}
  Let $y  \in T_pM$ and let $X,Y \in \Gamma^\infty(TM)$ be local extensions of $x,y$ then
  \[
    \inn*{N}{Y} = 0
  \]
  and hence
  \[
    \begin{aligned}
      \inn*{S_\eta(x)}{y} &= \inn*{B(X,Y)(p)}{N(p)} \\
                          &= \inn*{\bar{\nabla}_{\bar{X}}\bar{Y} - \nabla_X Y}{N}_p \\
                          &= \inn*{\bar{\nabla}_{\bar{X}}\bar{Y}}{N}_p \\
                          &= -\inn*{Y}{\bar{\nabla}_{\bar{X}} N}_p \\
                          &= -\inn*{\bar{\nabla}_{\bar{X}} N}{Y}_p
    \end{aligned}
  \]
  As shown in the proof of the weingarten equation. Since this holds for arbitrary $x,y$ then we have $S_\eta(x) = -(\bar{\nabla}_{\bar{X}}N)^T$
\end{proof}

Now we can do an example. Suppose $f:M^n \rightarrow \bar{M}^{n+1}$ (a hypersurface, possibly with self-intersections). Now let $p \in M$ with $\eta \in (T_pM)^\perp$, $|\eta| = 1$. Then $S_\eta: T_pM \rightarrow T_pM$ is self-adjoint. This implies that there is an orthonormal eigenbasis $\left\{ b_1, \dots, b_n \right\}$ of $T_pM$ such that $S_\eta(b_i) = \lambda_i b_i$ and $\lambda_i \in \mathds{R}$. If both $M, \bar{M}$ are oriented, then we have a unique choice for $\eta$ by saying $\left\{ b_1, \dots, b_n, \eta \right\}$ to a positive orientation for $T_p\bar{M}$ whenever $\left\{ b_1, \dots, b_n \right\}$ positively oriented bases for $T_pM$. We call the eigenvectors $b_j$ for $S_\eta$ the principal directions of $f$ and $\lambda_j$ the principle curvatures for $f$. and we call
\[
  \det (S_\eta) = \lambda_1 \cdots \lambda_n
\]
the gauss-kronecker or gaussian curvature and we call 
\[
  \frac{1}{n}\left( \lambda_1 + \cdots + \lambda_n \right)
\]
the mean curvature. 
Now consider the a more specific example $f: M^n \rightarrow \mathds{R}^{n+1}$. Let $N$ be a unit normal field which extends $\eta$ locally. So consdier $S^n = \left\{ x \in \mathds{R}^{n+1} | \norm{x} = 1 \right\}$. So we can define the gauss map as
\[
  \begin{aligned}
    G:& M^n \rightarrow S^n \\
      & p \mapsto N(p)
  \end{aligned}
\]
where we identify $T_p\mathds{R}^{n+1} \cong \mathds{R}^{n+1}$ in the usual way. So we think of $S^n \subset T_p\mathds{R}^{n+1}$. Furthermore we $T_pM$ and $T_{G(p)}S^n$ are parallel so we can identify them. Then 
\[
  dG_p:T_pM \rightarrow T_{G(p)}S^n \cong T_pM
\]
is given by
\[
  dG_p(x) = \dv{t} \eval{\left( G \circ \alpha(t) \right)}_{t=0} = \dv{t} \eval{N \circ \alpha(t)}_{t=0} = \bar{\nabla}_{\bar{X}} N
\]
where $\alpha:I \rightarrow M$, $\alpha(0) = p$, $\alpha'(0) = X$, since $\mathds{R}^{n+1}$ is flat, so we can ignore the christoffel symbols. Since $\inn*{N}{N} = 1$ we have
\[
  \begin{aligned}
    0 &= X(\inn*{N}{N}) \\
      &= 2 \inn*{\bar{\nabla}_{\bar{X}}N}{N}
  \end{aligned}
\]
Hence we have that $(\bar{\nabla}_{\bar{X}}N)^N = 0$ and 
\[
  \bar{\nabla}_{\bar{X}}N = \left( \bar{\nabla}_{\bar{X}} N \right)^T
\]
giving us
\[
  dG_p(x) = \left( \bar{\nabla}_{\bar{X}}N \right)^T = -S_\eta (x)
\]
A topological application of Gauss-Kronecker curvature is
\begin{thm}
  Suppose $M^n$ is a compact, connected, orientable manifold, and suppose there exists an immersion $f: M^n \rightarrow \mathds{R}^{n+1}$ with nonvanishing gauss-kronecker curvature at each point of $M$. Then $M^n \cong S^n$.
\end{thm}
Recall that if $X,Y$ are smooth manifolds, then $X$ is a covering space of $Y$ if there exists a local diffeomorphism $F: X \rightarrow Y$, such that for every $q \in Y$, there exists a $U_q$, $q \in U_q$ such that $F^{-1}(U_q) = V_1 \coprod V_2 \coprod \cdots \coprod V_k \cdots$ such that 
\[
  \eval{F}_{V_j}: V_j \rightarrow U_q
\]
is a diffeomorphism. If $F: X \rightarrow Y$ is a local diffeomorphism, $X$ compact then $F$ is a covering map. Also we have that the only covering map of $S^n$ are diffeomorphisms $F: S^n \rightarrow S^n$, $n \geq 2$. Now we can prove the theorem.

\begin{thm}
  If $f:M \rightarrow \mathds{R}^{n+1}$ has nonvanishing GK curvature. Then we have $dG_p \neq 0$ for all $p \in M$. Furthermore $dG_p$ is nonsingular at all $p$. Then by invertible mapping theorem then $G:M \rightarrow S^n$ is a local diffeomorphism. Then $G$ is a covering map, and thus $G$ is a diffeomorphism.
\end{thm}
Let $x,y \in T_pM$ be linearly independent and let $K(x,y)$ be the sectional curvature of $M$ along $\sigma = \text{span} \left\{ x,y \right\}$ and $\bar{k}(x,y)$ be the sectional curvature of $\bar{M}$ along $\sigma$.

\begin{thm}
  Let $p \in M$ and $x,y \in T_pM$ orthonormal. Then 
  \[
    k(x,y) - \bar{k}(x,y) = \inn*{B(x,x)}{B(y,y)} - \inn*{B(x,y)}{B(x,y)}
  \]
\end{thm}

\begin{proof}
  Let $X,Y$ orthonormal extensions of $x,y$ to  $M$, $\bar{X}, \bar{Y}$ are local extensions to $\bar{M}$. 
  \[
    \begin{aligned}
      k(x,y) - \bar{k}(x,y) &= \frac{\inn*{R(x,y)x}{y}}{\norm{x}^2\norm{y}^2 - \inn*{x}{y}^2} - \frac{\inn*{\bar{R}(x,y)x}{y}}{\norm{x}^2 \norm{y}^2 - \inn*{x}{y}^2} \\
                            &= \inn*{\nabla_Y \nabla_X X - \nabla_X \nabla_Y X + \nabla_{[X,Y]}X - \left(\bar{\nabla}_{\bar{Y}} \bar{\nabla}_{\bar{X}} \bar{X} - \bar{\nabla}_{\bar{X}}\bar{\nabla}_{\bar{Y}}\bar{X} + \bar{\nabla}_{[\bar{X},\bar{Y}]} \bar{X} \right)}{Y} \\
                            &= \inn*{\nabla_Y \nabla_X X - \nabla_X \nabla_Y X - \left(\bar{\nabla}_{\bar{Y}} \bar{\nabla}_{\bar{X}} \bar{X} - \bar{\nabla}_{\bar{X}}\bar{\nabla}_{\bar{Y}}\bar{X}\right)}{Y} \\
    \end{aligned}
  \]
  Now let $E^1, \dots, E^m \in \Gamma^{\infty}(TM)^\perp$ be an orthonormal vector field normal to $M$ where $m$ is the codimension of $M$. So we can say
  \[
    B(X,Y) = H_{E^i}  \left( X,Y \right) E^i
  \]
  Thus we have that
  \[
    \begin{aligned}
      \bar{\nabla}_{\bar{Y}} \bar{\nabla}_{\bar{X}} \bar{X} &= \bar{\nabla}_{\bar{Y}} \left( B(x,x) + \nabla_X X\right) = \bar{\nabla}_{\bar{Y}} \left( H_{E^i}(X,X)E^i + \nabla_X X \right)  \\
                                                            &= \bar{Y} \left( H_{E^i}(x,x) \right) E^i + H_{E^i}(x,x) \bar{\nabla}_{\bar{Y}} E^i + \bar{\nabla}_{\bar{Y}} \nabla_X X
    \end{aligned}
  \]
  So 
  \[
    \begin{aligned}
      \inn*{\bar{\nabla}_{\bar{Y}} \bar{\nabla}_{\bar{X}} \bar{X}}{Y} &= \inn*{H_{E^i}(x,x) \bar{\nabla}_{\bar{Y}} E^i + \nabla_Y \nabla_X X}{Y} \\
                                                                      &= H_{E^i}(x,x) \inn*{\bar{\nabla}_{\bar{Y}}E^i}{Y} + \inn*{\nabla_Y \nabla_X X}{Y} \\
                                                                      &= -H_{E^i}(x,x) \inn*{E^i}{\bar{\nabla}_{\bar{Y}} \bar{Y}} + \inn*{\nabla_Y \nabla_X X}{Y} \\
                                                                      &= -H_{E^i}(x,x) \inn*{\bar{\nabla}_{\bar{Y}} \bar{Y} - \nabla_Y Y}{E^i} + \inn*{\nabla_Y \nabla_X}{Y} \\
                                                                      &= -H_{E^i}(x,x) \inn*{B(Y,Y)}{E^i} + \inn*{\nabla_Y \nabla_X X}{Y} \\
                                                                      &= \sum_i -H_{E^i}(x,x) H_{E^i}(Y,Y) + \inn*{\nabla_y \nabla_X X}{Y} \\
    \end{aligned}
  \]
  So we have
  \[
    k(x,y) - \bar{k}(x,y) = \sum_i H_{E^i}(x,x) H_{E^i}(y,y) - \sum_i H_{E^i}(x,y)^2
  \]
\end{proof}
So what does this tell us about gaussian curvature? Let $f: M^n \rightarrow \bar{M}^{n+1}$ and let $\eta \in T_pM^\perp$. Let $\left\{ e_1, \dots, e_n \right\}$ be an orthonormal basis for $T_pM$ consisting of eigenvectors of $S_\eta$ so that $S(e_i) = \lambda_i e_i$ so we have
\[
  H_\eta (e_i, e_i) = \lambda_i
\]
\[
  H_\eta (e_i, e_j) = 0
\]
So this tells us that
\[
  k(e_i, e_j) - \bar{k}(e_i,e_j) = \lambda_i \lambda_j
\]
when $n =2$ and $\bar{M} = \mathds{R}^3$ so
\[
  \lambda_1 \lambda_2 = k(e_i, e_2) - \bar{k}(e_1, e_2) = k(e_1, e_2)
\]

\begin{defn}
  $f: M^n \rightarrow \bar{M}^{n+m}$ immersion, is geodesic at $p \in M$ if for all $\eta \in T_pM^\perp$,
  \[
    H_\eta = 0
  \]
  at $p$ and $f$ is totally geodesic if $f$ is geodesic at all $p \in M$.
\end{defn}

\begin{prop}
  An immersion $f:M \rightarrow \bar{M}$ is geodesic at $p$ if and only if every geodesic $\gamma$ of $M$ starting from $p$ is also a geodesic of $\bar{M}$ starting at $p$.
\end{prop}

\begin{proof}
  Let $\gamma(0) = p$, $\gamma'(0) = x \in T_pM$ and let $\eta \in \left( T_pM \right)^\perp$, and $N \in \Gamma^\infty(TM)^\perp$ which is a local extension of $\eta$. Let $X \in \Gamma^\infty(TM)$ be a local extension of the vector $x$ to $M$. So we know that $\inn*{X}{N}$, which tells us $\inn*{\bar{\nabla}_{\bar{X}} \bar{X}}{\bar{N}} + \inn*{X}{\bar{\nabla}_{\bar{X}} \bar{N}}$ so we have that
  \[
    H_\eta(x,x) = \inn*{S_\eta (x)}{x} = \inn*{-(\bar{\nabla}_{\bar{X}} N)^T}{X}_p = - \inn*{\bar{\nabla}_{\bar{X}}N}{X}_p = \inn*{N}{\bar{\nabla}_{\bar{X}} \bar{X}}_p
  \]
  So $f$ is geodesic at $p \in M$ if and only if $H_\eta$ vanishes at all $\eta \in T_pM^\perp$ if and only if $\bar{\nabla}_{\bar{X}} \bar{X}$ has no normal component if and only if every geodesic of $M$ starting at $p$ is also a geodesic of $\bar{M}$ from $p$.
\end{proof}
A nonexample of this is $S^2$ embedded in $\mathds{R}^3$ is not geodesic at any $p$. Some examples are: Affine subspaces of $\mathds{R}^n$ and $S^n \cap \left\{ \text{linear subspaces} \right\}$ totally geodesic inside $S^n$. 

Using this new information, we can make a geometric interpretation of sectional curvature. Let $p \in M$, $B \subset T_pM$ a small ball on which $\exp_p$ is a diffeomorpism and $\sigma \subset T_pM$ a 2 dimensional subspace. Let $S = \exp+p(\sigma \cap B)$. So $S$ is a is a surface on $M$ formed by short arcs through $p$ tangent to $\sigma$. $S$ is geodesic at $p$ by construction and therefore $H_\eta = 0$ for all $\eta \in T_pM^\perp$ and therefore $H_\eta(x,y) = \inn*{B(x,y)}{\eta}$ and thus $B(x,y) = 0$, $S_\eta = 0$ and by guass's formula we have
\[
  k_s(x,y) = \bar{k}_{\bar{M}}(x,y)
\]
in other words, geodesic submanifolds have the same sectional curvature as their ambient space.
\begin{thm}[Cartan]
  If for every $p \in M$,  and for every $\sigma \in T_pM$, there is a a totally geodesic submanifold to $\sigma$, then $M$ has constant sectional curvature.
\end{thm}

\begin{defn}
  $f: M \rightarrow \bar{M}$ immersion is minimal if for every $p \in M$ and every $\eta \in (T_pM)^\perp$ $\tr \left( S_\eta \right)=0$.
\end{defn}

A consequence of this is that mininmal submanifolds minimize volume. If $E^1, \dots, E^n$ is a local orthonormal frame along $M$, we can define.
\[
  H = \frac{1}{n} \tr(S_{E^i})E^i
\]
this is called the mean curvature. Clearly $f$ is minimal, meaning that $\tr(S_\eta) = 0$ for all $\eta$ if and only if $H = 0$ for all $p$.

Questions: How does geometric data constrain topology?

\begin{thm}[Sphere]
  If $M^n$ compact, simply connect and sectional curvature $k$ with max value $k_{\max}$, and if
  \[
    0 < k_{\max}/4 < k \leq k_{\max}
  \]
  $M^n$ is homoemorphic to $S^n$.
\end{thm}

\begin{thm}[Hadamard]
  $M$ is complete, simply connected, with nonpositive sectional curvature. Then $M$ is diffeomorphic to $\mathds{R}^n$ and $\exp_p:T_pM \rightarrow M$ is the diffeomorphism.
\end{thm}

\end{document}
