%!TEX TS-program = xelatex
%!TEX encoding = UTF-8 Unicode

\documentclass[a4paper]{article}

\usepackage{xltxtra}
\usepackage{amsfonts}
\usepackage{polyglossia}
\usepackage{fancyhdr}
\usepackage{geometry}
\usepackage{dsfont}
\usepackage{amsmath}
\usepackage{amsthm}
\usepackage{amssymb}
\usepackage{physics}
\usepackage[shortlabels]{enumitem}
\usepackage{mathtools}

\geometry{a4paper,left=15mm,right=15mm,top=20mm,bottom=20mm}
\pagestyle{fancy}
\lhead{Devon Morris}
\chead{Differential Geometry - Homework 6}
\rhead{\today}
\cfoot{\thepage}

\setlength{\headheight}{23pt}
\setlength{\parindent}{0.0in}
\setlength{\parskip}{0.0in}

\newtheorem*{prop}{Proposition}
\newtheorem*{defn}{Definition}

\DeclarePairedDelimiterX{\inn}[2]{\langle}{\rangle}{#1, #2}

\begin{document}
Denote by $(u,v)$ the cartesian coordinates of $\mathds{R}^2$. Show that the function $\varphi: U \rightarrow \mathds{R}^3$ given by 
\[
  \varphi(u,v) = (f(v) \cos u, f(v) \sin(u), g(v)),
\]
\[
  U = \left\{ (u,v) \in \mathds{R}^2: u_0 < u < u_1; v_0 < v < v_1 \right\},
\]
where $f$ and $g$ are differentiable functions, with $f'(v)^2 + g'(v)^2 \neq 0$ and $f(v) \neq 0$, is an immersion. The image $\varphi(U)$ is the surface generated by the rotation of the curve $(f(v), g(v))$ around the axis $0z$ and is called a surface of revolution $S$. The image by $\varphi$ of the curves $u = $ constant and $v = constant$ are called meridians and parallels, respectively, of $S$.
\begin{enumerate}[a)]
  \item Show that the induced metric in the coordinates $(u,v)$ is given by 
    \[
      g_{11} = f^2, \quad g_{12} = 0, g_{22} = (f')^2 + (g')^2
    \]
  \item Show that the local equations of a geodesic $\gamma$ are
    \[
      \begin{aligned}
        &\dv[2]{u}{t} + \frac{2 ff'}{f^2} \dv{u}{t} \dv{v}{t} = 0 \\
        &\dv[2]{v}{t} - \frac{ff'}{(f')^2 + (g')^2} \left( \dv{u}{t} \right)^2 + \frac{f'f'' + g'g''}{(f')^2 + (g')^2} \left( \dv{v}{t} \right)^2 = 0
      \end{aligned}
    \]
  \item Obtain the following geometric meaning of the equations above: the second equation is, except for meridians and parallels, equivalent to the fact that the "energy" $|\gamma'(t)|^2$ of a geodesic is constant along $\gamma$; the first equation signifies that if $\beta(t)$ is the oriented angle, $\beta(t) < \pi$, of $\gamma$ with a parallel $P$ intersection $\gamma$ at $\gamma(t)$, then 
    \[
      r \cos \beta = \text{const.}
    \]
    where $r$ is the radius of the parallel $P$.
\end{enumerate}

\begin{proof}[Solution.]
  \begin{enumerate}[a)]
    \item First we note that in local coordinates the differential under this chart is given by
      \[
        \begin{bmatrix}
          -f(v)\sin u &  f'(v) \cos(u) \\
          f(v) \cos u & f'(v) \sin(u) \\
          0 & g'(v)
        \end{bmatrix}
      \]
      which implies that
      \[
        d \varphi \pdv{u} = 
        \begin{bmatrix}
          -f(v) \sin u \\
          f(v) \cos u \\
          0
        \end{bmatrix}
        \quad 
        d \varphi \pdv{v} = 
        \begin{bmatrix}
          f'(v)\cos(u) \\
          f'(v) \sin(u) \\
          g'(v)
        \end{bmatrix}
      \]
      where the local coordinates on $\mathds{R}^3$ are given by the identity map. Using the standard metric on $\mathds{R}^3$ we have
      \[
        \begin{aligned}
          g_{11} &= f^2(v) \left( \sin^2 u + \cos^2 u \right) = f^2(v) \\
          g_{12} &= -f(v)\sin u f'(v)\cos u + f(v)\cos u f'(v) \sin u = 0 \\
          g_{22} &= (f'(v))^2(\cos^2 u + \sin^2 u) + (g'(v))^2 = (f'(v))^2 + (g'(v))^2
        \end{aligned}
      \]
    \item To do this we first need to find the levi-civita connection and the associated christoffel symbols. We have that $\Gamma_{12}^k = \Gamma_{21}^k$ since we are using the levi civita connection, so we will first take partial derivatives
      \[
        \begin{aligned}
          &\pdv{g_{11}}{u} = 0 \quad &\pdv{g_{11}}{v} = 2ff' \\
          &\pdv{g_{12}}{u} = 0 \quad &\pdv{g_{12}}{v} = 0 \\
          &\pdv{g_{22}}{u} =  0 \quad &\pdv{g_{22}}{v} = 2f'f'' + 2g'g''
        \end{aligned}
      \]
      Thus we have that
      \[
        \begin{aligned}
          \Gamma^1_{11} &= \frac{1}{2}g^{11} \left(\pdv{g_{11}}{u} + \pdv{g_{11}}{u} - \pdv{g_{11}}{u}  \right) = 0 \\
          \Gamma^1_{12} &= \frac{1}{2}g^{11} \left(\pdv{g_{11}}{v} + \pdv{g_{12}}{u} - \pdv{g_{11}}{u}  \right) = \frac{ff'}{f^2} \\
          \Gamma^1_{22} &= \frac{1}{2}g^{11} \left(\pdv{g_{12}}{v} + \pdv{g_{12}}{v} - \pdv{g_{22}}{u}  \right) = 0 \\
          \Gamma^2_{11} &= \frac{1}{2}g^{22} \left(\pdv{g_{12}}{u} + \pdv{g_{12}}{u} - \pdv{g_{11}}{v}  \right) = -\frac{ff'}{(f')^2 + (g')^2} \\
          \Gamma^2_{12} &= \frac{1}{2}g^{22} \left(\pdv{g_{12}}{v} + \pdv{g_{22}}{u} - \pdv{g_{12}}{v}  \right) = 0 \\
          \Gamma^2_{22} &= \frac{1}{2}g^{22} \left(\pdv{g_{22}}{v} + \pdv{g_{22}}{v} - \pdv{g_{22}}{v}  \right) = \frac{f'f'' + g'g''}{(f')^2 + (g')^2} \\
        \end{aligned}
      \]
      So we have the following 
      \[
        \begin{aligned}
          \dv[2]{u}{t} + \left(\dv{u}{t}\right) \Gamma_{11}^1 + 2\dv{u}{t}\dv{v}{t} \Gamma_{12}^1 + \left(\dv{v}{t}\right)\Gamma_{22}^1 &= \dv[2]{u}{t} + \frac{2ff'}{f^2} \dv{u}{t}\dv{v}{t} = 0 \\
          \dv[2]{v}{t} + \left(\dv{u}{t}\right) \Gamma_{11}^2 + 2\dv{u}{t}\dv{v}{t} \Gamma_{12}^2 + \left(\dv{v}{t}\right)\Gamma_{22}^2 &= \dv[2]{v}{t} - \frac{ff'}{(f')^2 + (g')^2} \left(\dv{u}{t}\right)^2 + \frac{f'f'' + g'g''}{(f')^2 + (g')^2}\left( \dv{v}{t} \right)^2 = 0 \\
        \end{aligned}
      \]
    \item Consider the energy given by
      \[
        |\gamma'(t)|^2 = f^2 \left(\dv{u}{t}\right)^2 + ((f')^2 + (g')^2) \left(\dv{v}{t}\right)^2
      \]
      Taking the time derivative of this quantity yields
      \[
        \begin{aligned}
          2ff' \left( \dv{u}{t} \right)^2 \dv{v}{t} &+ 2 f^2 \dv{u}{t} \dv[2]{u}{t} +  (2f'f'' + 2g'g'') \left(\dv{v}{t}\right)^3  + 2((f')^2 + (g')^2) \dv{v}{t} \dv[2]{v}{t} \\
                                                    &= 2\dv{u}{t}\left( 2 ff' \dv{u}{t} + f^2 \dv[2]{u}{t} \right) + 2\dv{v}{t} \left( ((f')^2 + (g')^2) \dv[2]{v}{t} + (f'f'' + g'g'')\left(\dv{v}{t}\right) - ff' \left(\dv{u}{t}\right)^2 \right) \\
                                                    &= 0
        \end{aligned}
      \]
      So we have that the energy is constant. Now we note that along any parallel $P$ that the component of the tangent vector along $\pdv{v}$ must be zero. Without loss of generality assume that the tangent vector is given by$\pdv{u}$ (this is fine since we will just divide by the length later). We now wish to compute $r \cos \beta$. We first note that the radius is given by $f$, now we compute $\cos \beta$. Let $k = |\gamma'(t)|$, which must be constant, we have that
      \[
        \begin{aligned}
          \cos \beta &=  \inn*{\dv{u}{t} \pdv{u} + \dv{v}{t} \pdv{v}}{\pdv{u}}/ k\\
                     &=  \frac{f^2}{k} \dv{u}{t}
        \end{aligned}
      \]
      Now let's take the time derivative of $f \cos \beta = \frac{f^3}{k} \dv{u}{t}$, and we get
      \[
        \begin{aligned}
          3\frac{f^2f'}{k}\dv{u}{t}\dv{v}{t} + \frac{f^3}{k} \dv[2]{u}{t} = \frac{f^3}{k} \left(\dv[2]{u}{t} + 3\frac{ff'}{f^2}\dv{u}{t}\dv{v}{t}\right)
        \end{aligned}
      \]
  \end{enumerate}
\end{proof}

\end{document}
