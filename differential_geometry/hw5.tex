%!TEX TS-program = xelatex
%!TEX encoding = UTF-8 Unicode

\documentclass[a4paper]{article}

\usepackage{xltxtra}
\usepackage{amsfonts}
\usepackage{polyglossia}
\usepackage{fancyhdr}
\usepackage{geometry}
\usepackage{dsfont}
\usepackage{amsmath}
\usepackage{amsthm}
\usepackage{amssymb}
\usepackage{physics}
\usepackage[shortlabels]{enumitem}
\usepackage{mathtools}

\geometry{a4paper,left=15mm,right=15mm,top=20mm,bottom=20mm}
\pagestyle{fancy}
\lhead{Devon Morris}
\chead{Differential Geometry - Homework 5}
\rhead{\today}
\cfoot{\thepage}

\setlength{\headheight}{23pt}
\setlength{\parindent}{0.0in}
\setlength{\parskip}{0.0in}

\newtheorem*{prop}{Proposition}
\newtheorem*{defn}{Definition}

\DeclarePairedDelimiterX{\inn}[2]{\langle}{\rangle}{#1, #2}

\begin{document}
\section*{Problem 1}%
Let $M$ be a smooth manifold with Riemannian metric $g$. Let $\gamma: [a,b] \rightarrow M$ be a smooth curve, and $\tau: [c,d] \rightarrow [a,b]$ a smooth map. Prove that the length of the curve $\gamma$ is the same as the length of the curve $\gamma \circ \tau: [c,d] \rightarrow M$.

\begin{proof}
Lets first start by examining the length of the curve $\gamma \circ \tau$ 
\[
  \begin{aligned}
    \ell_c^d(\gamma \circ \tau) &= \int_c^d \sqrt{g\left(\dv{\gamma \circ \tau}{t}, \dv{\gamma \circ \tau}{t}\right)}\ dt \\
             &= \int_c^d \sqrt{g\left(\dv{\tau}{t} \dv{\gamma}{\tau}, \dv{\tau}{t} \dv{\gamma}{\tau}\right)}\ dt
  \end{aligned}
\]
At this point, we note that $\dv{\tau}{t}$ is a scalar, so we have
\[
  \begin{aligned}
    \ell_c^d(\gamma \circ \tau) &= \int_c^d \left|  \dv{\tau}{t} \right|\sqrt{g\left(\dv{\gamma}{\tau}, \dv{\gamma}{\tau}\right)}\ dt \\
             &= \int_a^b \sqrt{g \left( \dv{\gamma}{\tau}, \dv{\gamma}{\tau} \right)} d\tau \\
             &= \ell_a^b(\gamma)
  \end{aligned}
\]
using substitution. Therefore, both $\gamma$ and $\gamma \circ \tau$ have the same length.
\end{proof}

\section*{Problem 2}%
Consider the metric on $S^n \subset \mathds{R}^{n+1}$ induced from the standard metric on $\mathds{R}^{n+1}$. Show that the antipodal map $\alpha: S^n \rightarrow S^n$ is an isometry. Use this fact to define a metric on $\mathds{RP}^n$ such that the projection map $S^n \rightarrow \mathds{RP}^n$ is a local isometry.

\begin{proof}
  We note that if we think of $S^n$ as a subset of $\mathds{R}^3$ that our antipodal map is given by $\alpha(p) = -p$. Thus we have that $d\alpha_p(V) = -V$. Thus we have 
  \[
    g_p(V,W) = g_p(-V,-W) = g_{-p}(d\alpha_p(V), d\alpha(W))
  \]
  therefore, $\alpha$ is an isometry. Now we wish to define a metric $g^{\mathds{RP}^n}$ so that that makes $\pi: S^3 \rightarrow \mathds{RP}^2$ a local isometry. Let $U_i \subset S^n$, be the set $ U_i = \left\{ x \in S^n | x^i > 0 \right\}$. We note that $\eval{\pi}_{U_i}: U_i \rightarrow \mathds{RP}^n$ is a diffeomorphism onto its image. So there is a $(\eval{\pi}_{U_i})^{-1}: \pi(U_i) \rightarrow U_i$. Consider the candidate 
  \[
    g^{\mathds{RP}^n}_{[p]}(V,W) = g_p\left(\left(d\pi_p\right)^{-1} V, \left(d\pi_p\right)^{-1} W  \right)
  \]
  This automatically satisfies the properties of the inner product, due to differential being linear and the properties of $g$. Now we just have to make sure that $g^{\mathds{RP}^n}$ is well defined. Now we note that $\alpha^{-1} = \alpha$, so $d\alpha = (d\alpha)^{-1}$ and $\pi \circ \alpha = \pi$. Since $[p] \sim [-p]$ in $\mathds{RP}^n$
  \[
    \begin{aligned}
      g_{[-p]}^{\mathds{RP}^n}(V,W) &= g_{-p}\left(\left(d\pi_{-p}\right)^{-1} V, \left(d\pi_{-p}\right)^{-1} W  \right) \\
                                    &= g_{p}\left((d\alpha_p)^{-1}\left(d\pi_{-p}\right)^{-1} V, (d\alpha_p)^{-1}\left(d\pi_{-p}\right)^{-1} W  \right) \\
                                    &= g_{p}\left(\left(d(\pi \circ \alpha)_{p}\right)^{-1} V, \left(d(\pi \circ \alpha)_{p}\right)^{-1} W \right) \\
                                    &= g_{p} \left( (d\pi_p)^{-1}V, (d\pi_p)^{-1}W \right) \\
                                    &= g_{[p]}^{\mathds{RP}^n}(V,W)
    \end{aligned}
  \]
  So this metric is well-defined. Therefore we have created a local isometry.
\end{proof}

\section*{Problem 3}%
Let $M$ be a smooth path-connected manifold with Riemannian metric $g$. Recall that $\ell_a^b(\gamma)$ denotes the length of a smooth curve $\gamma: [a,b] \rightarrow M$. For any $p,q \in M$, let
\[
  d(p,q) = \inf \left\{ \ell_a^b(\gamma)\ |\ \gamma:[a,b] \rightarrow M \text{ is piecewise smooth with } \gamma(a)=p, \gamma(b) = q \right\}
\]
prove that the pair $(M,d)$ is a metric space.

\begin{proof}
  We will show that the 4 properties of metric spaces hold
  \begin{enumerate}
    \item We note that $d(p,q) \geq 0$, since $g \left( \dv{\gamma}{t}, \dv{\gamma}{t} \right) \geq 0$ by the properties of the Riemannian metric.
    \item We note  that if $p = q$ then there exists a gamma with $\dv{\gamma}{t} = 0$ so that $\ell^b_a = 0$ and thus $d(p,q) = 0$. Now suppose that $p \neq q$ every $\gamma$ connecting $p$ and $q$ must have $\dv{\gamma}{t} \neq 0$ on a set with positive measure, so $d(p,q) \neq 0$.
    \item We note that $d(p,q) = d(q,p)$ because
      \[
        \begin{aligned} 
          \ell_a^b(\gamma) &= \int_a^b \sqrt{ g \left( \dv{\gamma}{t}, \dv{\gamma}{t} \right)}\ dt\\
                   &= \int_b^a \sqrt{ g \left( \dv{\gamma}{t}, \dv{\gamma}{t} \right)}\ (-dt)\\
                   &= \int_b^a \sqrt{ g \left( -\dv{\gamma}{s}, -\dv{\gamma}{s} \right)}\ ds\\
                   &= \int_b^a \sqrt{ g \left( \dv{\gamma}{s}, \dv{\gamma}{s} \right)}\ ds\\
                   &= \ell_b^a(\gamma)
        \end{aligned}
     \]
     and since piecewise curves with $\gamma(a) = p$, $\gamma(b) = q$ can be reparameterized so that $\gamma'(b)=p$ and $\gamma'(a) = q$ so the curves connecting $p$ and $q$ are essentially the same up to direction.
   \item Consider $p,q,r \in M$ we note that
     \[
       \begin{aligned}
         d(p,q) =& \inf \left\{ \ell_a^b(\gamma)\ |\ \gamma:[a,b] \rightarrow M \text{ is piecewise smooth with } \gamma(a)=p, \gamma(b) = q \right\} \\
         \leq& \inf \left\{ \ell_a^b(\gamma)\ |\ \gamma:[a,b] \rightarrow M \text{ is piecewise smooth with } \gamma(a)=p, \gamma(b) = q, \gamma(c) = r, c \in [a,b] \right\} \\
         \leq& \inf \left\{ \ell_a^c(\gamma)\ |\ \gamma:[a,c] \rightarrow M \text{ is piecewise smooth with } \gamma(a)=p, \gamma(c) = r \right\} \\
                &+\inf \left\{ \ell_c^b(\gamma)\ |\ \gamma:[c,b] \rightarrow M \text{ is piecewise smooth with } \gamma(c)=r, \gamma(b) = q \right\} \\
                &= d(p,r) + d(r,q)
       \end{aligned}
     \]
  \end{enumerate}
\end{proof}

\section*{Problem 4}%
Let $F: M \rightarrow N$ be a smooth immersion, and let $g_N$ be a Riemannian metric on $N$. Let $(U, \varphi) = (U, (x^j))$ and $(V, \psi) = (V, (y^j))$ be charts on $M$ and $N$ with $F(U) \subset V$, and let $(g_N)_{ij}$ be the local coordinate description of $g_N$ under the chart $(V, \psi)$. If $g_M$ is the metric induced on $M$ from the immersion $F$, describe how the local coordinate description of $g_M$ under $(U, \varphi)$, denoted by $(g_M)_{ij}$ are related to functions $(g_N)_{ij}$.

\begin{proof}[solution]
 We first know that since $g_M$ is the induced metric that
 \[
   g_M(X,Y) = g_N\left( dF_p(X), dF_p(Y) \right)
 \]
 We are looking for $(g_M)_{ij}$ at the point $p \in M$, which is given by
 \[
   (g_M)_{ij} = g_M\left(\pdv{x^i}, \pdv{x^j}\right) = g_N \left( dF_p \left( \pdv{x^i} \right), dF_p\left(\pdv{x^j}\right) \right)
 \]
 At this point we note that in coordinates $\pdv{x^i}$ is just the standard basis vector $e^i$ in coordinates $dF_p \left( \pdv{x^i} \right)$ is
 \[
   W^k = \pdv{\hat{F}^k}{x^j}\delta^{ij} = \pdv{\hat{F}^k}{x^i} 
 \]
 This implies that 
 \[
   dF_p \left( \pdv{x^i} \right) = \pdv{\hat{F}^j}{x^i} \pdv{y^j}
 \]
 So we have that
 \[
   \begin{aligned}
     (g_M)_{ij} &= g_N \left( \pdv{\hat{F}^k}{x^i} \pdv{y^k}, \pdv{\hat{F}^l}{x^j} \pdv{y^l} \right)\\
                &= \pdv{\hat{F}^k}{x^i} \pdv{\hat{F}^l}{x^j} g_N \left( \pdv{y^k}, \pdv{y^l} \right) \\
                &= \pdv{\hat{F}^k}{x^i} \pdv{\hat{F}^l}{x^j} (g_N)_{kl}
   \end{aligned}
 \]
 Which makes sense because $g_N$ can be thought of as a (0,2) tensor, so it must transform covariantly.
\end{proof}





\end{document}
