%!TEX TS-program = xelatex
%!TEX encoding = UTF-8 Unicode

\documentclass[a4paper]{article}

\usepackage{xltxtra}
\usepackage{amsfonts}
\usepackage{polyglossia}
\usepackage{fancyhdr}
\usepackage{geometry}
\usepackage{dsfont}
\usepackage{amsmath}
\usepackage{amsthm}
\usepackage{amssymb}
\usepackage{physics}
\usepackage{mathtools}

\geometry{a4paper,left=15mm,right=15mm,top=20mm,bottom=20mm}
\pagestyle{fancy}
\lhead{Devon Morris}
\chead{Differential Geometry}
\rhead{\today}
\cfoot{\thepage}

\setlength{\headheight}{23pt}
\setlength{\parindent}{0.0in}
\setlength{\parskip}{0.0in}

\newtheorem*{prop}{Proposition}
\newtheorem*{defn}{Definition}
\newtheorem*{thm}{Theorem}
\newtheorem*{cor}{Corollary}
\newtheorem*{lem}{Lemma}
\newtheorem*{rem}{Remark}

\DeclarePairedDelimiterX{\inn}[2]{\langle}{\rangle}{#1, #2}

\begin{document}
\section*{Curvature}%

\subsection*{Curvature for curves and surfaces}%
Let $\alpha: I \rightarrow \mathds{R}^3$ be a curve parameterized by arc length. We have
\[
  T(s) =  \dv{\alpha}{s} = \alpha'(s)
\]
is the unit tangent vector and
\[
  N(s) = \frac{T'(s)}{\norm{T'(s)}} = \frac{a''(s)}{\norm{\alpha''(s)}}
\]
There is some plane by the unit normal and unit tangent vector called the osculating plane. If we put a circle such that has the same tangent as the curve and whose center lies along the subspace spanned by $N(s)$. We call this circle the osculating circle. If we let $r(s)$ be the radius of the circle at the point $\alpha(s)$, we define $r(s) = 1/k(s)$ is the curvature of $\alpha(s)$. It can be shown that 
\[
  k(s) = \frac{\norm{\alpha' \times \alpha''}}{\norm{\alpha'}^3}
\]
For surfaces we let $r: A \rightarrow \mathds{R}$ be a local paramaterization of a regular surface $S \subset \mathds{R}^3$. Then the unit normal is given by 
\[
N(p) = \frac{r_u \times r_v}{\norm{r_u \times r_v}}
\]
This gives a map $N: S \rightarrow S^2$, from the surface to the 2-sphere. We can identity $T_pS \cong T_{N(p)}S^2$. This gives us that $dN_p: T_pS \rightarrow T_{N(p)}S^2 \cong T_pS$, which is an automorphism which allows us to talk about the determinant
\begin{defn}
  The determinant differential of our normal map $\tilde{k}(p) = \det (dN_p)$ is called the Gaussian curvature of $S$ at $p$.
\end{defn}
We can think of the curvature of a surface instead as curves living in planes contiaining $N(p)$. We call the max and min curvatures of these curves the principle curvatures of $S$ at $p$. We note that both of these notions are extrinsic, in that they rely on some embedding in $\mathds{R}^3$. To define curvature natively on the manifold, we need a connection.

\subsection*{Curvature on a Riemannian Manifold}%

\begin{defn}
  The curvature $R$ of a Riemannian manifold is a correspondance which associates to every pair $X,Y \in \Gamma^{\infty}(TM)$ a mapping 
  \[
    \begin{aligned}
      R(X,Y): &\Gamma^{\infty}(TM) \rightarrow \Gamma^{\infty}(TM) \\
              & Z \mapsto \nabla_Y\nabla_X Z - \nabla_X \nabla_Y Z + \nabla_{[X,Y]} Z
    \end{aligned}
  \]
  for all $Z \in \Gamma^{\infty}(TM)$, where $\nabla$ is the Levi-Civita connection. We denote this operation as $R(X,Y)Z$ or sometimes as $R(X,Y,Z)$.
\end{defn}

Intuitively, we know that $\mathds{R}^n$ has zero curvature. So let's see that our definition agrees. Recall that in $\mathds{R}^n$ our Christoffel symbols are $\Gamma_{ij}^k = 0$. So we have
\[
  \begin{aligned}
    \nabla_X \nabla_Y Z &= \nabla_X \left( Y^j \pdv{Z^i}{x^j} \pdv{x^i} \right) = X^k \pdv{x^k} \left( Y^j \pdv{Z^i}{x^j} \right) \pdv{x^i} \\
                        &= X^k \pdv{Y^j}{x^k} \pdv{Z^i}{x^j} \pdv{x^i} + X^k Y^j \pdv[2]{Z^i}{x^k}{x^j} \pdv{x^i}
  \end{aligned}
\]
From this it is easily seen that
\[
  \nabla_X \nabla_Y Z - \nabla_Y \nabla_X Z = \left( X^k \pdv{Y^j}{x^k} \pdv{Z^i}{x^j} - Y^k \pdv{X^j}{x^k} \pdv{Z^i}{x^j} \right) \pdv{x^i}
\]
In coordinates our lie bracket is
\[
  [X,Y] = \left(X^j \pdv{Y^i}{x^j} - Y^j\pdv{X^i}{x^j} \right)\pdv{x^i}
\]
So we have that
\[
  \nabla_{[X,Y]} Z = \left( X^k \pdv{Y^j}{x^k} - Y^k \pdv{X^j}{x^k} \right) \pdv{Z^i}{x^j} \pdv{x^i}
\]
So we have that
\[
  R(Y,X)Z = \nabla_X \nabla _Y Z - \nabla_Y \nabla_X Z - \nabla_{[X,Y]} = 0
\]
for all $X,Y,Z \in \Gamma^{\infty}(T\mathds{R}^n)$. In this way we think of $\mathds{R}^n$ as being flat.

\subsection*{Geometric Interpretation}%
Let $X,Y \in \Gamma^{\infty}(TM)$ with $[X,Y] = 0$. Let $\tau_t$ and $\rho_t$ be the parallel transport map along the flows of $X$ and $Y$ respectively up to time $t$.  Then
\[
  \tau_t^{-1}\rho_s^{-1} \tau_t \rho_s: T_pM \rightarrow T_pM
\]
and
\[
  \dv{t}\dv{s} \eval{\left(\tau_t^{-1}\rho_s^{-1} \tau_t \rho_s \right)}_{t=s=0} = R(X,Y)Z
\]
i.e $R(X,Y)Z$ measures the failure of parallel transport to commute around parallelograms on $M$.


Recall that
\[
  R(X,Y)Z = \nabla_Y \nabla_X Z - \nabla_X \nabla_Y Z + \nabla_{[X,Y]} Z
\]
In coordinates, $(U, \varphi)= (U, (x^i))$ be local coordinates on $M$. Notice that in coordinates
\[
  \left[ \pdv{x^i}, \pdv{x^j} \right] = 0
\]
So we have that
\[
  R \left( \pdv{x^j}, \pdv{x^i} \right) \pdv{x^k} = \left(\nabla_{\pdv{x^i}} \nabla_{\pdv{x^j}} - \nabla_{\pdv{x^j}}\nabla_{\pdv{x^i}}\right) \pdv{x^k}
\]

\begin{prop}
 Let $M$ be a Riemannian manifold.  Then $R(X,Y)Z$ is $C^{\infty}$ linear in $X,Y,Z$, i.e.
     \[
       R(fX_1 + gX_2, Y)Z = fR(X_1,Y)Z + gR(X_2, Y)Z
     \]
     \[
       R(X, fY_1+gY_2)Z = fR(X,Y_1)Z + R(X, Y_2)Z
     \]
     \[
       R(X, Y)(fZ_1 + gZ_2) = fR(X,Y)Z_1 + R(X, Y)Z_2
     \]
\end{prop}

\begin{proof}
 Linearity in $X,Y$ left as a HW problem. Check $C^\infty$ linearity in $Z$. Clearly we have 
 \[
   R(X,Y)(Z+W) = R(X,Y)Z + R(X,Y)W
 \]
 So it suffices to show that
 \[
   R(X,Y)(fZ) = fR(X,Y)Z
 \]
 So we have that
 \[
   \nabla_Y \nabla_X (fZ) = \nabla_Y(f\nabla_X Z + X(f)Z) = f\nabla_Y\nabla_X Z + X(f)\nabla_Y Z + Y(X(f))Z
 \]
 So therefore we have that
 \[
   \begin{aligned}
     \nabla_Y \nabla_X (fZ) - \nabla_X \nabla_Y (fZ) &= f\nabla_Y\nabla_X Z + Y(X(f))Z - f\nabla_X\nabla_YZ - X(Y(f))Z \\
                                                     &= f \left( \nabla_Y \nabla_X Z - \nabla_X \nabla_Y Z \right) + \left[ Y,X \right](f) Z
   \end{aligned}
 \]
 Also we have that
 \[
   \nabla_{[X,Y]}(fZ) = f \nabla_{[X,Y]} Z + [X,Y](f) Z
 \]
 Therefore we have that
 \[
   \begin{aligned}
     R(X,Y)(fZ) &= \nabla_Y \nabla_X fZ - \nabla_X \nabla_Y fZ + \nabla_{[X,Y]} fZ \\
                &= f(\nabla_Y \nabla_X Z - \nabla_X \nabla_Y Z + \nabla_{[X,Y]} Z) \\
                &= f R(X,Y)Z
   \end{aligned}
 \]
\end{proof}

 \begin{prop}[The Bianchi Identity]
   $R(X,Y)Z + R(Y,Z)X + R(Z,X)Y = 0$
 \end{prop}

 \begin{proof}
   By symmetry of the Levi-Civita connection. 
   \[
     \begin{aligned}
       R(X,Y)Z + R(Y,X)Z + R(Z,X)Y =& \nabla_Y \nabla_X Z - \nabla_X \nabla_Y Z + \nabla_{[X,Y]} Z \\
                                    &+ \nabla_X \nabla_Y Z - \nabla_Y \nabla_X Z + \nabla_{[Y,X]} Z\\
                                    &+ \nabla_X \nabla_Z Y - \nabla_Z \nabla_X Y + \nabla_{[Z,X]} Z\\
       =& \\
                                    \vdots \\
       =& \nabla_Y[X,Z] + \nabla_Z [Y,Z] + \nabla_X[Z,Y] - \nabla_{[Y,X]}Z  - \nabla_{[Z,Y]}X - \nabla_{[X,Z]}Y \\
       =& [Y,[X,Z]] + [Z,[Y,X]] + [X,[Z,Y]] = 0
     \end{aligned}
   \]
 \end{proof}
 We will write $(X,Y,Z,T) = \inn*{R(X,Y)Z}{T}$, for all $X,Y,Z,T \in \Gamma^{\infty}(TM)$

 \begin{prop}
   \begin{enumerate}
     \item $(X,Y,Z,T) + (Y,Z,X,T) + (Z,X,Y,T) = 0$ 
     \item $(X,Y,Z,T) = -(Y,X,Z,T)$
     \item $(X,Y,Z,T) = -(X,Y,T,Z)$
     \item $(X,Y,Z,T) = (Z,T,X,Y)$
   \end{enumerate}
 \end{prop}

 \begin{proof}
   \begin{enumerate}
     \item This is really just the Bianchi identity
     \item This follows from the definition of $R$.
     \item $(X,Y,Z,T) = -(X,Y,T,Z)$ is equivalent to sayin that that $(X,Y,Z,Z) = 0$. To prove this we just plug in $Z+T$ in the last two spots. Note that
       \[
         \begin{aligned}
           (X,Y,Z,Z) &=  \inn*{\nabla_Y\nabla_X Z - \nabla_X \nabla_Y Z + \nabla_{[X,Y]}Z}{ Z}  \\
         \end{aligned}
       \]
       By compatibility we have
       \[
         Y \inn*{\nabla_X Z}{Z} = \inn*{\nabla_Y \nabla_X}{Z} + \inn*{\nabla_X Z}{\nabla_Y Z}
       \]
       and
       \[
         \left[ X,Y \right] \inn*{Z}{Z} = 2 \inn*{\nabla_{[X,Y]}Z}{Z}
       \]
       So therefore we have
       \[
         \begin{aligned}
           (X,Y,Z,Z) &= -\inn*{\nabla_X}{\nabla_Y Z} + \inn*{\nabla_X Z}{Z} - X \inn*{\nabla_Y Z}{Z} + \inn*{\nabla_Y Z}{\nabla_X Z} + \frac{1}{2} [X,Y] \inn*{Z}{Z} \\
                     &= Y \left(\frac{1}{2} X \inn*{Z}{Z} \right) - X \left( \frac{1}{2} Y \inn*{Z}{Z} \right) + \frac{1}{2} \left[ X,Y \right] \inn*{Z}{Z} \\
                     &= \frac{1}{2} [Y,X] \inn*{Z}{Z} + \frac{1}{2} [X,Y] \inn*{Z}{Z}
                     &= 0
         \end{aligned}
       \]
     \item You use the the Bianchi identity and cyclically permute it to get 4 equations.  Add these four equations and use anti symmetry in the first and last pairs to get
       \[
         2(Z,X,Y,T) + 2(T,Y,Z,X) = 0
       \]
       so we have
       \[
         (Z,X,Y,T) = -(T,Y,Z,X) = (Y,T,Z,X)
       \]
   \end{enumerate}
  \end{proof}
\subsection*{Coordinate Descriptions for $R$}%
Let $(U, \varphi) = (U, (x^i))$ be local coordinates. Define 
\[
  R\left(\pdv{x^i},\pdv{x^j}\right) \pdv{x^k} = R_{ijk}^l \pdv{x^l}
\]
If $X = x^i \pdv{x^i}$, $Y = Y^j \pdv{x^j}$ and $Z = Z^k \pdv{x^k}$
We get that
\[
  R(X,Y)Z = R_{ijk}^l X^iY^jZ^k \pdv{x^l}
\]
We can solve for $R_{ijk}^l$ (keeping in mind $\left[ \pdv{x^i}, \pdv{x^j} \right] = 0$). Let $\nabla_s = \pdv{x^s}$
\[
  \begin{aligned}
    R_{ijk}^l \pdv{x^l} &= \nabla_j \nabla_i \pdv{x^k} - \nabla_i \nabla_j \pdv{x^k} \\
                        &= \nabla_j \left( \Gamma_{ij}^l \pdv{x^l} \right) - \nabla_i \left( \Gamma_{jk}^l \pdv{x^l} \right) \\
                        &= \Gamma_{ik}^l \nabla_j \pdv{x^j} + \pdv{x^j} \left( \Gamma_{ik}^l \right)\pdv{x^l} - \Gamma_{jk}^l \nabla_i \left( \pdv{x^l} \right) - \pdv{x^i} \left( \Gamma_{jk}^l \right) \pdv{x^l} \\
                        &= \Gamma_{ik}^l \Gamma_{jl}^s \pdv{x^s} + \pdv{x^j} \left( \Gamma_{ik}^s \right) \pdv{x^s} - \Gamma_{jk}^l \Gamma_{il}^s \pdv{x^s} - \pdv{x^i} \left( \Gamma^s_{jk} \right) \pdv{x^s}
  \end{aligned}
\]
So plugging in $f = x^s$, 
\[
  \begin{aligned}
    R_{ijk}^s\pdv{x^s}{x^l} &= R_{ijk}^l \delta_l^s = R_{ijk}^s \\
                            &= \Gamma_{ik}^l\Gamma_{jl}^s + \pdv{x^j} \Gamma_{ik}^s - \Gamma_{jk}^l \Gamma_{il}^s - \pdv{x^i} \left( \Gamma_{jk}^s \right)
  \end{aligned}
\]
Now let
\[
  \begin{aligned}
    R_{ijks} &= \inn*{R \left( \pdv{x^i}, \pdv{x^j} \right) \pdv{x^k}}{\pdv{x^s}} \\
             &= \inn*{R_{ijk}^l \pdv{x^l}}{ \pdv{x^j}} \\
             & R_{ijk}^l \inn*{\pdv{x^l}}{\pdv{x^s}} = R_{ijk}^l g_{ls}
  \end{aligned}
\]
\begin{cor}
 By a previous proposition we have
 \begin{enumerate}
   \item $R_{ijks} + R_{jkis} + R_{kijs} = 0$
   \item $R_{ijks} = -R_{jiks}$
   \item $R_{ijks} = -R_{ijsk}$ 
   \item $R_{ijks} = R_{ksij}$
 \end{enumerate} 
\end{cor}

\begin{rem}
  The above compuations all depend of $R(X,Y)Z$ being linear in each component. So we have that $R(X,Y)Z$ at $p \in M$ only depends on values of $X,Y,Z$ at $p$. We call such an object a tensor.
\end{rem}

\subsection*{Sectional Curvature}%
Sectional curvature is a measure of the curvature of 2-dimensional embedded (geodesic) surfaces. Let $V$ be a vector space, for $x,y \in V$, $| x \wedge y| = \sqrt{\norm{x}^2\norm{y}^2 - \inn*{x}{y}^2}$. This is simply the area of the parallelogram formed by $x,y$. 

\begin{prop}
  Let $\sigma \subset T_pM$ be a 2-dimensional subspace and let $x,y \in \sigma$ be linearly independent. Then
  \[
    k(x,y) =  \frac{(x,y,x,y)}{|x \wedge y|^2} = \frac{\inn*{R(x,y)x}{y}}{\norm{x}^2\norm{y}^2 - \inn*{x}{y}}
  \]
  is invariant under any definition of basis.
\end{prop}
\begin{proof}
  Clearly $k(x,y) = k(y,x)$. Also for $\lambda \in \mathds{R} \setminus \left\{ 0 \right\}$.
  \[
    k(\lambda x, y) = \frac{\inn*{R(\lambda x, y) \lambda x}{y}}{\norm{\lambda x}^2 \norm{y}^2 - \inn*{\lambda x}{y}^2} = k(x,y)
  \]
  and finally we want to show that
  \[
    k(x + y, y) = k(x,y) 
  \]
  analyzing the numerator we have
  \[
    (x + y, y, x+y, y) = (x,y, x,y) + (y,y,x,y) + (x,y,y,y) + (y,y,y,y) = (x,y,x,y)
  \]
  due to symmetry properties. Now analyzing the denominator
  \[
    \begin{aligned}
      |(x + y) \wedge y|^2 &= \norm{x + y}^2 \norm{y}^2 - \inn*{x+y}{y}^2 \\ 
                           &= \norm{y}^2 \left( \norm{x}^2 + 2 \inn*{x}{y} + \norm{y}^2 \right) - \left( \inn*{x}{y}^2 + 2 \inn*{x}{y} \norm{y}^2 + \norm{y}^4 \right) \\
                           &= \norm{x}^2\norm{y}^2 - \inn*{x}{y}^2
    \end{aligned}
  \]
  Therefore we have $k(x+y,y) = k(x,y)$. Since any two bases of $\sigma$ can be related by these transformations we have invariance under bases.
\end{proof}

\begin{defn}
  Given $p \in M$, and $\sigma \subset T_pM$ a 2-dimensional subspace of $T_pM$ the value of $k(\sigma) = k(x,y)$ for any basis $\left\{ x,y \right\}$ of $\sigma$ is called the sectional curvature of $\sigma$ at $p$.
\end{defn}

By knowing $k(\sigma)$ for all 2-dimensional subspaces of $\sigma \subset T_pM$ and all $p \in M$, we can recover the Riemannian curvature $R$. The crux of this theorem is that we know the riemannian metric.

\begin{lem}
  Let $V$ be a vector space with $\dim V \geq 2$. Let $R, R' : V^3 \rightarrow V$ which are linear in each factor, such that $(x,y,z,t) = \inn*{R(x,y)z}{t}$ and $(x,y,z,t)' = \inn*{R'(x,y)z}{t}$ satisfy properties 1-4. If for all linearly independent sets $\left\{ x,y \right\} \subset V$, $(x,y,x,y) = (x,y,x,y)'$, then $R = R'$.
\end{lem}

\begin{cor}
  $k(\sigma)$ completely determines the curvature $R$. (given a metric $g$ on $M$).
\end{cor}

\begin{proof}
  It's enough to show $(x,y,z,t) = (x,y,z,t)'$ for all $x,y,z,t \in V$. First we know that 
  \[
    \begin{aligned}
      (x+z,y, x+z,y) &= (x+z,y,x+z,y)' \\
      (x,y,x,y) + 2(x,y,z,y) + (z,y,z,y) &=(x,y,x,y)' + 2(x,y,z,y)' + (z,y,z,y)' \\
      2(x,y,z,y) &= 2(x,y,z,y)'
    \end{aligned}
  \]
  Which implies that $(x,y,z,y) = (x,y,z,y)'$. Now we have
  \[
    \begin{aligned}
      (x,y+t,z,y+t) &= (x,y+t, z, y+t)' \\
      (x,y,z,t) + (x,t,z,y) &= (x,y,z,t)' + (x,t,z,y)' \\
    \end{aligned}
  \]
  This implies 
  \[
    \begin{aligned}
      (x,y,z,t) - (x,y,z,t)'  &= (x,t,z,y)' - (x,t,z,y) \\
                              &= (z,y,x,t)' - (z,y,x,t) \\
                              &= -(y,z,x,t)' + (y,z,x,t)
    \end{aligned}
  \]
  So we have $(x,y,z,t) - (x,y,z,t)' = (y,z,x,t) - (y,z,x,t)'$ which gives us invariance under cyclic permutations in the first three factors. Therefore 
  \[
    3 \left( (x,y,z,t) - (x,y,z,t)' \right) = 0
  \]
  so we have that $(x,y,z,t) = (x,y,z,t)'$.
\end{proof}

\begin{lem}
  Let $M$ be Riemannian, and $p \in M$ define $R': T_pM \times T_pM \rightarrow T_pM$ by
  \[
    \inn*{R'(x,y,w)}{z} = \inn*{x}{w}\inn*{y}{z} - \inn*{y}{w}\inn*{x}{z}
  \]
  for all $x,y,z \in T_pM$. Then $M$ has constant sectional curvature equal to $k_0$ if and only if the Riemann curvature tensor $R = k_0 R'$.
\end{lem}

\begin{proof}
  Assume that $k(\sigma) = k_0$ for all $\sigma, p \in M$. Let 
  \[
    \inn*{R'(x,y,w)}{z} = (x,y,w,z)'
  \]
  Observe 
  \begin{enumerate}
    \item $(x,y,w,z)' + (y,w,x,z)' + (w,x,y,z)' = 0$
    \item $(x,y,w,z)' = -(y,x,w,z)' = -(x,y,z,w) = (w,z,x,y)$ 
  \end{enumerate}
  Since we have constant sectional curvature we have $k_0 = \frac{\inn*{R(x,y)x}{y}}{\norm{x}^2\norm{y}^2 - \inn*{x}{y}}$, which implies
  \[
    \inn*{R(x,y)x}{y} = k_0 (x,y,x,y)'
  \]
  so by previous lemma we have $R = k_0R'$. The converse is immediate.
\end{proof}

\begin{cor}
  Let $M$ be Riemannian and let $\left\{ e_1, \dots, e_n \right\}$ be an orthonormal basis for $T_pM$ and let $R_{ijkl} = \inn*{R(e_i,e_j) e_k}{e_l}$. Then $k(\sigma) = k_0$ for all planes $\sigma \subset T_pM$ if and only if $R_{ijkl} = k_0(\delta_{ik}\delta_{jl} - \delta_{il}\delta_{jk})$. 
\end{cor}
Thus we have
\[
  R_{ijij} = -R_{jiji} = k_0,
\]
for $i \neq j$ and $R_{ijkl} = 0$ for all other choices of indices. These statements characterize $R$ pointwise for manifolds on which $k(\sigma)$ does  not depend on $\sigma \subset T_pM$, i.e. $k(\sigma) = k_0(p)$. 

\subsection*{Ricci and Scalar Curvature}%
Let $x = z_n$ be a unit vector in $T_pM$, $(z_1, \dots, z_{n-1})$ and orthonoromal basis for the hyperplane orthogonal to $x$. Define the
\[
  \begin{aligned}
    \text{Ric}_p(x) &= \frac{1}{n-1} \sum_{i=1}^{n-1} \inn*{R(x,z_i)x}{z_i}  \\
                    &= \frac{1}{n-1} \sum_{i=1}^{n-1} k(x,z_i)
  \end{aligned}
\]
we call this the Ricci curvature. Define
\[
  \begin{aligned} 
    k(p) &= \frac{1}{n} \sum_{j=1}^n \text{Ric}_p (z_j) \\
         &= \frac{1}{n(n-1)} \sum{i,j=1}^n  \sum_{i,j=1}^n \inn*{R(z_j,z_i)z_j}{z_i} \\
         &= \frac{1}{n(n-1)} \sum_{i,j=1, i\neq j}^n k(z_i,z_j)
  \end{aligned}
\]
We note that in in $\dim M =2$ then $\text{Ric}_p(x) = k(p) = k(T_pM)$. The interpretation is the following
\[
  \text{Vol}(B_r(p)) = \left(1 - k(p) \frac{r^2}{6(n+2)} + \mathcal{O}(r^4)\right) V_{\mathds{R}^n}(B_r(0))
\]
where $V_{\mathds{R}^n}(B_r(0))$ is the volume of the ball of radius $r$ in $\mathds{R}^n$. So we have that $k(p)$ measure the growth rate of the balls in $M$ of radius $r$ compared to balls in $\mathds{R}^n$ of the same radii. The Ricci curvature is similar but measure the volume of cones in the direction of $x$.


\end{document}
