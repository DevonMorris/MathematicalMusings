%!TEX TS-program = xelatex
%!TEX encoding = UTF-8 Unicode

\documentclass[a4paper]{article}

\usepackage{xltxtra}
\usepackage{amsfonts}
\usepackage{polyglossia}
\usepackage{fancyhdr}
\usepackage{geometry}
\usepackage{dsfont}
\usepackage{amsmath}
\usepackage{amsthm}
\usepackage{amssymb}
\usepackage{physics}
\usepackage[shortlabels]{enumitem}
\usepackage{mathtools}

\geometry{a4paper,left=15mm,right=15mm,top=20mm,bottom=20mm}
\pagestyle{fancy}
\lhead{Devon Morris}
\chead{Differential Geometry - Exam 1}
\rhead{\today}
\cfoot{\thepage}

\setlength{\headheight}{23pt}
\setlength{\parindent}{0.0in}
\setlength{\parskip}{0.0in}

\DeclarePairedDelimiterX{\inn}[2]{\langle}{\rangle}{#1, #2}

\newtheorem*{prop}{Proposition}
\newtheorem*{defn}{Definition}
\newtheorem*{thm}{Theorem}
\begin{document}
\section*{Problem 1}%
Let $M$ be a smooth manifold of dimension $n$, and let $\left\{ (U){\lambda}, \varphi_{\lambda} \right\}_{\lambda \in \Lambda}$ be a collection of coordinate charts that cover $M$. Suppose that for each $\lambda \in \Lambda$, there are $n^2$ functions $g_{i,j}^\lambda: U_{\lambda} \rightarrow \mathds{R}$, where $1 \leq i,j \leq n.$ State a set of necessary and sufficient conditions on the functions $\left\{ g_{i,j}^\lambda \right\}$, so that there exists a Riemannian metric $g$ on $M$ whose coordinate description under the chart $(U_\lambda, \varphi_\lambda)$ is given by the $n^2$ functions $g_{i,j}^\lambda$.

\begin{prop}
  $g$ is a riemannian metric on $M$ if and only if $g_{i,j}^\lambda$ have the following properties
  \begin{enumerate}
    \item $g_{i,j}^\lambda \in C^{\infty}(U_\lambda)$
    \item $g_{i,j}^\lambda(p) = g_{j,i}^\lambda(p)$ for all $p  \in U_\lambda$
    \item The matrix $g^\lambda(p)$ formed by $[g^\lambda(p)]_{i,j} = g_{i,j}^\lambda(p)$, is positive definite for all $p \in U_\lambda$
    \item The $g_{i,j}^\lambda$ are related by the differential of the transition maps as follows
      \[
        g_{i,j}^{\lambda} = \pdv{\tilde{\lambda}^k}{x^i}\pdv{\tilde{\lambda}^l}{x^j} g_{k,l}^{\lambda'}
      \]
      where $\tilde{\lambda}: \lambda(U_\lambda \cap U_{\lambda'}) \rightarrow \lambda'(U_\lambda \cap U_{\lambda'})$ and $\tilde{\lambda} = \lambda' \circ \lambda^{-1}$. In other words the $g_{i,j}^\lambda$ are related by the covariant transformation of the transition maps.
  \end{enumerate}
\end{prop}

\begin{proof}
  First let us suppose that $g$ is a riemannian metric. 
  \begin{enumerate}
    \item By definition we have that $g_{i,j} \in C^{\infty}(U_\lambda)$
    \item Since $g(p)$ is symmetric, we have that 
      \[
        g_{i,j}(p) = \inn*{\eval{\pdv{}{x^i}}_p}{\eval{\pdv{}{x^j}}_p} = \inn*{\eval{\pdv{}{x^j}}_p}{\eval{\pdv{}{x^i}}_p} = g_{j,i}(p)
      \]
    \item Assume to the contrary that $[g^\lambda(p)]$ is not positive definite for some $p \in U_\lambda$, then there exists an element, $X \in T_pM$, such that $X \neq 0$ and its coordinate representation $x$ satisfies
      \[
        \inn*{X}{X} = x^ig_{i,j}(p)x^j \leq 0.
      \]
      This violates the positive-definiteness of the inner product, therefore we must have that $[g^\lambda(p)]$ is positive definite.
    \item Let $(U_\lambda, \lambda) = (U_\lambda, (x^j))$ and $(U_{\lambda'}, \lambda') = (U_{\lambda'}, (y^j))$, be coordinate charts with nontrivial intersection. Consider the identity transformation on the manifold $\text{Id}_M$, the differential of this transformation is also the identity transformation on $T_pM$. Then we must have that
      \[
        \pdv{x^j} = d(Id_M) \left( \pdv{x^j} \right) = \pdv{\tilde{\lambda^i}}{x^j} \pdv{y^i}
      \]
      Thus, we have that
      \[
        \begin{aligned}
          g_{ij}^\lambda(p) &= \inn*{\pdv{x^i}}{\pdv{x^j}} \\
                            &= \inn*{\pdv{\tilde{\lambda}^k}{x^i}\pdv{y^k}}{\pdv{\tilde{\lambda}^l}{x^j}\pdv{y^l}} \\
                            &= \pdv{\tilde{\lambda}^k}{x^i}\pdv{\tilde{\lambda}^l}{x^j} \inn*{\pdv{y^k}}{\pdv{y^l}} \\
                            &=  \pdv{\tilde{\lambda}^k}{x^i}\pdv{\tilde{\lambda}^l}{x^j} g_{k,l}^{\lambda'}
        \end{aligned}
      \]
  \end{enumerate}

  Now we will assume that the conditions on $g_{i,j}$ hold. For a coordinate chart $(U_{\lambda}, \lambda) = (U_{\lambda}, (x^j))$, and vectors $X, Y, Z \in T_pM$,
  \[
    X = u^i \pdv{x_i} \quad Y = v^i \pdv{x_i} \quad Z = w^i \pdv{x_i}
  \]
  we define
  \[
    \inn*{X}{Y} = g_{i,j}^{\lambda}(p)u^iv^j.
  \]
  Thus, it suffices to show that this operation is a well-defined inner product on $T_pM$. We will first check the conditions of an inner product. Let $\alpha, \beta \in \mathds{R}$, we have that
  \begin{enumerate}
    \item Using the symmetry of $g_{i,j}^\lambda$, we have that
      \[
        \inn*{X}{Y} = g_{i,j}^\lambda(p) u^iv^j = g_{j,i}^\lambda(p) v^j u^i = \inn*{Y}{X}.
      \]
    \item For scaling, we have
      \[
        \inn*{\alpha X}{Y} = g_{i,j}^\lambda(p) (\alpha u^i)v^j = \alpha g_{i,j}^\lambda(p) u^iv^j = \alpha \inn*{X}{Y}
      \]
    \item For linearity, we have
      \[
        \inn*{X + Y}{Z} = g_{i,j}^\lambda(p) (u^i + v^i)w^j = g_{i,j}^\lambda(p) u^iw^j + g_{i,j}^\lambda(p) v^iw^j = \inn*{X}{Z} + \inn*{Y}{Z}
      \]
    \item For positive-definiteness, we have
      \[
        \inn*{X}{X} = g_{i,j}^\lambda(p) u^iu^j \geq 0
      \]
      since $[g(p)]_{i,j}$ is positive definite. Furthermore since $[g(p)]_{i,j}$ is strictly positive definite we have equality if and only if $X = 0$.
  \end{enumerate}
  Now we know that we have a coordinate representation under this chart that acts like an inner product. However, we have made a choice among different coordinates about the same point, so we must show that it is well defined under different coordinates. Let $(U_{\lambda'}, \lambda') = (U_{\lambda'}, (y^j))$ so that
  \[
    X = u^i \pdv{x^i} = a^i \pdv{y^i} \quad Y = v^i \pdv{x^i} = b^i \pdv{y^i}
  \]
  Let us consider
  \[
    \begin{aligned}
      g_{k,l}^{\lambda'}(p) a^k b^l
    \end{aligned}
  \]
  We note that by the same logic as above $a^k = u^i \pdv{\tilde{\lambda}^k}{x^i}$ and $b^l = v^j \pdv{\tilde{\lambda}^l}{x^j}$, so we have that
  \[
    \begin{aligned}
      g_{k,l}^{\lambda'}(p) a^k b^l &= g_{k,l}^{\lambda'}(p) u^i v^j \pdv{\tilde{\lambda}^k}{x^i} \pdv{\tilde{\lambda}^l}{x^j} \\
                                    &= g_{i,j}^\lambda(p) u^i v^j
    \end{aligned}
  \]
  So we have that our inner product is well-defined. Therefore we have found necessary and sufficient conditions for $g_{i,j}$ to define an inner product on the manifold.
    
  
\end{proof}

\section*{Problem 2}%
Let $M$ be a smooth manifold, and $S \subset M$. Prove that $S$ is an embedded manifold of $M$ if and only if $S$ the image of a smooth embedding $\iota: S_0 \rightarrow M$.

\begin{proof}
  Suppose that $S$ is an embedded submanifold of $M$ of dimension $k$. Thus, for every $p \in M$, there exists a $k$-slice chart $(U,\varphi)$ for $S$. So we have that 
  \[
    \varphi(U \cap S) = \left\{ \left( x^1, \dots, x^k, 0, \dots, 0 \right) \right\}
  \]
  We will now define a new map
  \[
    \begin{aligned}
      \psi &= \pi \circ \varphi: U \cap S \rightarrow \mathds{R}^k \\
           &p \mapsto (x^1, \dots, x^k)
    \end{aligned}
  \]
  where we are essentially just projecting onto the the first $k$ coordinates. Now we want to see if these $U \cap S$ form a smooth structure on $S$. We note these necessarily cover $S$ and that since they are the composition of smooth maps, we must have that the coordinate charts are compatible. Now, we note that in coordinates we can create a map $\iota: S \rightarrow M$, with a coordinate representation 
  \[
    \varphi \circ \iota \circ \psi^{-1}(x^1, \dots, x^k) = (x^1, \dots, x^k, 0, \dots, 0)
  \]
  By inspection this is obviously a smooth immersion. Furthermore, since the projection is the inverse operation we have that $\iota$ is homeomorphic onto its image. Therefore, we have a smooth embedding.


  Now suppose that $S$ is the image of some smooth embedding $\iota: S_0 \rightarrow M$. It suffices to show that for each $p \in \iota(S_0)$, there is a chart $(U, \varphi)$ containing $\iota(p)$ such that $\iota(S_0) \cap U$ is a $k$-slice. We note that $\iota$ must have constant rank. By the rank theorem there exist charts $(V, \psi)$ about $p$ and $(U,\varphi)$ about $\iota(p)$ such that $\iota(U) \subset V$ and 
  \[
    \varphi \circ \iota \circ \psi^{-1}(x^1, \dots, x^k) = (x^1, \dots, x^k, 0, 0, \dots, 0)
  \]
  So we essentially have a slice about $\iota(V) \cap U$. Since $\iota$ is a homeomorphism and has a smooth inverse, there must be some $W$ such that $\iota(V) = W \cap \iota(S_0)$. Thus let $U' = U \cap W$ and restrict $\varphi$ to $U'$. Thus we have constructed a $k$-slice and therefore $S = \iota(S_0)$ is an embedded submanifold.
\end{proof}

\section*{Problem 3 (Version 2)}%

Define the torus $\mathds{T}^2$ as the quotient
\[
  \mathds{T}^2 = \mathds{R}^2 / 2\pi \mathds{Z}^2 = \mathds{R}^2 / \left\{ (x,y) \sim (x+2\pi n, y+2\pi m)\text{ for } m,n \in \mathds{Z} \right\}
\]
Notice that any open square of the form $(a, a+2\pi) \times (b, b+2\pi) \subset \mathds{R}^2$ induces a coordinate chart $(U_{a,b}, \varphi_{a,b})$ on $\mathds{T}^2$ in a natural way.

\begin{enumerate}[(a)]
  \item Show that the standard metric on $\mathds{R}^2$ induces a metric on $\mathds{T}^2$. Compute the coordinate representations $g_{i,j}$ of this metric on $\mathds{T}^2$ in the chart $(U_{a,b}, \varphi_{a,b})$, as well as the Christoffel symbols of the associated Levi-Civita connection.
  \item Notice that the map $\psi: \mathds{R}^2 \rightarrow \mathds{R}^4$ defined by 
    \[
      \psi(\alpha,\beta) = (\cos(\alpha), \sin(\alpha), \cos(\beta), \sin(\beta))
    \]
    induces a well-defined map $\tilde{\psi}: \mathds{T}^2 \rightarrow \mathds{R}^4$, which is a diffeomerophism onto its image $T_1 = \tilde{\psi}(\mathds{T}^2)$. Equip $T_1$ with the metric induced by the standard metric on $\mathds{R}^4$. Show that $\tilde{\psi}: \mathds{T}^2 \rightarrow T_1$ is an isometry. (You do not need to show that it is a diffeomorphism).
  \item Notice that the map $\omega: \mathds{R}^2 \rightarrow \mathds{R}^3$ defined by
    \[
      \omega(\alpha,\beta) = ((\cos(\beta) + 4)\cos(\alpha), (\cos(\beta) + 4)\sin(\alpha) , \sin(\beta))
    \]
    induces a well defined map $\tilde{\omega}: \mathds{T}^2 \rightarrow \mathds{R}^3$, which is a diffeomorphism onto its image $T_2 = \tilde{\omega}(\mathds{T}^2)$. Equip $T_2$ with the metric induced by the standard metric on $\mathds{R}^3$.

    If $S$ is the open square $(0, 2\pi) \times (0, 2\pi)$ and $S' = \omega(S) \subset T_2$, then the restriction of $\omega^{-1}$ to $S'$ is a coordinate chart $\omega^{-1}: S' \rightarrow \mathds{R}^2$ on $T_2$. Compute the coordinate representation of the metric on $T_2$ in this chart, as well as the associated Christoffel symbols.
\end{enumerate}

\begin{proof}[Solution]
  \begin{enumerate}[(a)]
    \item  First we note that $\mathds{T}^2$ contains elements of the form $[(a,b)]$. Our charts $(U_{a,b}, \varphi_{a,b})$, thus choose the representative that lies in the region $(a,a + 2\pi) \times (b, b + 2\pi)$, if such one exists. Note, for example if $a = b = 0$ then $[(0,0)] \notin U_{0,0}$. These charts are easily characterized by looking at their inverse mappings
      \[
        \begin{aligned}
          \varphi_{a,b}^{-1}:& (a, a+ 2\pi) \times (b, b+2\pi) \rightarrow \mathds{T}^2 \\
          &(y^1,y^2) \mapsto [(y^1,y^2)] 
        \end{aligned}
      \]
      Let $\left\{ \pdv{y^j} \right\}$ be the basis for $T_{[(y^1, y^2)]}\mathds{T}^2$. We wish to investigate the coordinates of the differential $d \varphi_{a,b}$ where we are using the identity chart on $T_{(y^1 + 2\pi m, y^2 + 2\pi n)} \mathds{R}^2 \cong \mathds{R}^2$ (i.e. $\text{Id}((x^1, x^2)) = (x^1, x^2)$. So in coordinates we have
      \[
        \hat{\varphi}_{a,b}(y^1, y^2) = \text{Id} \circ \varphi_{a,b} \circ \varphi_{a,b}^{-1} (y^1, y^2) = (y^1, y^2)
      \]
      So for $v = v^i \pdv{y^i} \in  T_{[(y^1, y^2)]}\mathds{T}^2$. We have that the vector under the differential simply is
      \[
        d\varphi_{a,b} \left( \eval{\pdv{y^i}}_{[(y^1, y^2)]} \right) = \eval{\pdv{x^i}}_{(y^1 + 2\pi m, y^2 + 2\pi n)}
      \]
      Thus we can use the metric on $T_{(x^1, x^2)}\mathds{R}^2$. Specifically, we have
      \[
        \inn*{v}{w}^{\mathds{T}^2}_{[(y^1, y^2)]} = \inn*{d\varphi_{a,b}(v)}{d\varphi_{a,b}(w)}_{(y^1 + 2\pi m, y^2 + 2\pi n)}^{\mathds{R}^2}
      \]
      Now the question arises of whether under a different coordinate chart this is well-defined. The answer is yes, since the metric on $\mathds{R}^2$ is translation invariant. For a coordinate chart $(U_{c,d}, (z^i))$, where $[(z^1, z^2)] = [(y^1, y^2)]$, the translation invariance of the metric on $\mathds{R}^2$, gives us
      \[
        \begin{aligned}
          \inn{v}{w}^{\mathds{T}^2}_{[(z^1, z^2)]} &= \inn*{d \varphi_{c,d}(v)}{d \varphi_{c,d}(w)}_{(z^1 + 2\pi m', z^2 + 2\pi n')}^{\mathds{R}^2} \\
                                                   &= \inn*{ v^i \eval{\pdv{x^i}}_{(z^1 + 2\pi m', z^2 + 2\pi n')}}{w^j \eval{\pdv{x^i}}_{(z^1 + 2\pi m', z^2 + 2\pi n')}}_{(z^1 + 2\pi m', z^2 + 2\pi n')}^{\mathds{R}^2}  \\
                                                   &= \inn*{ v^i \eval{\pdv{x^i}}_{(y^1 + 2\pi m, y^2 + 2\pi n)}}{w^j \eval{\pdv{x^i}}_{(y^1 + 2\pi m, y^2 + 2\pi n)}}_{(y^1 + 2\pi m, y^2 + 2\pi n)}^{\mathds{R}^2}  \\
                                                   &= \inn*{d\varphi_{a,b}(v)}{d\varphi_{a,b}(w)}_{(y^1 + 2\pi m, y^2 + 2\pi n)}^{\mathds{R}^2} \\
                                                   &= \inn{v}{w}^{\mathds{T}^2}_{[(y^1, y^2)]}
        \end{aligned}
      \]
      Therefore, this induced metric is well defined. Fortunately, the components of the metric $g_{i,j}$ are very easy to compute
      \[
        g_{i,j} = \inn*{\pdv{y^i}}{\pdv{y^j}}^{\mathds{T}^2} = \inn*{\pdv{x^i}}{\pdv{x^j}}^{\mathds{R}^2} = \delta_{i,j}
      \]
      We further note that
      \[
        g^{i,j} = \delta^{i,j}
      \]
      Computing the Christoffel symbols of the Levi-Civita connection gives
      \[
        \Gamma^{k}_{i,j} = 0
      \]
      for all $i,j,k \in {1,2}$, since we essentially have the same metric as $\mathds{R}^2$.

    \item We note that this map $\tilde{\psi}: \mathds{T}^2 \rightarrow \mathds{R}^4$ is given by
      \[
        \tilde{\psi}\left([(\alpha,\beta)]\right) = \psi(\alpha,\beta)
      \]
      We note that this map is well defined since $\sin$ and $\cos$ are $2\pi$ periodic and any other representative of the same class will give the same output. At this point we note that $\tilde{\psi}(\mathds{T}^2) = T_1 \subset \mathds{R}^4$, so our immersion from $T_1$ to $\mathds{R}^4$ can be simply given as the identity map. This induces the following metric on $T_1$
      \[
        \inn*{v}{w}_p^{T_1} = \inn*{d \text{Id}(v)}{d \text{Id}(w)}_{\text{Id}(p)}^{\mathds{R}^4} = \inn*{v}{w}_p^{\mathds{R}^4}
      \]
      Since we know that $\tilde{\psi}$ is a diffeomorphism it suffices to show that
      \[
        \inn*{v}{w}_{[(a,b)]}^{\mathds{T}^2} = \inn*{d\tilde{\psi}(v)}{d\tilde{\psi}(w)}_{\tilde{\psi}([(a,b)])}^{T_1}
      \]
      for all $v,w \in T_{[(a,b)]}\mathds{T}^2$. It suffices to show that this relation holds for all combinations of $v,w \in \left\{ \pdv{\alpha}, \pdv{\beta} \right\}$ since we can leverage the linearity of the differential and the inner product. Let us first analyze the differential $d\tilde{\psi}$. In local coordinates $(U_{a,b}, (\alpha, \beta))$ about $[(\alpha, \beta)]$ and the standard coordinates $(y^i)$ of $\mathds{R}^4$ for $T_1$, we have
      \[
        \begin{aligned}
          d \tilde{\psi} \left( \pdv{\alpha} \right) &= -\sin(\alpha)\pdv{y^1} + \cos(\alpha) \pdv{y^2} \\
          d \tilde{\psi} \left( \pdv{\beta} \right) &= -\sin(\beta)\pdv{y^3} + \cos(\beta) \pdv{y^4}
        \end{aligned}
      \]
      Using the metric derived in part $(a)$, we have that
      \[
        \begin{aligned}
          \inn*{\pdv{\alpha}}{\pdv{\alpha}}_{[(\alpha,\beta)]}^{\mathds{T}^2} &= 1 \\
                                                                              &= (-\sin(\alpha))^2 + \cos(\beta)^2 + 0 \\
                                                                              &= (-\sin(\alpha))^2\inn*{\pdv{y^1}}{\pdv{y^1}}^{T_1} + \cos(\beta)^2 \inn*{\pdv{y^2}}{\pdv{y^2}}^{T_1} - 2\sin{\alpha}\cos{\beta} \inn*{\pdv{y^1}}{\pdv{y^2}}^{T_1} \\
                                                                              &= \inn*{-\sin(\alpha)\pdv{y^1}}{-\sin(\alpha) \pdv{y^1}}^{T_1} + \inn*{\cos(\beta)\pdv{y^2}}{\cos(\beta)\pdv{y^2}}^{T_1} + 2\inn*{-\sin(\alpha) \pdv{y^1}}{\cos(\beta)\pdv{y^2}}^{T_1} \\
                                                                              &= \inn*{-\sin(\alpha) \pdv{y^1} + \cos(\beta) \pdv{y^2}}{-\sin(\alpha) \pdv{y^1} + \cos(\beta) \pdv{y^2}}^{T_1} \\
                                                                              &= \inn*{d \tilde{\psi} \left( \pdv{\alpha} \right)}{d \tilde{\psi}\left( \pdv{\alpha}\right)}^{T_1}
        \end{aligned}
      \]
      Similarly we have that
      \[
        \begin{aligned}
          \inn*{\pdv{\beta}}{\pdv{\beta}}_{[(\alpha,\beta)]}^{\mathds{T}^2} &= 1 \\
                                                                              &= (-\sin(\beta))^2 + \cos(\beta)^2 + 0 \\
                                                                              &= (-\sin(\beta))^2\inn*{\pdv{y^3}}{\pdv{y^3}}^{T_1} + \cos(\beta)^2 \inn*{\pdv{y^4}}{\pdv{y^4}}^{T_1} - 2\sin{\beta}\cos{\beta} \inn*{\pdv{y^3}}{\pdv{y^4}}^{T_1} \\
                                                                              &= \inn*{-\sin(\beta)\pdv{y^3}}{-\sin(\beta) \pdv{y^3}}^{T_1} + \inn*{\cos(\beta)\pdv{y^4}}{\cos(\beta)\pdv{y^4}}^{T_1} + 2\inn*{-\sin(\beta) \pdv{y^3}}{\cos(\beta)\pdv{y^4}}^{T_1} \\
                                                                              &= \inn*{-\sin(\beta) \pdv{y^3} + \cos(\beta) \pdv{y^4}}{-\sin(\beta) \pdv{y^3} + \cos(\beta) \pdv{y^4}}^{T_1} \\
                                                                              &= \inn*{d \tilde{\psi} \left( \pdv{\beta} \right)}{d \tilde{\psi}\left( \pdv{\beta}\right)}^{T_1}
        \end{aligned}
      \]
      and lastly
      \[
        \begin{aligned}
          \inn*{\pdv{\alpha}}{\pdv{\beta}}_{[(\alpha, \beta)]}^{\mathds{T}^2} &= 0 \\
                                                                              &= \inn*{ -\sin(\alpha)\pdv{y^1} + \cos(\beta)\pdv{y^2}}{-\sin(\beta)\pdv{y^3} + \cos(\beta)\pdv{y^4}}^{T_1} \\
                                                                              &= \inn*{d \tilde{\psi} \left( \pdv{\alpha} \right)}{d \tilde{\psi}\left( \pdv{\beta}\right)}^{T_1}
        \end{aligned}
      \]
      Since $ \text{span}\left\{ \pdv{y^1}, \pdv{y^2} \right\} \perp \text{span} \left\{ \pdv{y^3}, \pdv{y^4} \right\}$. Therefore we have shown that $\tilde{\psi}$ is an isometry.
    \item We note that this map $\tilde{\omega}: \mathds{T}^2 \rightarrow \mathds{R}^4$ is given by
      \[
        \tilde{\omega} \left( [(\alpha, \beta)] \right) =  \omega(\alpha, \beta).
      \]
      This map is well defined because $\sin$ and $\cos$ are $2\pi$ periodic and any other representative of this same class will give the same output. At this point we note that $\tilde{\omega}(\mathds{T}^2) = T_2 \subset \mathds{R}^3$, so our immersion can be simply given by the identity map. This mapping induces the following metric on $T_2$
      \[
        \inn*{v}{w}_p^{T_2} = \inn*{d \text{Id}(v)}{d \text{Id}(w)}_{\text{Id}(p)}^{\mathds{R}^3} = \inn*{v}{w}_p^{\mathds{R}^3}
      \]
      Now we can define a coordinate chart on $S' = \omega(S) \subset T_2$, given by $\omega^{-1}: S' \rightarrow \mathds{R}^2$. Using the standard coordinates $(y^i)$ on $\mathds{R}^3$ and the coordinates given by $\omega^{-1}$ the coordinate representation of the identity map is
      \[
        \hat{\text{Id}}(\alpha, \beta) = (\text{Id} \circ (\omega^{-1})^{-1})(\alpha,\beta) = \omega(\alpha,\beta)
      \]
      Now, we will analyze the differential of this map in coordinates
      \[
        \begin{aligned}
          d \text{Id} \left( \pdv{\alpha} \right) &= -(\cos(\beta) + 4)\sin(\alpha)\pdv{y^1} + (\cos(\beta) + 4)\cos(\alpha) \pdv{y^2} \\
          d \text{Id} \left( \pdv{\beta} \right)                  &= -\sin(\beta)\cos(\alpha)\pdv{y^1} - \sin(\beta)\sin(\alpha) \pdv{y^2} + \cos(\beta) \pdv{y^3}
        \end{aligned}
      \]
      Now we can find the components $g_{i,j}$, where $1$ denotes $\alpha$ and $2$ denotes $\beta$. Thus, we have
      \[
        \begin{aligned}
          g_{11} &= \inn*{\pdv{\alpha}}{\pdv{\alpha}}^{T_2} \\
                 &= \inn*{-(\cos(\beta) + 4)\sin(\alpha)\pdv{y^1} + (\cos(\beta) + 4)\cos(\alpha) \pdv{y^2}}{ -(\cos(\beta) + 4)\sin(\alpha)\pdv{y^1} + (\cos(\beta) + 4)\cos(\alpha) \pdv{y^2}} \\
                 &= (\cos(\beta) + 4)^2\sin(\alpha)^2  + (\cos(\beta) + 4)^2\cos(\alpha)^2 \\
                 &= (\cos(\beta) + 4)^2
        \end{aligned}
      \]
      \[
        \begin{aligned}
        g_{22} &= \inn*{\pdv{\beta}}{\pdv{\beta}}^{T_2} \\
               &= \inn*{-\sin(\beta)\cos(\alpha)\pdv{y^1} - \sin(\beta)\sin(\alpha) \pdv{y^2} + \cos(\beta) \pdv{y^3}}{-\sin(\beta)\cos(\alpha)\pdv{y^1} - \sin(\beta)\sin(\alpha) \pdv{y^2} + \cos(\beta) \pdv{y^3}} \\
               &= \sin(\beta)^2\cos(\alpha)^2 + \sin(\beta)^2\sin(\alpha)^2 + \cos(\beta)^2 \\
               &= 1
        \end{aligned}
      \]
      \[
        \begin{aligned}
          g_{12} &= \inn*{-(\cos(\beta) + 4)\sin(\alpha)\pdv{y^1} + (\cos(\beta) + 4)\cos(\alpha) \pdv{y^2}}{-\sin(\beta)\cos(\alpha)\pdv{y^1} - \sin(\beta)\sin(\alpha) \pdv{y^2} + \cos(\beta) \pdv{y^3}} \\
                 &= (\cos(\beta) + 4)\sin(\alpha)\sin(\beta)\cos(\alpha) - (\cos(\beta) + 4)\cos(\alpha)\sin(\beta)\sin(\alpha) \\
                 &= 0 = g_{21}
        \end{aligned}
      \]
      So we have the matrix
      \[
        [g]_{ij} = 
        \begin{bmatrix}
          (\cos(\beta) + 4)^2 & 0 \\
          0 & 1
        \end{bmatrix}
      \]
      which has an inverse of
      \[
        [g]^{ij} = 
        \begin{bmatrix}
          (\cos(\beta) + 4)^{-2} & 0 \\
          0 & 1
        \end{bmatrix}
      \]
      Now (assuming we are using the Levi-Civita connection) we can calculate our christoffel symbols as
      \[
        \Gamma_{kl}^i  = \frac{1}{2}g^{im} \left(\pdv{g_{mk}}{x^l} + \pdv{g_{ml}}{x^k} - \pdv{g_{kl}}{x^m} \right)
      \]
      Note due to the symmetry under the Levi-Civita connection, we only have to compute $\Gamma_{11}^1, \Gamma_{12}^1, \Gamma_{22}^1, \Gamma_{11}^2, \Gamma_{12}^2, \Gamma_{22}^2$, since $\Gamma_{ij}^k = \Gamma_{ji}^k$. So we have
      \[
        \begin{aligned}
        \Gamma_{11}^1 &= \frac{1}{2} g^{1,m} \left( \pdv{g_{m,1}}{\alpha} + \pdv{g_{m,1}}{\alpha} - \pdv{g_{11}}{x^m} \right) \\
                      &= \frac{1}{2} \left(g^{1,1}\pdv{g_{1,1}}{\alpha} + g^{1,2}\pdv{g_{2,1}}{\alpha} + g^{1,1}\pdv{g_{1,1}}{\alpha} + g^{1,2}\pdv{g_{1,2}}{\alpha} - g^{1,1}\pdv{g_{1,1}}{\alpha} - g^{1,2}\pdv{g_{1,1}}{\beta} \right) \\
                      &= 0 \\
        \Gamma_{12}^1 &= \frac{1}{2} g^{1,m} \left( \pdv{g_{m,1}}{\beta} + \pdv{g_{m,2}}{\alpha} - \pdv{g_{1,2}}{x^m}\right) \\
                      &= \frac{1}{2} \left(g^{1,1} \pdv{g_{1,1}}{\beta} + g^{1,2}\pdv{g_{2,1}}{\beta} + g^{1,1} \pdv{g_{2,1}}{\alpha} + g^{1,2} \pdv{g_{2,2}}{\alpha} - g^{1,1}\pdv{g_{1,2}}{\alpha} - g^{1,2}\pdv{g_{1,2}}{\beta} \right) \\
                      &= \frac{1}{2} \left( -2(\cos(\beta) + 4)^{-2}(\cos(\beta) + 4)\sin(\beta) \right) \\
                      &= - \sin(\beta)(\cos(\beta) + 4)^{-1} \\
        \Gamma_{22}^1 &= \frac{1}{2} g^{1,m} \left( \pdv{g_{m,2}}{\beta} + \pdv{g_{m,2}}{\beta} - \pdv{g_{2,2}}{x^m} \right) \\
                      &= \frac{1}{2} \left( g^{1,1}\pdv{g_{1,2}}{\beta} + g^{1,2}\pdv{g_{2,2}}{\beta} + g^{1,1}\pdv{g_{1,2}}{\beta} + g^{1,2}\pdv{g_{2,2}}{\beta} - g^{1,1}\pdv{g_{2,2}}{\alpha} - g^{1,2}\pdv{g_{2,2}}{\beta} \right) \\
                      &= 0 \\
        \Gamma_{11}^2 &= \frac{1}{2} g^{2,m} \left( \pdv{g_{m,1}}{\alpha} + \pdv{g_{m,1}}{\alpha} - \pdv{g_{1,1}}{x^m}\right) \\
                      &=  \frac{1}{2} \left( g^{2,1} \pdv{g_{1,1}}{\alpha} + g^{2,2} \pdv{g_{2,1}}{\alpha} + g^{2,1}\pdv{g_{1,1}}{\alpha} + g^{2,2} \pdv{g_{2,1}}{\alpha} - g^{2,1}\pdv{g_{1,1}}{\alpha} - g^{2,2} \pdv{g_{1,1}}{\beta} \right) \\
                      &= \frac{1}{2}(-2(\cos(\beta)+4)\sin(\beta)) \\
                      &= -\sin(\beta)(\cos(\beta) + 4) \\
        \Gamma_{12}^2 &= \frac{1}{2} g^{2,m}\left(\pdv{g_{m,1}}{\beta} + \pdv{g_{m,2}}{\alpha} - \pdv{g_{1,2}}{x^m} \right) \\
                      &= \frac{1}{2} \left( g^{2,1}\pdv{g_{1,1}}{\beta} + g^{2,2}\pdv{g_{2,1}}{\beta} + g^{2,1} \pdv{g_{2,1}}{\alpha} + g^{2,2} \pdv{g_{2,2}}{\alpha} - g^{2,1} \pdv{g_{1,2}}{\alpha} - g^{2,2} \pdv{g_{1,2}}{\beta} \right) \\
                      &=  0 \\
      \Gamma_{22}^2 &= \frac{1}{2} g^{2,m} \left( \pdv{g_{m,2}}{\beta} + \pdv{g_{m,2}}{\beta} - \pdv{g_{2,2}}{x^m} \right) \\
                    &= \frac{1}{2} \left( g^{2,1} \pdv{g_{1,2}}{\beta} + g^{2,2}\pdv{g_{2,2}}{\beta} + g^{2,1} \pdv{g_{1,2}}{\beta} + g^{2,2} \pdv{g_{2,2}}{\beta} - g^{2,1} \pdv{g_{2,2}}{\alpha} - g^{2,2} \pdv{g_{2,2}}{\beta}  \right) \\
                    &=  0
        \end{aligned}
      \]
  \end{enumerate}
\end{proof}

\section*{Problem 4}%

Let $(\tilde{M}, \tilde{g})$ be a Riemannian manifold of dimension $m$, with Levi-Civita connection $\tilde{\nabla}$. Let $M$ be a $n$-dimensional submanifold of $\tilde{M}$, and let $g$ be the metric on $M$ induced by $\tilde{g}$.

\medskip

For each $p \in M$, the tangent space $T_pM$ is an $n$-dimensional subspace of $T_p\tilde{M}$, and we let $\text{pr}: T_p \tilde{M} \rightarrow T_pM$ be the orthogonal projection.
\begin{enumerate}[(a)]
  \item For any $X,Y \in \Gamma^{\infty}(TM)$, let $\tilde{X}$ and $\tilde{Y}$, denote extensions of $X$ and $Y$ to the manifold $\tilde{M}$. Show that the map 
    \[
      \nabla: \Gamma^{\infty}(TM) \times \Gamma^{\infty}(TM) \rightarrow \Gamma^{\infty}(TM)
    \]
    defined by $\nabla_X Y = \text{pr}(\tilde{\nabla}_{\tilde{X}}\tilde{Y})$ is a connection on $M$.
  \item Show that $\nabla$ is in fact the Levi-Civita connection on $(M,g)$.
  \item Use the above fact to prove that the great circles of $S^2 \subset \mathds{R}^3$, with the induced metric and Levi-Civita connection, are geodesics.
\end{enumerate}

\begin{proof}
  \begin{enumerate}[(a)]
    \item Let $X,Y,Z \in \Gamma^{\infty}(TM)$ and let $f,g \in C^{\infty}(M)$. Let $\tilde{X}, \tilde{Y}, \tilde{Z} \in \Gamma^{\infty}(T\tilde{M})$ be extensions of $X,Y,Z$. By this, we mean that $\tilde{X}_p = X_p$ for all $p \in M$. Furthermore, let $\tilde{f}, \tilde{g} \in C^{\infty}(\tilde{M})$, be smooth extensions of $f,g$, i.e. that $\eval{\tilde{f}}_M = f$. We note that $\tilde{f}\tilde{X}$ and $\tilde{g}\tilde{Y}$ are smooth extensions of $fX$ and $gY$, and $\tilde{f}\tilde{X} + \tilde{g}\tilde{Y}$ is a smooth extension of $fX + gY$, from the linearity of $\text{pr}$. Now we will check the properties of an affine connection.
      \begin{enumerate}[1.]
        \item Consider $\nabla_{fX + gY}Z$
          \[
            \begin{aligned}
              \nabla_{fX + gY}Z &= \text{pr} \left( \tilde{\nabla}_{\tilde{f}\tilde{X} + \tilde{g}\tilde{Y}}\tilde{Z}\right) \\
                                &= \text{pr} \left( \tilde{f} \tilde{\nabla}_{\tilde{X}}\tilde{Z} + \tilde{g} \tilde{\nabla}_{\tilde{Y}}\tilde{Z}  \right)
            \end{aligned}
          \]
          At this point we note that $\text{pr}$ is a linear operator and $\text{pr} \left( \tilde{f} \tilde{X} \right) = f \text{pr} \left( \tilde{X} \right)$, since $\tilde{f}(p) = f(p)$ by construction. So we have
          \[
            \begin{aligned}
              \nabla_{fX + gY}Z &= f \text{pr} \left( \tilde{\nabla}_{\tilde{X}}\tilde{Z} \right) + g\text{pr} \left( \tilde{\nabla}_{\tilde{Y}}\tilde{Z} \right) \\
                                &= f \nabla_{X}Z + g \nabla_{Y}Z
            \end{aligned}
          \]
        \item  Now consider $\nabla_{X}(Y+Z)$
          \[
            \begin{aligned}
              \nabla_{X}(Y+Z) &= \text{pr} \left( \tilde{\nabla}_{\tilde{X}} (\tilde{Y} + \tilde{Z}) \right) \\
                              &= \text{pr} \left( \tilde{\nabla}_{\tilde{X}} \tilde{Y} + \tilde{\nabla}_{\tilde{X}} \tilde{Z} \right) \\
                              &= \text{pr} \left(\tilde{\nabla}_{\tilde{X}} \tilde{Y}\right) + \text{pr} \left( \tilde{\nabla}_{\tilde{X}} \tilde{Z}\right) \\
                              &= \nabla_X Y + \nabla_X Z
            \end{aligned}
          \]
          again using linearity of $\text{pr}$.
        \item Lastly, consider $\nabla_X (fY)$
          \[
            \begin{aligned}
            \nabla_X (fY) &= \text{pr} \left( \tilde{\nabla}_{\tilde{X}} \tilde{f}\tilde{Y} \right) \\
                          &= \text{pr} \left( \tilde{f} \tilde{\nabla}_{\tilde{X}} \tilde{Y} +  \tilde{X}(f)\tilde{Y}\right)\\
                          &= f\text{pr} \left(\tilde{\nabla}_{\tilde{X}} \tilde{Y}\right) + \tilde{X}(\tilde{f})\text{pr} \left(\tilde{Y}\right) \\
                          &= f \nabla_X Y + \tilde{X}(\tilde{f})Y 
            \end{aligned}
          \]
          At this point we note that $\tilde{X}$ and $\tilde{f}$ have to agree with $X$ and $f$ on a small neighborhood of $p$ or they would not be smooth extensions. Thus we have that $\tilde{X}(\tilde{f}) = X(f)$.
      \end{enumerate}
      Lastly, We must determine if $\nabla$ is well-defined, since we could choose a different $\tilde{X}'$ such that $\tilde{X} \neq \tilde{X}'$. However, we know that in the $X$ argument the connection only depends on $X_p$. Furthermore in the $Y$ argument, the connection depends on $Y$ in a neighborhood of $p$. However for two extensions of $Y$, $\tilde{Y} \neq \tilde{Y}'$, they must agree on some neighborhood of $p$ or they wouldn't be smooth extensions. Therefore, $\nabla$ is well-defined.
    \item Let $X,Y,Z \in \Gamma^{\infty}(TM)$. Consider $g \left(Z, \nabla_Y X\right)$.  Note $\tilde{g}$ and $g$ and have to agree on $T_pM$. Also note that $[\tilde{X}, \tilde{Y}](\tilde{f}) = [X, Y](f)$ since the extensions must agree on some neighborhood of $p$.
      \[
        \begin{aligned}
          g(Z, \nabla_Y X) &=  g\left(Z, \text{pr} \left( \tilde{\nabla}_{\tilde{Y}} \tilde{X} \right)\right) \\
                           &=  \tilde{g} \left(\tilde{Z}, \tilde{\nabla}_{\tilde{Y}} \tilde{X} \right) \\
                           &= \frac{1}{2}\left( \tilde{X}( \tilde{g}( \tilde{Y}, \tilde{Z})) + \tilde{Y}(\tilde{g}(\tilde{Z},\tilde{X})) - \tilde{Z}(\tilde{g}(\tilde{X}, \tilde{Y})) - \tilde{g}([\tilde{X}, \tilde{Z}], \tilde{Y}) - g([\tilde{Y}, \tilde{Z}], \tilde{X}) - g([\tilde{X}, \tilde{Y}], \tilde{Z}) \right) \\
                           &= \frac{1}{2}\left( X( g( Y, Z)) + Y(g(Z,X)) - Z(g(X, Y)) - g([X, Z], Y) - g([Y, Z], X) - g([X, Y], Z) \right)
        \end{aligned}
      \]
      Therefore, $\nabla$ is the Levi-Civita connection on $M$.
    \item Let $\gamma: [a,b] \rightarrow S^2$ be an arc of a great circle. It suffices to show that
      \[
        \frac{D}{dt} \left( \dv{\gamma}{t} \right) = \nabla_{\dv{\gamma}{t}} \tilde{\dv{\gamma}{t}} = 0
      \]
  \end{enumerate}
\end{proof}

\end{document}
